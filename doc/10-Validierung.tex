\section{Validierung}

Die detailierten Anforderungen sind im Kapitel \ref{appendix:user-stories} verfügbar.

\begin{longtable}{l l p{10cm} l}
	\hline
	\textbf{Nr.} & \textbf{Prio.} & \textbf{Beschreibung} & \textbf{Status} \\
	\hline
	\textbf{1} & & \textbf{Allgemein} & \\
	\hline
	1.1 & M & Die erkannte Fläche wird angezeigt. Durch antippen der Fläche wird diese augewählt. Die Position der neu zu setzenden Kugelbahn orientiert sich an der ausgewählten Fläche. & Erfüllt \\
	1.2 & M & Die Kugelbahn wird auf der ausgewählten Fläche angezeigt. Die Kugelbahn befindet sich zental im Kamerabild. Mittels Tap kann die Bahn fixiert werden. & Erfüllt \\
	1.3 & M & Die Bahn ist parallel zur Kamera ausgerichtet und behält dies auch beim drehen des Gerätes bei. & Erfüllt \\
	1.4 & M & Auf dem AR Bildschirm wird oben links in schwarzer Schrift der aktuelle Modul aufgeführt. & Erfüllt \\
	1.5 & S & Diese Funktionalität wurde nicht implementiert. Als möglicher Workaround kann das Modul neu gestartet werden. & Nicht Erfüllt \\
	1.6 & K & Es besteht die Möglichkeit den Moduls im Menu direkt zu wechseln. Beim Wechsel bleibt die aktuelle Szene nicht bestehen und muss somit neu ausgerichtet werden. & Teilweise Erfüllt \\
	\hline
	\textbf{2} & & \textbf{Guide} & \\
	\hline
	2.1 & M & Alle persistierten Bahnen werden die Namen der Kugelbahnen in der View MarbleRuns angezeigt. Diese Bahnen können durch antippen ausgewählt werden. Nach erfolgreicher selektion wird die Bahn der AR View bereitgestellt. & Erfüllt \\
	2.2 & M & Beim Aufbau der Bahn wird das nächte physische zu platzierende Element rot transparent Angezeigt. Die restlichen Elemente werden weiss transparent Angezeigt. Anhand der transparenten Darstellung wird verdeutlicht welches Element für den Aufbau verwendet werden muss und in welcher Position es sich befinden muss  & Erfüllt \\
	2.3 & M & Falls nächste Schritte in der Bauanleitung zur Verfügung stehen, ist auf der AR View unten rechts ein oranger Knopf mit einem Pfeil nach rechts verfügbar. Mit diesem Knopf kann der nächste Schritt eingeleitet werden kann. & Erfüllt \\
	2.4 & S & Falls vorherige Schritte in der Bauanleitung zur Verfügung stehen, ist auf der AR View unten links ein oranger Knopf mit einem Pfeil nach links verfügbar. Mit diesem Knopf kann der vorherige Schritt angezeigt werden. & Erfüllt \\
	2.5 & K & Die Bauanleitung kann über das Menu oben rechts neu gestartet werden. Zuoberst im Menu befindet sich das neustarten des Baumodus unter "'Restart Guide`"'. & Erfüllt \\
	\hline
	\textbf{3} & & \textbf{Builder} & \\
	\hline
	3.1 & M & Zur Auswahl stehen 13 cuboro Elemente. Diese können mittels dem orangen Plus-Knopf in der unteren Mitte selektiert werden. & Erfüllt \\
	3.2 & M & Wird auf dem Startbildschirm der Editor-Modus ausgewählt so wird eine Liste von gespeicherten Bahnen angezeigt. Das erste Element dieser Liste ist ein Plus bei dem eine neue Bahn erstellt werden kann. Beim selektiern einer neuen Bahn kann der Namen der Bahn mittels einem Popover eingegeben werden. Die neue Bahn wird unter dem eingegeben Namen persistiert und steht ab diesem Zeitpunkt in der Liste zur Verfügung. & Erfüllt \\
	3.3 & M & Nach dem Mutieren einer Kugelbahn kann diese über das Menu oben links gespeichert werden. Dies geschieht über den Knopf "'Save`". & Erfüllt \\
	3.4 & M & In der MarbleRuns liste kann eine gespeicherte Bahn ausgewählt werden. Diese steht anschliessend zur positioniertung in der AR View zur verfügung. Sobald die Bahn platziert wurde kann sie beliebig mutiert werden. & Erfüllt \\
	3.5 & M & Valide Position werden nach der Selektion eines Elementes auf der Kugelbahn als Bouding Boxes dargestellt. Die Bounding Boxes werden mittels einem Algorithmus nur an Stellen angezeigt, die auf der ersten Ebene direkt an einem bereits erstellten Element angrenzen oder oberhalb platziert werden können. & Erfüllt \\
	3.6 & M & Ein Element kann mittels dem antippen einer Bounding Box hinzugefügt werden. Es ist nur möglich Element an den Stellen einer Bounding Box hinzuzufügen. & Erfüllt \\
	3.7 & M & Durch das antippen von einem Element kann dieses selektiert werden. Bei erfolgreicher Selektion wird dieses rot transparent hervorgehoben. Anschliessend kann das Element durch Streichgesten in einer der drei Achsen in richtung der Geste um 90° rotiert werden. & Erfüllt \\
	3.8 & M & Durch das antippen und halten kann ein Kugelbahn element entfehrnt werden. Dies kann jedoch nur gemacht wenn die Kugelbahn zusammenhängend bleibt und physikalisch möglich ist. Es ist nicht Möglich element unterhalb von Elementen zu entfehrnen (keine schwebende Elemente). & Erfüllt \\
	3.9 & S & Diese Funktionalität ist nicht implementiert, da die zeitlichen Aufwände andersweitig investiert werden mussten. & Nicht Erfüllt \\
\end{longtable}