\section{Validierung}


\subsection{Funktionale Anforderungen}

Die detaillierten funktionalen Anforderungen sind im Kapitel \ref{appendix:user-stories} aufgeführt.

\begin{longtable}{l l p{10cm} l}
	\hline
	\textbf{Nr.} & \textbf{Prio.} & \textbf{Beschreibung} & \textbf{Status} \\
	\hline
	\textbf{1} & & \textbf{Allgemein} & \\
	\hline
	1.1 & M & Die erkannte Fläche wird mit einem augmentierten Gitternetz angezeigt. Durch Antippen der Fläche wird diese ausgewählt. Die Position der neu zu setzenden Kugelbahn orientiert sich an der ausgewählten Fläche. & Erfüllt \\
	\hline
	1.2 & M & Die Kugelbahn wird auf der ausgewählten Fläche angezeigt. Die Kugelbahn befindet sich zentral im Kamerabild. Mit einem Tap kann die Bahn fixiert werden. & Erfüllt \\
	\hline
	1.3 & M & Die Bahn ist parallel zur Kamera ausgerichtet und behält dies auch beim drehen des Gerätes bei. & Erfüllt \\
	\hline
	1.4 & M & Auf der AR View wird oben links in schwarzer Schrift der aktuelle Modus visualisiert. & Erfüllt \\
	\hline
	1.5 & S & Diese Funktionalität wurde nicht implementiert. Als möglicher Workaround kann der Modus neu gestartet werden. & Nicht Erfüllt \\
	\hline
	1.6 & K & Der Modus kann direkt im Menu gewechselt werden. Beim Wechsel bleibt die aktuelle Szene nicht bestehen und muss somit neu ausgerichtet werden. & Teilweise Erfüllt \\
	\hline
	\textbf{2} & & \textbf{Guide} & \\
	\hline
	2.1 & M & Alle persistierten Bahnen werden mit dem Namen der Kugelbahn in der View MarbleRuns angezeigt. Diese Bahnen können durch Antippen ausgewählt werden. Nach erfolgreicher auswahl einer Bahn wird diese der AR View bereitgestellt. & Erfüllt \\
	\hline
	2.2 & M & Beim Aufbau der Bahn wird das nächte physisch zu platzierende Element rot transparent angezeigt. Die restlichen Elemente sind weiss transparent ersichtlich. Anhand der transparenten Darstellung wird hervorgehoben, welches Element für den Aufbau verwendet werden muss und in welcher Position es sich befinden soll.  & Erfüllt \\
	\hline
	2.3 & M & Falls ein nächster Schritt in der Bauanleitung zur Verfügung stehen, ist auf der AR View unten rechts ein oranger Knopf mit einem Pfeil nach rechts verfügbar. Mit diesem Knopf kann der nächste Schritt angezeigt werden. Sollte ein nächster Schritt verfügbar sein, so wird der Knopf ausgegraut. & Erfüllt \\
	\hline
	2.4 & S & Falls ein vorheriger Schritt in der Bauanleitung zur Verfügung steht, ist auf der AR View unten links ein oranger Knopf mit einem Pfeil nach links verfügbar. Mit diesem Knopf kann der vorherige Schritt angezeigt werden. Sollte kein vorheriger Schritt verfügbar sein, so wird der Knopf ausgegraut. & Erfüllt \\
	\hline
	2.5 & K & Die Bauanleitung kann über das Menu oben rechts neu gestartet werden. Zuoberst im Menu befindet sich der Knopf unter "'Restart Guide`"'. & Erfüllt \\
	\hline
	\textbf{3} & & \textbf{Builder} & \\
	\hline
	3.1 & M & Zur Auswahl stehen 13 cuboro Elemente. Diese können mit dem grünen Plus-Knopf in der unteren Mitte selektiert werden. & Erfüllt \\
	\hline
	3.2 & M & Wird auf dem Startbildschirm der Editor-Modus ausgewählt so wird eine Liste von gespeicherten Bahnen angezeigt. Das erste Element dieser Liste ist ein Plus, bei dem eine neue Bahn erstellt werden kann. Beim erstellen einer neuen Bahn wird der Namen der Bahn mit einem Popover eingegeben. Die neue Bahn wird unter dem eingegeben Namen persistiert und steht ab diesem Zeitpunkt in der Liste zur Verfügung. & Erfüllt \\
	\hline
	3.3 & M & Nach dem bearbeiten einer Kugelbahn, kann diese über das Menu oben links gespeichert werden. Dies geschieht über den Knopf "'Save`". & Erfüllt \\
	\hline
	3.4 & M & In der MarbleRuns-Liste kann eine gespeicherte Bahn ausgewählt werden. Diese steht anschliessend zur Positionierung in der AR View zur verfügung. Sobald die Bahn platziert wurde, kann sie beliebig verändert werden. & Erfüllt \\
	\hline
	3.5 & M & Valide Positionen werden nach der Selektion eines Elementes auf der Kugelbahn als Bounding Boxes dargestellt. Die Bounding Boxes werden mit einem Algorithmus nur an Stellen angezeigt, die auf der ersten Ebene direkt an einem bereits erstellten Element angrenzen oder darauf platziert werden kann. & Erfüllt \\
	\hline
	3.6 & M & Mit dem Antippen einer Bounding Box wird ein Element an der jeweiligen Stelle hinzugefügt. Es ist nur möglich Element an den Stellen einer Bounding Box hinzuzufügen. & Erfüllt \\
	\hline
	3.7 & M & Durch das Antippen von einem Element kann dieses selektiert werden. Bei erfolgreicher Selektion, wird dieses rot transparent hervorgehoben. Anschliessend kann das Element durch Streichgesten in einer der drei Achsen in richtung der Geste um 90° rotiert werden. & Erfüllt \\
	\hline
	3.8 & M & Durch das Antippen und halten kann ein Element entfernt werden. Dies kann jedoch nur gemacht werden, wenn die Kugelbahn zusammenhängend bleibt und physikalisch möglich ist. Es ist somit nicht möglich Elemente unterhalb von Elementen zu entfernen (keine schwebende Elemente). & Erfüllt \\
	\hline
	3.9 & S & Diese Funktionalität ist nicht implementiert, da die zeitlichen Aufwände anderweitig investiert werden mussten. & Nicht Erfüllt \\
	\hline
	\caption{Validierung der funktionalen Anforderungen}
\end{longtable}


\subsection{Nichtfunktionale Anforderungen}


Die detaillierten nichtfunktionale Anforderungen sind im Kapitel \ref{appendix:nichtfunktionale-anforderungen} aufgeführt.

\begin{longtable}{l l l p{10cm}}
	\hline
	\textbf{Nr.} & \textbf{Prio.} & \textbf{Typ} & \textbf{Beschreibung} \\
	\hline
	 & & \textbf{Constraints} & \\
	\hline
	01 & M & Technologie & Die App wurde in Swift 4.1 programmiert. \\
	\hline
	02 & M & Technologie & Die App wurde auf den iPhones 6s und 7 mit iOS 11.3 getestet. \\
	\hline
	03 & M & Technologie & Die App wurde mit ARKit 1.5 entwickelt. \\
	\hline
	04 & M & Geschäftlich & Die App zeigt die technologischen Möglichkeiten auf. \\
	\hline
	 & & \textbf{Qualitäten}& \\
	\hline
	05 & M & Laufzeit & Farben wurden nur unterstützend verwendet. Die Interaktionen sind unmissverständlich mit Symbolen oder Beschriftung versehen. \\ 
	\hline
	06 & M & Laufzeit & Der Benutzer wird aktuell nicht über den aktuellen Trackingstatus informiert. \\
	\hline
	07 & M & Laufzeit & Die Bildwiederholungsfrequenz mit 30 Bilder pro Sekunde kann bei 20 Elementen garantiert werden und wurde getestet. \\
	\hline
	08 & M & Laufzeit & Die Persistierung der Kugelbahnen erfolgt auf dem Gerät. \\
	\hline
	09 & M & Compilier-Zeit & Der Code wurde möglichst klar und sauber geschrieben. \\
	\hline
	10 & M & Compilier-Zeit & Schlüsselstellen im Code sind in der Dokumentation festgehalten. Zusätzlich wurde sporadisch auf kurze Kommentare zurückgegriffen, die bei einer Code passage erklären was passiert. \\
	\hline
	\caption{Validierung der nichtfunktionalen Anforderungen}
\end{longtable}


\subsection{Technisches Fazit}
Bei der Umsetzung der App wurde zuerst die Architektur diskutiert und erstellt. Dies hat anfangs zwar Zeit gekostet, konnte aber im Verlauf des Projektes seinen Mehrwert präsentieren. Die VIPER Architektur hat mit seiner losen Koppelung und klaren Modularisierung überzeugt. Diese Modularisierung verhilft ein schnelles einarbeiten in das Projekt und eine einfache Möglichkeit dieses um neue Funktionalitäten zu erweitern. Der Editiermodus zur Erstellung und Modifikation, sowie die interaktive Aufbauanleitung von virtuellen Kugelbahnen konnte realisiert werden.

% AR Experience wird unterbrochen
Bei der Auswahl des nächsten Elements für die virtuelle Kugelbahn wird das AR Erlebnis unterbrochen. Hierbei benötigt das Framework einige Sekunden die Bahn wieder richtig zu Positionieren.

% Anforderungen eingehen
Die Erkennung dreidimensionaler Objekte mit den vorgeschlagenen Frameworks konnte nicht erfolgreich implementiert werden und wurde in der Umsetzung der Demo-App nicht weiter berücksichtigt. Die Demo-App erfüllt alle Muss-Anforderungen und enthält einen Baumodus sowie eine interaktive Aufbauanleitung.
