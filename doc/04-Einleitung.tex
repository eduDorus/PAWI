\section{Einleitung}

In dieser Arbeit sollen die technischen Grenzen von Augmented Reality mit iOS Frameworks aufgezeigt werden. Die Projektmethodik findet im explorativem Verfahren statt. 
Die Arbeit setzt sich zuerst mit der Problemstellung, Ziele und Abgrenzung auseinander. Anschliessend werden die für diese Arbeit relevanten Themen beschrieben und den Stand der Technik in deren Rubrik aufgeführt. Im nächsten Kapitel werden die iOS Frameworks CoreML, Vision, ARKit etc. veranschaulicht. Nach der theoretischen Einleitung werden im Kapitel Lösungsentwicklung die Prototypen vorgestellt, die in der explorativen Phase erstellt wurden. Nach einer Lösungswahl wird im Kapitel Umsetzung ist die erarbeitete Demo-App detailliert dokumentiert. Im anschliessenden Kapitel Validierung wird die erarbeitete Lösung mit dem Anforderungskatalog verglichen und ein technisches Fazit gezogen. Im letzten Kapitel Schlussfolgerung werden die Lessons Learned, persönliche Fazit und ein Ausblick auf weitere Arbeiten gegeben.
% Forschungskonzept, Fragestellung, Method, Aufbau der Arbeit
