\section{Einleitung}

Mobile Plattformen entwickeln sich stetig weiter und bieten auf der Hardware-Seite mit schnelleren Chips, besseren Sensoren (z.B. Kameras, Gyroskopen usw.) neue Möglichkeiten. Dazu folgen auf der Software-Seite mit neuen Anwendungen und erweiterten Programmierschnittstellen für die Endanwender bzw. die Software-Entwickler. Für letztere hat beispielsweise Apple an der letztjährigen Entwicklerkonferenz WWDC 2017 mit Core ML, Vision und dem ARKit drei neue Frameworks für maschinelles Lernen, Computer Vision und Augmented Reality vorgestellt für das mobile Betriebssystem iOS.


Die Forschungsgruppe Algorithmic Business (ABIZ) der Hochschule Luzern - Informatik hat verschiedene Forschungsprojekte und viel Know-How im Umfeld von maschinellem Lernen, Computer Vision und Augmented Reality. Im Rahmen von diesem Projekt sollen die Möglichkeiten von den drei oben genannten (und allenfalls weitern) iOS-Frameworks an einer exemplarischen Anwendung ausgelotet und dann in Form einer attraktiven Demo-App illustriert werden.


Die vorgegebene Anwendungsdomäne sind Holzkugelbahnen, ABIZ stellt dazu den Studierenden entsprechende Bausätze der Marke Cuboro (https://cuboro.ch/) zur Verfügung. Diese Firma stellt unter http://www.cuboro-webkit.ch/ ebenfalls virtuelle Kugelbahnen inkl. Ball-Simulationen zur Verfügung, evtl. kann daran angeknüpft oder aufgebaut werden. Die zu erstellende App soll in einer Kombination von Objekt-Erkennung mit AR/VR eine physisch aufgebaute Kugelbahn auch digital erfassen, diese virtuell darstellen können (VR) und auch das physische Modell durch Überlagerungen anreichern (AR).