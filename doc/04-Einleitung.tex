\section{Einleitung}



%Mobile Plattformen entwickeln sich stetig weiter und bieten auf der Hardware-Seite mit schnelleren Chips, besseren Sensoren (z.B. Kameras, Gyroskopen usw.) neue Möglichkeiten. Dazu folgen auf der Software-Seite mit neuen Anwendungen und erweiterten Programmierschnittstellen für die Endanwender bzw. die Software-Entwickler. Für letztere hat beispielsweise Apple an der letztjährigen Entwicklerkonferenz WWDC 2017 mit CoreML, Vision und dem ARKit drei neue Frameworks für maschinelles Lernen, Computer Vision und Augmented Reality vorgestellt für das mobile Betriebssystem iOS.

Die grössten Tech-Unternehmen probieren Jahr für Jahr die best mögliche Plattform für Benutzer bereitzustellen. Vorallem die mobilen Plattformen haben in den letzten Jahren, mit der beliebtheit von Smartphones, an potenzial gewonnen. Smartphones bieten immer schnellere Prozessoren und bessere Sensoren. Zusätzlich werden den Entwicklern stetig bessere Frameworks zur Verfügung gestellt, die den Entwicklungsprozess unterstützen und neue Möglichkeiten bieten. Apple hat an der letztjährigen Entwicklerkonferenz WWDC 2017 mit CoreML, Vision und dem ARKit drei neue Frameworks für maschinelles Lernen, Computer Vision und Augmented Reality vorgestellt.


%Die Forschungsgruppe Algorithmic Business (ABIZ) der Hochschule Luzern - Informatik hat verschiedene Forschungsprojekte und viel Know-How im Umfeld von maschinellem Lernen, Computer Vision und Augmented Reality. Im Rahmen von diesem Projekt sollen die Möglichkeiten von den drei oben genannten (und allenfalls weitern) iOS-Frameworks an einer exemplarischen Anwendung ausgelotet und dann in Form einer attraktiven Demo-App illustriert werden.

% TODO: Erster Satz wurde übernommen
Die Forschungsgruppe Algorithmic Business (ABIZ) der Hochschule Luzern - Informatik hat verschiedene Forschungsprojekte und viel Know-How im Umfeld von maschinellem Lernen, Computer Vision und Augmented Reality. In dieser Industriearbeit (PAWI) sollen die Möglichkeiten dieser neuen iOS-Frameworks aufgezeigt werden. Zur Veranschaulichung dieser Technologien soll eine attraktive Demo-App entwickelt werden. 

%Die vorgegebene Anwendungsdomäne sind Holzkugelbahnen, ABIZ stellt dazu den Studierenden entsprechende Bausätze der Marke Cuboro zur Verfügung. Diese Firma stellt ebenfalls virtuelle Kugelbahnen inkl. Ball-Simulationen zur Verfügung, evtl. kann daran angeknüpft oder aufgebaut werden. Die zu erstellende App soll in einer Kombination von Objekt-Erkennung mit AR/VR eine physisch aufgebaute Kugelbahn auch digital erfassen, diese virtuell darstellen können (VR) und auch das physische Modell durch Überlagerungen anreichern (AR).

Die vorgegebene Anwendungsdomäne sind Holzkugelbahnen, wobei die Forschungsgruppe entsprechende Bausätze der Marke cuboro zur Verfügung stellt. Das Unternehmen cuboro arbeitet ebenfalls an Softwarelösungen die es ermöglichen mit virtuellen Kugelbahnen zu interagieren. Die Demo-App soll aufzeigen, dass es möglich ist physisch aufgebaute Bahnen digital zu erfassen und das physische Modell durch Überlagerungen anzureichern (AR).