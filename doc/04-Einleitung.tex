\section{Einleitung}

In dieser Projektarbeit sollen die technischen Grenzen von Augmented Reality mit den iOS Frameworks wie ARKit, Vision, CoreML etc. aufgezeigt werden. Die Projektmethodik findet im explorativem Verfahren statt. 
Die Dokumentation veranschaulicht zuerst die Problemstellung, Ziele und Abgrenzung des Projekts. Anschliessend werden die für diese Arbeit relevanten Themen beschrieben sowie den aktuellen Stand der Technik. Im anschliessenden Kapitel werden die iOS Frameworks CoreML, Vision, ARKit etc. veranschaulicht. Nach der theoretischen Einleitung werden im Kapitel Lösungsentwicklung die Prototypen vorgestellt, die in der explorativen Phase erstellt wurden. Nach einer Lösungswahl wird im Kapitel Umsetzung ist die erarbeitete Demo-App detailliert dokumentiert. Im nächsten Kapitel Validierung wird die erarbeitete Lösung mit dem Anforderungskatalog verglichen und ein technisches Fazit gezogen. Im letzten Kapitel Schlussfolgerung werden die Lessons Learned, persönliche Fazit und ein Ausblick auf weitere Arbeiten gegeben.
% Forschungskonzept, Fragestellung, Method, Aufbau der Arbeit
