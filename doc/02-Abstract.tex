\section*{Abstract}

\subsection*{Deutsch}

% Relevanz
Anlässlich der Entwicklerkonferenz WWDC 2017 hat der Technologiekonzern Apple neue Frameworks zur Nutzung von Augmented Reality, Machine Learning und Computer Vision auf mobilen Geräten veröffentlicht.
Die Forschungsgruppe Algorithmic Business (ABIZ) der Hochschule Luzern - Informatik hat verschiedene Forschungsprojekte und viel Know-How in diesen Bereichen.
% Thema / Problemstellung
Am Beispiel von Holzkugelbahnen soll aufgezeigt werden, welche Möglichkeiten die neuen Frameworks von Apple bieten.
% Vorgehen Methode
Diese Arbeit erforscht Zusammenspiel und Grenzen dieser Technologien mit dem Bau von Prototypen und veranschaulicht die Ergebnisse in einer Demo-App. 
% Ergebnisse
Anhand den Prototypen konnten die Limitationen der Frameworks aufgezeigt werden.
Mit der iOS Demo-App "`ARMarbleRun"' können durch die Nutzung von ARKit 1.5 virtuelle Kugelbahnen in Augmented Reality erstellt und bearbeitet werden.
Weiter kann eine physische Kugelbahn mit Hilfe einer virtuellen Bauanleitung aufgebaut werden.
% Folgerungen
Die Demo-App demonstriert, dass ein produktiver Einsatz von ARKit durchaus möglich ist, aber im Funktionsumfang noch limitiert ist.
Eine direkte Interaktion mit dreidimensionalen Objekten ist mit dem verwendeten ARKit 1.5 nicht gegeben.

\subsection*{English}

At the developer conference WWDC 2017, technology company Apple released new frameworks for the use of augmented reality, machine learning and computer vision on mobile devices.
The research group Algorithmic Business (ABIZ) of the Lucerne School of Information Technology has various research projects and a lot of know-how in these areas.
The possibilities offered by the new frameworks from Apple should be demonstrated by the example of wooden marble runs.
This work explores the interaction and limitations of these technologies with the creation of prototypes and illustrates the results in a demo app.
Based on the prototypes, the boundaries of the frameworks could be demonstrated.
The iOS demo app ``ARMarbleRun'' can be used to create and edit virtual marble runs in augmented reality.
Furthermore, a physical marble run can be built with the aid of a virtual guide.
The demo app demonstrates that a productive use of ARKit is possible, but its amount of functions is still limited.
A direct interaction with three-dimensional objects is not given with ARKit 1.5.
