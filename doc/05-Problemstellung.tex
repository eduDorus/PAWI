\section{Problemstellung}
In diesem Projekt soll eine iOS-App erstellt werden, welche es erlaubt mit einer physischen Kugelbahn zu interagieren. Einerseits soll es die zu erstellende App erlauben, physische Kugelbahnen digital zu erfassen. Diese Erfassung kann beispielsweise parallel zum Aufbau einer Kugelbahn geschehen, indem die App Bauteil um Bauteil diese zuerst erkennt (mit passenden Methoden aus dem Bereich Bilderkennung/Computer Vision bzw. dem maschinellen Lernen) und dann dessen relative Position innerhalb von der Kugelbahn aufzeichnet. Es sollen zwingend mehrere unterschiedliche Kugelbahnen in der App erfasst sein, ggf. werden diese in der Software fix hinterlegt bzw. können auf einem anderen Weg als mithilfe der App erfasst werden, z.B. durch einen Import aus dem Cuboro-Webkit.


Erfasste Kugelbahnen sollen nach Möglichkeit in AR-Manier als Echtzeit-Überblendung auf Kamerabildern der physischen Kugelbahn projiziert werden können, z.B. indem Bauteile eingefärbt werden, in Form von angezeigten "`Bounding-Boxes"' oder indem virtuelle Kugeln auf der physischen Kugelbahn simuliert werden können. Oder weiter könnten erfasste Kugelbahnen mittels AR beispielsweise auf einen physischen Tisch "`gestellt"' werden und darauf könnte das Rollen von virtuellen Kugeln simuliert werden. 


Das Hauptziel ist es, dass am Schluss eine attraktive und lauffähige Demo-App zur Verfügung steht. Die App soll möglichst intuitiv angewendet werden können und schön die Möglichkeiten (und Grenzen) dieser neuen Technologien illustrieren.


In dieser Arbeit gibt es einige Freiheitsgrade, sprich viele funktionale und technische Aspekte gilt es zu erarbeiten, evaluieren und fest zu legen. Damit das Endresultat den Erwartungen von Auftraggeber und Betreuer entspricht, sollen Evaluationen und Entscheidungsfindungen in enger Rücksprache mit dem Auftraggeber und dem Betreuer stattfinden. Wichtige Entscheide (wie z.B. Fixierung von funktionalen Anforderungen, Wahl von Frameworks, usw.) müssen vom Auftraggeber bzw. dem Betreuer genehmigt werden, entsprechend sollen diese möglichst proaktiv in Evaluationen, Bau von Prototypen usw. miteinbezogen, sowie über Vor- und Nachteile, Alternativen usw. informiert werden.

\subsection{Ausgangslage}
\subsection{Ziele}
\subsection{Abgrenzung (Scope)}
\subsection{Anforderungen}
\subsection{Lösungsskizze}
