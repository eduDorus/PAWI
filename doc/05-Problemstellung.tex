\section{Problemstellung}

\subsection{Ausgangslage}

Die grössten Tech-Unternehmen probieren Jahr für Jahr die best mögliche Plattform für Benutzer bereitzustellen. Vor allem die mobilen Plattformen haben in den letzten Jahren, mit der Beliebtheit von Smartphones, an Potenzial gewonnen. Smartphones bieten immer schnellere Prozessoren und bessere Sensoren. Zusätzlich werden den Entwicklern stetig bessere Frameworks zur Verfügung gestellt die den Entwicklungsprozess unterstützen und neue Möglichkeiten bieten. Apple hat an der Entwicklerkonferenz WWDC 2017 mit Core ML, Vision und dem ARKit drei neue Frameworks für maschinelles Lernen, Computer Vision und Augmented Reality vorgestellt.


Die Forschungsgruppe Algorithmic Business (ABIZ) der Hochschule Luzern - Informatik hat verschiedene Forschungsprojekte und viel Know-How im Umfeld von maschinellem Lernen, Computer Vision und Augmented Reality. In dieser Industriearbeit (PAWI) sollen die Möglichkeiten der oben erwähnten iOS-Frameworks aufgezeigt werden. Zur Veranschaulichung dieser neuen Technologien soll eine attraktive Demo-App entwickelt werden. 


Die vorgegebene Anwendungsdomäne sind Holzkugelbahnen, wobei die Forschunsgruppe entsprechende Bausätze der Marke cuboro zur Verfügung stellt. Das Unternehmen cuboro arbeitet ebenfalls an Softwarelösungen, die es ermöglichen mit virtuellen Kugelbahnen zu interagieren. Die Demo-App soll aufzeigen, dass es möglich ist physisch aufgebaute Bahnen digital zu erfassen und mit virtuellen Inhalten zu erweitern.


\subsection{Ziele}
% Was soll mit dem Projekt erreicht werden? Was soll nach dem Projekt „besser“ sein als vorher?
Das Hauptziel besteht darin am Ende der Projektarbeit eine attraktive und lauffähige Demo-App zu haben. Die App soll möglichst intuitiv bedienbar sein und die Möglichkeiten und Grenzen dieser neuen Technologien illustrieren. Weitere Ziele sind das erfolgreiche Einarbeiten in die neue Programmiersprache Swift und den iOS Frameworks Core ML, Vision und ARKit. Die Frameworks sollen mit dem Bau von Prototypen genauer auf einzelne Aspekte geprüft werden.

\subsection{Abgrenzung (Scope)}
% Welche Sachverhalte sind nicht Bestandteil des Projektes. Welche Restriktionen gibt es im Projekt?
Nicht Bestandteil des Projektes sind Systemkomponenten ausserhalb von iOS. Die iOS-App wird ausschliesslich mit Swift entwickelt. Es werden ausschliesslich native iOS UI Komponente für die Applikation verwendet.

\subsection{Resultate}
Alle erarbeiteten Resultate sind in einem Git Repository auf GitHub unter \url{https://github.com/eduDorus/PAWI} abgelegt. Im Rahmen dieser Projektarbeit werden folgende Resultate erarbeitet. 
 
\begin{longtable}{l p{12.5cm}}
	\hline
	\textbf{Resultat} & \textbf{Beschreibung} \\
	\hline
	Dokumentation & Beinhaltet die vollständige Dokumentation (Wissensaneignung, erstellte Prototypen, Umsetzung der Demo-App, Fazit, Projektmanagement, Architekturdokumentation) \\
	Sourcecode & Xcode Projekte der Prototypen und Demo-App \\
	Demo-App & Installationsdatei der Demo-App \\
	Demo Video & Kurzes Demo Video zur veranschaulichung der Lösung \\
	\hline
	\caption{Auflistung der Resultate}
	\label{tab:meilensteine}
\end{longtable}

\subsection{Grobplan/Meilensteine}

Wie in der Abbildung \ref{fig:timeline} ersichtlich, hat das Projekt fünf Meilensteine und ist in vier Phasen unterteilt. Im Zweiwochenrhythmus wurde jeweils eine Sitzung mit dem betreuenden Dozenten abgehalten. Die detaillierten Meilenstein Ziele sind in der Übersicht \ref{tab:meilensteine} aufgelistet.

\bild{1}{timeline}{Meilenstein Planung}

\begin{longtable}{l l l}
	\hline
	\textbf{MS} & \textbf{Datum} & \textbf{Inhalt} \\
	\hline

	1	& 27. Februar 	& 
	\begin{tabular}[t]{@{} l @{}}
		\tabitem Kick-Off Meeting \\
		\tabitem Besprechung von Aufgabenstellung \\
		\tabitem Projektvorgehen und Meetings besprechen \\
	\end{tabular} \\
	\hline

	2	& 13. März 		& 
	\begin{tabular}[t]{@{} l @{}}
		\tabitem Projekt- und Risikomanagement \\
		\tabitem Grobplanung Projektablauf (Meilensteine) \\
		\tabitem Grobkonzept (grobe Erfassung Fragestellungen, Scope, Anforderungen) \\
		\tabitem Grundaufbau Dokumentation \\
	\end{tabular} \\
	\hline

	3	& 8.  Mai 		& 
	\begin{tabular}[t]{@{} l @{}}
		\tabitem Besprechung Ideenfindung \\
		\tabitem Recherche von Swift und den relevanten Frameworks \\
		\tabitem Technische Grenzen definiert \\
		\tabitem Mehrere Problemlösungszyklen durchlaufen \\
		\tabitem Prototypen zu den Problemlösungszyklen \\
		\tabitem (funktionale) Anforderungen festgelegt \\
		\tabitem Planung der Umsetzung, Arbeitspakete \\
	\end{tabular} \\
	\hline

	4	& 22. Mai 		& 
	\begin{tabular}[t]{@{} l @{}}
		\tabitem Testatsitzung \\
		\tabitem Besprechung erreichter Ziele \\
		\tabitem funktionierende App, die min. alle Muss-Anforderungen erfüllt \\
		\tabitem Entwurf der abzugebenden Dokumentation \\
	\end{tabular} \\
	\hline

	5	& 8.  Juni 		& 
	\begin{tabular}[t]{@{} l @{}}
		\tabitem Abgabe Projekt \\
		\tabitem fertige Dokumentation \\
		\tabitem fertige App \\
		\tabitem Entwurf Schlusspräsentation \\
	\end{tabular} \\

	\hline
	\caption{Auflistung der Meilensteine}
	\label{tab:meilensteine}
\end{longtable}
