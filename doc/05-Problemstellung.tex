\section{Problemstellung}

\subsection{Ausgangslage}
%TODO: Hier müssen wir noch drüber diskutieren. Es fühlt sich irgenndwie nicht richtig an den originalen Auftrag abzuändern oder anders zu spezifizieren. Dieser wird jedoch im Anhang dabei sein wobei es vermutlich kein problem ist.
% Welche Problematik führt zu diesem Projekt? Was ist am derzeitigen Zustand unbefriedigend?
%Für die PAWI Arbeit im Frühlingssemester 2018 soll eine iOS-App erstellt werden, welche es erlaubt mit einer physischen cuboro Kugelbahn zu interagieren. Einerseits soll es die zu erstellende App erlauben, physische Kugelbahnen digital zu erfassen. Diese Erfassung kann beispielsweise parallel zum Aufbau einer Kugelbahn geschehen, indem die App Bauteil um Bauteil diese zuerst erkennt (mit passenden Methoden aus dem Bereich Bilderkennung/Computer Vision bzw. dem maschinellen Lernen) und dann dessen relative Position innerhalb von der Kugelbahn aufzeichnet. Es sollen zwingend mehrere unterschiedliche Kugelbahnen in der App erfasst sein, ggf. werden diese in der Software fix hinterlegt bzw. können auf einem anderen Weg als mithilfe der App erfasst werden, z.B. durch einen Import aus dem Cuboro-Webkit.

Im Industrieprojekt des Frühlingssemesters 2018 soll eine attraktive Demo-App erstellt werden, die mittels Augmented Reality die physischen cuboro Kugelbahn erweitert. Diese Funktionalität wiederspiegelt sich in einer interaktiven Aufbauanleitung und einem Editiermodus.

%Erfasste Kugelbahnen sollen nach Möglichkeit in AR-Manier als Echtzeit-Überblendung auf Kamerabildern der physischen Kugelbahn projiziert werden können, z.B. indem Bauteile eingefärbt werden, in Form von angezeigten "`Bounding-Boxes"' oder indem virtuelle Kugeln auf der physischen Kugelbahn simuliert werden können. Oder weiter könnten erfasste Kugelbahnen mittels AR beispielsweise auf einen physischen Tisch "`gestellt"' werden und darauf könnte das Rollen von virtuellen Kugeln simuliert werden. 

Die interaktive Aufbauanleitung soll es ermöglichen eine virtuell erfasste Bahn in einer Schritt für Schritt Anleitung nachzubauen. Hierbei sollen "`Bounding-Boxes"' verwendet werden, welche die jeweilige Position der hinzuzufügenden Elementen aufzeigt. Hierbei soll eine logisch durchdachte Reihenfolge verwendet werden.

Der Editiermodus soll dass digitalisieren einer physischen Bahn ermöglichen. Diese Bahnen sollen anschliessend auf dem Gerät persistiert werden können. Zusätzlich sollen mit dieser Funktionalität physische Bahnen um neue virtuelle Elemente erweitert werden können.

%In dieser Arbeit gibt es einige Freiheitsgrade, sprich viele funktionale und technische Aspekte gilt es zu erarbeiten, evaluieren und fest zu legen. Damit das Endresultat den Erwartungen von Auftraggeber und Betreuer entspricht, sollen Evaluationen und Entscheidungsfindungen in enger Rücksprache mit dem Auftraggeber und dem Betreuer stattfinden. Wichtige Entscheide (wie z.B. Fixierung von funktionalen Anforderungen, Wahl von Frameworks, usw.) müssen vom Auftraggeber bzw. dem Betreuer genehmigt werden, entsprechend sollen diese möglichst proaktiv in Evaluationen, Bau von Prototypen usw. miteinbezogen, sowie über Vor- und Nachteile, Alternativen usw. informiert werden.

\subsection{Ziele}
% Was soll mit dem Projekt erreicht werden? Was soll nach dem Projekt „besser“ sein als vorher?
Das Hauptziel ist es, dass am Schluss eine attraktive und lauffähige Demo-App zur Verfügung steht. Die App soll möglichst intuitiv angewendet werden können und die Möglichkeiten und Grenzen dieser neuen Technologien illustrieren. Weitere Ziele sind das erfolgreiche Einarbeiten in die neue Programmiersprache Swift und den iOS Frameworks Core ML, Vision und ARKit. Die Frameworks sollen anhand von Problemlösungszyklen genauer auf einzelne Aspekte geprüft werden.

\subsection{Abgrenzung (Scope)}
% Welche Sachverhalte sind nicht Bestandteil des Projektes. Welche Restriktionen gibt es im Projekt?
Nichtbestandteil des Projektes sind Betriebssysteme / Frameworks ausserhalb von iOS. Die iOS-App wird ausschliesslich mit Swift entwickelt. Es werden ausschliesslich native iOS UI Komponente für die Applikation verwendet.
