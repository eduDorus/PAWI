\section{Problemstellung}

\subsection{Ausgangslage}

%Mobile Plattformen entwickeln sich stetig weiter und bieten auf der Hardware-Seite mit schnelleren Chips, besseren Sensoren (z.B. Kameras, Gyroskopen usw.) neue Möglichkeiten. Dazu folgen auf der Software-Seite mit neuen Anwendungen und erweiterten Programmierschnittstellen für die Endanwender bzw. die Software-Entwickler. Für letztere hat beispielsweise Apple an der letztjährigen Entwicklerkonferenz WWDC 2017 mit CoreML, Vision und dem ARKit drei neue Frameworks für maschinelles Lernen, Computer Vision und Augmented Reality vorgestellt für das mobile Betriebssystem iOS.

Die grössten Tech-Unternehmen probieren Jahr für Jahr die best mögliche Plattform für Benutzer bereitzustellen. Vorallem die mobilen Plattformen haben in den letzten Jahren, mit der beliebtheit von Smartphones, an potenzial gewonnen. Smartphones bieten immer schnellere Prozessoren und bessere Sensoren. Zusätzlich werden den Entwicklern stetig bessere Frameworks zur Verfügung gestellt, die den Entwicklungsprozess unterstützen und neue Möglichkeiten bieten. Apple hat an der letztjährigen Entwicklerkonferenz WWDC 2017 mit CoreML, Vision und dem ARKit drei neue Frameworks für maschinelles Lernen, Computer Vision und Augmented Reality vorgestellt.


% TODO: Erster Satz wurde übernommen
Die Forschungsgruppe Algorithmic Business (ABIZ) der Hochschule Luzern - Informatik hat verschiedene Forschungsprojekte und viel Know-How im Umfeld von maschinellem Lernen, Computer Vision und Augmented Reality. In dieser Industriearbeit (PAWI) sollen die Möglichkeiten dieser neuen iOS-Frameworks aufgezeigt werden. Zur Veranschaulichung dieser Technologien soll eine attraktive Demo-App entwickelt werden. 


Die vorgegebene Anwendungsdomäne sind Holzkugelbahnen, wobei die Forschungsgruppe entsprechende Bausätze der Marke cuboro zur Verfügung stellt. Das Unternehmen cuboro arbeitet ebenfalls an Softwarelösungen, die es ermöglichen mit virtuellen Kugelbahnen zu interagieren. Die Demo-App soll aufzeigen, dass es möglich ist physisch aufgebaute Bahnen digital zu erfassen und dass die physischen Elemente durch AR Überlagerungen erweitert werden können.


\subsection{Ziele}
% Was soll mit dem Projekt erreicht werden? Was soll nach dem Projekt „besser“ sein als vorher?
Das Hauptziel ist es, dass am Schluss eine attraktive und lauffähige Demo-App zur Verfügung steht. Die App soll möglichst intuitiv angewendet werden können und die Möglichkeiten und Grenzen dieser neuen Technologien illustrieren. Weitere Ziele sind das erfolgreiche Einarbeiten in die neue Programmiersprache Swift und den iOS Frameworks Core ML, Vision und ARKit. Die Frameworks sollen anhand von Problemlösungszyklen genauer auf einzelne Aspekte geprüft werden.

\subsection{Abgrenzung (Scope)}
% Welche Sachverhalte sind nicht Bestandteil des Projektes. Welche Restriktionen gibt es im Projekt?
Nichtbestandteil des Projektes sind Betriebssysteme / Frameworks ausserhalb von iOS. Die iOS-App wird ausschliesslich mit Swift entwickelt. Es werden ausschliesslich native iOS UI Komponente für die Applikation verwendet.

\subsection{Resultate}
Im Rahmen dieser PAWI werden folgende Resultate erarbeitet:
\begin{itemize}
• Dokumentation
– Fachkonzept
Besteht aus einer Beschreibung des fachlichen Use Cases der Passenger Prediction,
den Theoriegrundlagen der eingesetzten Deep Learning Verfahren, speziell die Entwicklung
von Convolutional Neural Networks, und einem Evaluationsdokument wo
die Lösung mit dem bestehenden Verfahren verglichen wird.
– Software-Design-Konzept
– Benutzerdokumentation
• Sourcecode
• Trainiertes Deep Learning-Modell


1.4 Randbedingungen
Sourcecode Verwaltung
Für die Sourcecode Verwaltung wird von der HSLU eine GitLab Instanz zur Verfügung gestellt.
https://gitlab.enterpriselab.ch
Programmiersprachen
Als Programmiersprache wird grundsätzlich Python 3.X verwendet.
Technologien
Es muss das Machine Learning Framework Tensorflow in der Version 1.4 oder höher benutzt
werden.
Zielsysteme
Der Code muss plattformunabhängig lauffähig sein.
Fremdsoftware
Es darf nur Fremdsoftware genutzt werden, welche eine kommerzielle Nutzung nicht einschränkt
und der entstandene Sourcecode nicht veröffentlicht werden muss.