\section{Stand der Technik}

\subsection{Augmented Reality} \label{sub:augmented-reality}
Um zu Verstehen wobei es sich bei Augemented Reality handelt müssen wir den Ursprung des Wortes analysieren. Da augmented soviel wie hinzufügen oder erweitern bedeutet und reality die physische Wahrnehmung der Umgebung wiedergibt kann das Word auf deutsch als erweiterte Realität bezeichnet werden.

Augmented Reality kann wie folgt beschrieben werden: Eine erweiterte Version der Realität, in der direkte oder indirekte die Ansichten physischer Realitätsumgebungen oder Objekte mit überlagerten computergenerierten Bildern über die Sicht eines Benutzers auf die reale Welt projeziert werden, wodurch die aktuelle Wahrnehmung der Realität erweitert wird. (\cite{reality-technologies})

Die AR Technologie hat viele Anwendungszwecke und es wird in der Zukunft eine grosse Rolle spielen. Anbei einige Beispiele damit der Verwendungszweck der Technologie ein klares Bild erhält:
\begin{itemize}
    \item Unterhaltung: Spiele wie Tetris können auf jeder beliebigen Unterlage gespielt werden und von diversen Blichwinkel erlebt werden.
    \item Gesundheitswesen: Dem Arzt werden vitalwerte des Patienten eingeblendet und informationen zur aktuellen operation sowie Mögliche Schnittmuster etc.
    \item Wartung: Bei der Wartung von speziellen Turbinensystemen wird der Wartungsarbeiter auf die Herangehensweise aufmerksamgemacht und welche Teile in welcher Reinenfolge abmontiert, gewartet und wiedermontiert werden müssen.
    \item Navigation: Der Navigationsweg wird auf die Strasse projeziert, wobei der Lenker sein Blick nicht von der Strasse wenden muss.
    \item Bildung: Der Dozent kann Objekte aus dem Unterrichtsstoff anzeigen und in seine Einzelteile erklären ohne dabei ein physisches Examplar zu besitzen. Dies kann z.B ein menschlicher Körper bei Medizinstudenten sein damit jedes Organ betrachet werden kann.
    \item Einkaufen: Möbel können in der Wohnung virtuell platziert werden. Dies Unterstützt den Käufer beim kauf der Wohnungseinrichtung
\end{itemize}

\subsection{Virtual Reality} \label{sub:virtual-reality}
Eine Unterscheidung muss zu Virtual Reality (VR) gemacht werden. Bei VR wird die gesamte realität mit computergenerierten Bildern wiedergegeben. Dies ist eine hervorragende Technologie wenn es darum geht virtuelle Welten, Filme oder neue Gebäude zu erleben. Eine wesentliche Herausforderung bei VR stellt das Manövrieren dar. Es ist schwierig in einer komplett virtuellen Welt zu agieren, ohne dass in der realen Welt ein Hindernis im weg steht.

\subsection{Computer Vision} \label{sub:computer-vision}
Bei Computer Vision handelt es sich Bild- und Videoerkennung. Aus aufgenommen Bildern sollen Informationen extrahiert werden und zur interpretierung genutzt werden. Diese Interpretierung kann sich in meherern Aufgaben unterscheiden wie z.B das Erkennen von Objekten, das Beschreiben eines Bildes, die Möglichen nächsten Bilder eines Films voraus sagen etc. Computer Vision wird heutzutage in vielen Bereichen genutzt und hat mit Deep Learning neue Massstäbe angenommen. In gewissen Subsets von Bildern können Computer diese Bereits besser Kategorisieren denn Menschen.

\subsection{cuboro Webkit}

\subsection{Swift}
Swift ist die neue Programmiersprache für macOS, iOS, watchOS und tvOS. Die Programmiersprache wurde so aufgebaut, dass sie möglichst intuitive für Entwichler ist aber trotzdem ein mächtiges arsenal an Features besitzt. Swift wird mittels dem LLVM Compiler zu optimiertem nativem Code transformiert. Swift ist ein Open Source Projekt unter Swift.org und kann vom jedem mitgestaltet werden. Für unser Projekt wird die vierte Version von Swift verwendet.

% TODO: kleine Einführung zu Datentypen und Collections

\subsection{Xcode}
% Interface Builder usw.
