\section{Stand der Technik}

\subsection{Augmented Reality} \label{sub:augmented-reality}
Um zu verstehen wobei es sich bei Augmented Reality handelt, müssen wir den Ursprung des Wortes analysieren. Da Augmented soviel wie hinzufügen oder erweitern bedeutet und Reality die physische Wahrnehmung der Umgebung wiedergibt kann das Wort auf Deutsch als erweiterte Realität bezeichnet werden.

Die AR Technologie hat viele Anwendungszwecke.
Mit dem Erscheinen der Frameworks ARKit für iOS und ARCore für Android Smartphones wird die Technologie für den Massenmarkt zugänglich.
Nachfolgend einige Beispiele, um den Verwendungszweck der Technologie zu verdeutlichen:
\begin{itemize}
    \item Unterhaltung: Computerspiele können auf jeder beliebigen Unterlage der realen Welt gespielt werden und von diversen Blickwinkel erlebt werden.
    \item Gesundheitswesen: Dem Arzt werden Vitalwerte des Patienten eingeblendet und Informationen zur aktuellen Operation sowie mögliche Schnittmuster angezeigt.
    \item Wartung: Bei der Wartung von speziellen Turbinensystemen wird der Wartungsarbeiter auf die Herangehensweise aufmerksam gemacht und welche Teile in welcher Reihenfolge abmontiert, gewartet und wiedermontiert werden müssen.
    \item Navigation: Der Navigationsweg wird auf die Strasse projiziert, wobei der Lenker sein Blick nicht von der Strasse wenden muss.
    \item Bildung: Die Lehrperson kann Objekte aus dem Unterrichtsstoff anzeigen und in seinen Einzelteilen erklären ohne dabei ein physisches Exemplar zu besitzen. Dies kann beispielsweise ein menschlicher Körper bei Medizinstudenten sein.
    \item Einkaufen: Möbel können in der Wohnung virtuell platziert werden. Dies unterstützt den Käufer beim Kauf der Wohnungseinrichtung.
\end{itemize}


\subsection{Virtual Reality} \label{sub:virtual-reality}
Bei VR wird im gegensatz zu AR die gesamte Realität mit computergenerierten Bildern wiedergegeben. Dies ist eine hervorragende Technologie wenn es darum geht virtuelle Welten, Filme oder neue Gebäude zu erleben. Es ist jedoch eine Herausforderung in einer komplett virtuellen Welt zu agieren, ohne dass in der realen Welt ein Hindernis im Weg steht.

\subsection{Computer Vision} \label{sub:computer-vision}
Bei Computer Vision handelt es sich Bild- und Videoverarbeitung. Aus Bildern sollen Informationen extrahiert und Interpretiert werden. Diese Interpretierung kann sich in mehreren Aufgaben unterscheiden wie z. B. das Erkennen von Objekten, das Beschreiben eines Bildes oder die möglichen nächsten Bilder eines Films voraus sagen. Computer Vision wird heutzutage in vielen Bereichen genutzt und hat mit Deep Learning (\cite{DBLP:journals/corr/SzegedyVISW15}) neue Massstäbe in der Bilderkennung angenommen.

\subsection{Swift} \label{sub:swift}
Swift wurde anlässlich der WWDC 2014 als neue Programmiersprache für macOS, iOS, watchOS und tvOS vorgestellt. Die Programmiersprache wurde so aufgebaut, dass sie möglichst intuitiv für Entwickler ist, aber trotzdem ein mächtiges Arsenal an Features besitzt. Swift wird mittels dem LLVM Compiler zu optimiertem nativem Code transformiert. Seit Dezember 2015 ist Swift ein Open Source Projekt unter Swift.org und kann vom jedem mitgestaltet werden (\cite{swift-org}). Für das vorliegende Projekt wird die vierte Version von Swift verwendet.

% TODO: kleine Einführung zu Datentypen und Collections

\subsection{Xcode} \label{sub:xcode}
Xcode ist die von Apple zur Verfügung gestellte IDE um iOS oder macOS Software herzustellen. Es bietet Features wie Codevervollständigung, Refactoring Hinweise und einen Interface Builder.

Mit dem Interface Builder kann das UI einer App schnell und sauber erstellt werden. Es werden UI Komponenten für fast jeden erdenklichen Fall angeboten. Diese Komponenten können bequem über Drag und Drop auf dem sogenannten Storyboard platziert werden. Anschliessend besteht die Möglichkeit die neue Komponente genauer zu konfigurieren und mit dem View Controller zu verbinden.
