\subsubsection{Schrittweise augmentierte Bauanleitung einer Kugelbahn}\label{subsub:prot-kugelbahnaufbau}
\begin{description}
	\item[Fragestellung:] Wie kann eine schrittweise Bauanleitung für eine Kugelbahn umgesetzt/augmentiert werden? Es soll Würfel für Würfel den Aufbau einer einfache Kugelbahn auf einer Fläche angezeigt werden.
	\item[Resultat:] 
	\item[Versuchsaufbau:] Dieser Versuch baut direkt auf \ref{subsub:prot-kugelbahn} auf und erweitert diesen. Der View wurde zusätzlich ein Button hinzugefügt, um die verschiedenen Schritte des Aufbaus zu bestätigen. Sobald die Kugelbahn platziert wird, ist der Button aktiviert und die Aufbauanleitung wird mit einem Tap darauf gestartet.

	\textbf{Variante 1: Rekonstruktion der Bahn}\\
	Als erster Versuch soll jede horizontale Ebene der Bahn als Schritt des Aufbauens separat hervorgehoben werden. In der Node Hierarchie gibt es keine Struktur, welche die Positionen der Elemente abbildet. Daher müsste beim Auswählen von bestimmten Würfeln immer durch alle Kindknoten iteriert werden. Beim Start der Bauanleitung werden aber zunächst alle Würfel entfernt. In der Folge wird die Kugelbahn wieder neu aus dem Array von Tuples erstellt. Dabei werden nur die Würfel erstellt, die sich auf der aktuellen Ebene des Aufbaus befinden (Code \ref{code:prot-kugelbahnaufbau-loadcurrentlevel1}, Zeile 3-4). All diese Würfel werden rot hervorgehoben (Zeile 4).

	\begin{code}{prot-kugelbahnaufbau-loadcurrentlevel1}{Hinzufügen der Würfel, die sich auf der aktuellen Aufbau-Ebene befinden}
		private func loadTrackCurrentBuildingLayer() {
			for block in track {
				if block.1 == currentBuildingStep - 1 {
					addCube(x: block.0, y: block.1, z: block.2).set(color: UIColor.red)
				}
			}
		}
	\end{code}

	Ab der zweiten Ebene werden vor dem Hinzufügen alle bisherigen Würfel weiss gefärbt, damit schlussendlich nur die aktuelle Ebene hervorgehoben ist. Dazu wird die \texttt{SCNNode} Methode \texttt{enumerateChildNodes(\_:)} zur Hilfe genommen (Code \ref{code:prot-kugelbahnaufbau-removehighlight1}).

	\begin{code}{prot-kugelbahnaufbau-removehighlight1}{Iteration durch Kindknoten zur Änderung der Farbe}
		private func clearHighlights() {
			enumerateChildNodes { (node, stop) in
				if let cube = node as? BasicCube {
					cube.set(color: UIColor.white)
				}
			}
		}
	\end{code}

	Mit jedem Tap auf den Button wird \texttt{currentBuildingStep} inkrementiert und die beiden Methoden \texttt{clearHighlights()} und \texttt{loadTrackCurrentBuildingLayer()} ausgeführt (Code \ref{code:prot-kugelbahnaufbau-loadcurrentlevel1} und \ref{code:prot-kugelbahnaufbau-removehighlight1}). Ein Ende des Aufbaus gibt es nicht, da die Höhe der Bahn nicht geprüft wird.

	\textbf{Variante 2: Informationen über die Position von Elementen erhalten}\\
	Es wäre eine deutliche Vereinfachung des Vorgehens, wenn man direkt über die Koordinaten auf ein Elemen, einen Würfel, zugreifen könnte. Dazu müssen die Würfel in einer Datenstruktur refernziert werden. Ein dreidimensionales Array ist aufgrund der negativen Koordinatenwerte im momentanen Aufbau nicht geeignet. In diesem Versuch wird probiert dies mit einem Dictionary zu lösen. Hierbei soll ein Koordinaten-Tuple als Schlüssel und die \texttt{BasicCube} als Werte dienen.
	
	Der Schlüssel eines Dictionaries muss das Protokoll \texttt{Hashable} adoptieren, was Swift Tuple nicht machen. Deswegen übernimmt dies ein neues Struct \texttt{Triple<T,U,V>}, mit einem Tuple von drei Werten als Attribut. So kann das Dictionary für die dreidimensionale "`Karte"' der Kugelbahn wie folgt erstellt werden:
	\mint[style=xcode,breaklines]{swift}{var map : [Triple<Int,Int,Int> : BasicCube] = [:]}

	Damit die Operationen an der Karte von übrigen Aktionen auf der gesamten Kugelbahn getrennt ist, verwaltet die Klasse \texttt{TrackMap<E>} das Dictionary als privates Attribut. Sie bietet Methoden an um Elemente hinzuzufügen, zu entfernen und um bestimmte Elemente anhand der Koordinaten und umgekehrt zu erhalten.

	\texttt{MarbleTrack} hält nun eine \texttt{TrackMap<BasicCube>} und nutzt deren Methoden für die Operationen aus der vorherigen Variante 1 (Code \ref{code:prot-kugelbahnaufbau-trackmapoperationen}). Da im Gegensatz zur Iteration über Kindknoten die Elemente im Dictionary bereits als \texttt{BasicCube} Typ definiert sind, entfällt zudem das Downcasting von \texttt{SCNNode} auf \texttt{BasicCube} (wie in Code \ref{code:prot-kugelbahnaufbau-removehighlight1}, Zeile 3 notwendig).

	\begin{code}{prot-kugelbahnaufbau-trackmapoperationen}{Methoden aus Code \ref{code:prot-kugelbahnaufbau-loadcurrentlevel1} und \ref{code:prot-kugelbahnaufbau-removehighlight1} mit \texttt{map : TrackMap<BasicCube>}}
		private func loadTrackCurrentBuildingLayer() {
			map.getElements(atLevel: currentBuildingStep-1).forEach { (_, cube) in
				cube.show()
				cube.set(color: UIColor.red)
			}
		}

		private func clearHighlights() {
			map.forEach { (_, cube) in
				cube.set(color: UIColor.white)
			}
		}
	\end{code}

	Statt dass alle Würfel beim Beginn der Aufbauanleitung entfernt werden, werden sie nun bloss über Verändern der Transparenz ausgeblendet. Der Aufbau der Bahn geht weiterhin Ebene für Ebene.


\end{description}


