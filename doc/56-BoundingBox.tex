\subsubsection{Konzept zur Erkennung von einem physischen Objekt in einer Bounding Box}\label{subsub:prot-boundingbox}
\begin{description}
    \item[Fragestellung:] Wie kann festgestellt werden, ob sich ein physisches Objekt in einer Bounding Box befindet?
	\item[Resultat:] Es wurde ein möglicher Lösungsansatz konzeptionell festgehalten. Es besteht zur Zeit keine eingebaute Funktionalität in ARKit oder SceneKit, die das Erkennen von Objekten ermöglicht. 
    \item[Versuchsaufbau:] Es wurde versucht einen Lösungsansatz zu finden, bei der Elemente in einer augmentierten Bounding Box erkennt werden. Dies hat den Vorteil, dass beim Aufbau der Bahn nicht jedes physisch hinzugefügte Element manuell bestätigt werden muss. 
    Das ARKit 1.5 sowie SceneKit weisen keine Funktionalität auf, welche diesen Prozess vereinfacht oder unterstützt. Wir haben konzeptionell einen Lösungsansatz erstellt bei dem der Hit-Test von ARKit zum Einsatz kommt. 

    Es gibt grundsätzlich zwei mögliche Vorgehensweisen. Eine davon beschränkt sich auf das Hit-Testen des Mittelpunkts der Bounding Box. Dies bedeutet jedoch, dass auch Elemente, die nicht präzise in die Bounding Box gestellt wurden, als korrekt platziert ausgewertet werden. Dies würde kein befriedigendes Ergebnis aufweisen. Der Benutzer würde während dem Ausrichten des Elements bereits auf das nächste verwiesen werden. 
    
    Eine andere Vorgehensweise wäre die oberen vier Eckpunkte der Bounding Box mit Hit-Tests zu versehen. Diese Hit-Tests könnten feststellen, ob sich das Element ganz in der Bounding Box befindet. Der Test könnte regelmässig ausgeführt werden und bei einem positiven Ergebnis zur nächsten Bounding Box wechseln.

    \bild{0.7}{hit-test}{Schematische Darstellung für die Prüfung einer Bounding Box}

\end{description}
