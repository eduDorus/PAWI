\subsubsection{Konzept zur Erkennung von einem physichen Objekt in einer bounding Box}
\begin{description}
    \item[Fragestellung:] Wie kann mittels Hit-Tests festgestellt werden ob sich ein physisches Objekt in einer Bounding Box befindet?
	\item[Resultat:] Es wurden verschiedene Vorgehensweisen dokumentiert. 
    \item[Versuchsaufbau:] Um ein physisches Element in einer augmentierten Bounding Box zu erfassen und sicherzustellen dass sich dieses in der vorgegebenen Bounding Box befindet. 

    Es gibt grundsätzlich zwei mögliche Vorgehensweisen. Eine davon beschränkt sich auf das Hit-Testen des Mittelpunkts des Boxes. Eine andere Vorgehensweise wäre die oberen vier Eckpunkte der Bounding Box mit Hit-Tests zu versehen. Dies könnte auch ermitteln ob sich das Element an der Richtigen stelle in der Box befinden würde.
    \bild{1}{hit-test}{Vorgehensweisen}


\end{description}
