\section*{Begriffe \& Abkürzungen}
\begin{table}
	\begin{tabular}{@{} p{.12\textwidth} p{.85\textwidth} @{}}
		\hline
		\textbf{Begriff} & \textbf{Erklärung} \\
		\hline
		ABIZ	& Forschungsgruppe Algorithmic Business der Hochschule Luzern - Informatik \\
		AR 		& Augmented Reality, computergestützte – primär visuelle – Erweiterung der Realität. Siehe Kapitel \ref{sub:augmented-reality}. \\
		cuboro	& Schweizer Unternehmen, das Kugelbahnen aus Holz herstellt \\
		CV		& Computer Vision, maschinelles Sehen zur Interpretation von Bildern und Erkennung von Objekten. Siehe Kapitel \ref{sub:computer-vision}. \\
		Engine	& Ein Programm, das spezifische komplexe Berechnungen oder Simulationen durchführt \\
		Framework	& Ein Programmiergerüst, das dem Entwickler Funktionen und Strukturen bereitstellt \\
		HSLU	& Hochschule Luzern \\
		IDE 	& Integrated Development Environment unterstützt den Entwickler bei der Softwareherstellung \\
		iOS		& Apples mobiles Betriebssystem für iPhones und iPads \\
		ML		& Machine Learning, maschinelles Lernen und Gewinnung von Wissen aus Erfahrungen und Lerndaten durch den Computer \\
		Swift	& Programmiersprache, von Apple für deren Betriebssysteme (macOS, iOS, tvOS und watchOS). Siehe Kapitel \ref{sub:swift}. \\
		VR		& Virtual Reality, eine interaktive virtuelle Umgebung, die in Echtzeit generiert wird. Siehe Kapitel \ref{sub:virtual-reality}. \\
		WWDC	& Worldwide Developers Conference, jährliche Konferenz von Apple für Software-Entwickler, bei der der Konzern oft neue Produkte vorstellt \\
		Xcode	& Entwicklungsumgebung von Apple zur Entwicklung von Programmen für Apples Betriebssysteme (macOS, iOS, tvOS und watchOS). Siehe Kapitel \ref{sub:xcode}. \\
		\hline
	\end{tabular}
\end{table}

% TODO: Add PAWI	& Was heisst eigentlich PAWI \\
