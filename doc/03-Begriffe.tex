\section*{Begriffe \& Abkürzungen}

\begin{table}[htb!]
	\begin{tabular}{@{} p{.18\textwidth} p{.78\textwidth} @{}}
		\hline
		\textbf{Begriff} & \textbf{Erklärung} \\
		\hline
		ABIZ	& Forschungsgruppe Algorithmic Business der Hochschule Luzern - Informatik \\
		AR 		& Augmented Reality, computergestützte – primär visuelle – Erweiterung der Realität. Siehe Kapitel \ref{sub:augmented-reality}. \\
		Bounding Box & Hüllkörper, ein Quader oder Rechteck, das ein Objekt vollständig umschliesst \\
		cuboro	& Schweizer Unternehmen, das Kugelbahnen aus Holz herstellt \\
		CV		& Computer Vision, maschinelles Sehen zur Interpretation von Bildern und Erkennung von Objekten. Siehe Kapitel \ref{sub:computer-vision}. \\
		Deep Learning & Eine Methodik der Maschine Learning Disziplin \\
		Engine	& Ein Programm, das spezifische komplexe Berechnungen oder Simulationen durchführt \\
		Framework	& Ein Programmiergerüst, das dem Entwickler Funktionen und Strukturen bereitstellt \\
		HSLU	& Hochschule Luzern \\
		IDE 	& Integrated Development Environment, unterstützt den Entwickler bei der Softwareentwicklung \\
		Interface Builder & Grafischer Editor in Xcode zur Erstellung von Benutzeroberflächen \\
		iOS		& Apples mobiles Betriebssystem für iPhone und iPad \\
		ML		& Machine Learning, maschinelles Lernen und Gewinnung von Wissen aus Erfahrungen und Lerndaten durch den Computer \\
		PAWI	& Informatikprojekt and der HSLU Informatik \\
		Storyboard & Dateiformat für iOS Benutzeroberflächen, können im Interface Builder grafisch editiert werden \\
		Swift	& Programmiersprache, Open Source, von Apple für deren Betriebssysteme. Siehe Kapitel \ref{sub:swift}. \\
		Tracking & Verfolgung der Bewegung und Position eines Objektes \\
		UI & Grafische Benutzeroberfläche. \\
		VR		& Virtual Reality, eine interaktive virtuelle Umgebung, die in Echtzeit generiert wird. Siehe Kapitel \ref{sub:virtual-reality}. \\
		WWDC	& Worldwide Developers Conference, jährliche Konferenz von Apple für Software-Entwickler, bei der der Konzern oft neue Produkte vorstellt \\
		Xcode	& Entwicklungsumgebung von Apple zur Entwicklung von Programmen für Apples Betriebssysteme. Siehe Kapitel \ref{sub:xcode}. \\
		\hline
	\end{tabular}
	\caption{Übersicht Begriffe \& Abkürzungen}
\end{table}
