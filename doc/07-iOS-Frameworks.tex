\section{iOS Frameworks}
In den folgenden Abschnitten wird ein Überblick über die wesentlichen iOS Frameworks gegeben, die für dieses Projekt relevant sind. Alle Informationen stammen, sofern nicht anders angegeben, aus den offizellen Dokumentationen von Apple. % TODO: Quellen hier einsetzen

\subsection{Core ML}
https://developer.apple.com/documentation/coreml

\subsection{Vision}
https://developer.apple.com/documentation/vision

\subsection{ARKit}
https://developer.apple.com/documentation/arkit

Zusammen mit iOS 11 präsentierte Apple im Juni 2017 ARKit als neues Framework für Augmented Reality Anwendungen, das alle iOS Geräten mit mindestens einem Apple A9 Prozessor unterstützt. Für das World Tracking nutzt ARKit die "`visual-inertial odometry"' Technik. Dabei werden Informationen aus den Bewegungssensoren mit denen aus den Kamerabildern kombiniert. Das Framework errechnet damit ein Modell der realen Welt und die Position und Ausrichtung des Geräts. Um in diesem Modell virtuelle Objekte zu platzieren bietet ARKit einerseits \texttt{ARHitTestResult} um Punkte auf dem Kamerabild einer Oberfläche/Stelle der realen Welt zuordnen, und andererseits \texttt{planeDetection}, das ebene Flächen sucht. Ab ARKit Version 1.5 werden neben horizontalen auch vertikale Flächen erkannt und die Geometrie der Flächen wird statt nur rechteckig neu polygonal angegeben. Die eigentliche Beschreibung und Konstruktion virtueller Objekte wird von den Frameworks SceneKit und SpriteKit übernommen.

\subsection{SpriteKit}
https://developer.apple.com/documentation/spritekit

SpriteKit wurde mit iOS 7 eingeführt und bietet im Wesentlichen Werkzeuge für 2D Animationen. Es umfasst zudem eine Physik-Engine und Eventhandling, sodass es für Spiele genutzt werden kann.

\subsection{SceneKit}
https://developer.apple.com/documentation/scenekit
