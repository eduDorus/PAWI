\section{iOS Frameworks}
In den folgenden Abschnitten wird ein Überblick über die wesentlichen iOS Frameworks gegeben, die für dieses Projekt relevant sind. Alle Informationen stammen, sofern nicht anders angegeben, aus den offiziellen Dokumentationen von Apple. (\cite{apple-developer-doku})


\subsection{Core ML} \label{subsub:core-ml}
Mit dem Aufschwung von neuronalen Netzen, insbesondere Deep Learning, wollen Entwickler diese Technologie in ihren Apps nutzen. Mit dem Core ML Framework gibt Apple den Entwicklern diese Möglichkeit und erlaubt es selbst trainierte Netze in Apps zu integrieren. Core ML unterstützt die Frameworks Vision zur Bildverarbeitung, Foundation zum Verarbeiten von Text und GameplayKit zum Erstellen von Entscheidungsbäumen. (\cite{core-ml})


\subsection{Vision}
Apple bietet mit dem Vision Framework eine umfangreiche Lösung für Computer Vision Probleme. Es enthält Gesichts-, Text-, Barcodeerkennung und Feature Tracking. Zusätzlich kann das im Abschnitt \ref{subsub:core-ml} beschriebene Core ML verwendet werden, um spezifische Probleme mit neuronalen Netzen zu lösen. Die Integration mit dem Core ML Framework ist mit nur wenigen Zeilen Code möglich. (\cite{vision})

\subsection{ARKit}

Zusammen mit iOS 11 präsentierte Apple im Juni 2017 ARKit als neues Framework für Augmented Reality Anwendungen, das alle iOS Geräten mit mindestens einem Apple A9 Prozessor unterstützt. Für das World Tracking nutzt ARKit die "`visual-inertial odometry"' (\cite{arkit-world-tracking}) Technik. Dabei werden Informationen aus den Bewegungssensoren mit denen aus den Kamerabildern kombiniert. Das Framework errechnet damit ein Modell der realen Welt und die Position und Ausrichtung des Geräts. Um in diesem Modell virtuelle Objekte zu platzieren bietet ARKit einerseits Hit-Tests um Punkte auf dem Kamerabild einer Oberfläche/Stelle der realen Welt zuzuordnen, und andererseits die Erkennung ebener Flächen. Ab ARKit Version 1.5 werden neben horizontalen auch vertikale Flächen erkannt und die Geometrie der Flächen wird statt nur rechteckig neu auch polygonal angegeben. Die eigentliche Beschreibung und Konstruktion virtueller Objekte wird von den Frameworks SpriteKit, SceneKit oder Metal übernommen.

\subsubsection{Sessions}
Die Klasse \texttt{ARSession} koordiniert die wichtigsten Bestandteile der ARKit Funktionalität, darunter die Kamera und die Bewegungssensoren aber auch die Bildverarbeitung. Um das ARKit zu nutzen wird mindestens eine \texttt{ARSession} benötigt, die beiden ViewController \texttt{ARSCNView} und \texttt{ARSKView} beinhalten gleich eine Session-Instanz.

Verschiedene Eigenschaften der Session können konfiguriert werden: wird nur das Drehen (die Ausrichtung), nicht aber die Position des Geräts berücksichtigt? Werden die Koordinaten nach der ursprünglichen oder aktuellen Lage des Gerät oder gar nach dem Kompass ausgerichtet? Wird die Frontkamera und somit Gesichtserkennung verwendet?

Bei der standardmässigen Verwendung von \texttt{ARWorldTrackingConfiguration} als Session Konfiguration wird die Position und Ausrichtung des Geräts beim Start der Session als Nullpunkt des dreidimensionalen Koordinatensystems definiert ("`World Coordinate Space"'). Abbildung \ref{fig:arkit-worldalignment-gravity} aus der Apple Dokumentation zeigt diese Einstellung. Es entspricht der manuellen Konfiguration der Session mit Gravity als World Alignment.

\bild[https://developer.apple.com/documentation/arkit/arconfiguration.worldalignment/2873778-gravity]{0.4}{arkit-worldalignment-gravity}{AR-Koordinatensystem mit Gravity als World Alignment}

\subsubsection{Anker}
Dem Modell können sogenannte Anker vom Typ \texttt{ARAnchor} hinzugefügt werden, die genutzt werden können, um Objekte zu platzieren. Wenn die Flächendetektion mit \texttt{planeDetection} aktiviert ist, fügt ARKit der Session automatisch \texttt{ARPlaneAnchor} hinzu. Bei der Gesichtserkennung werden \texttt{ARFaceAnchor} verwendet und bei Bilderkennung \texttt{ARImageAnchor}.

\subsection{SpriteKit}
SpriteKit wurde mit iOS 7 eingeführt und bietet im Wesentlichen Werkzeuge für 2D Animationen. Es umfasst zudem eine Physik-Engine und Eventhandling, sodass es für Spiele genutzt werden kann.


\subsection{SceneKit} \label{sub:scene-kit}
Auf SpriteKit folgend, fügte Apple in iOS 8 mit SceneKit ein high-level Framework für 3D Grafiken hinzu. Das Framework war zuvor bereits in macOS im Einsatz. Wie SpriteKit beinhaltet es eine Physik-Engine, Eventhandling und ein Partikelsystem. Szenen können mit dreidimensionalen Geometrien, Materialien, Lichtern, Animationen und Kameras beschrieben werden. Die Elemente werden in \texttt{SCNScene} in einem Szenengraph hierarchisch verwaltet, mit der \texttt{rootNode} als Wurzelknoten.

Um die interne Physik-Engine zu verwenden, müssen Objekte einen virtuellen Körper erhalten. Dieser Körper enthaltet Informationen wie die Oberflächenbeschaffenheit, Masse und Reibungskoeffizient und entscheidet wie sich das Objekt in der virtuellen Umgebung verhält. Man unterscheidet zwischen statischer, dynamischer oder kinematischer Körpertypen. Statische Körper (z. B. Wände und Böden) werden nicht von Kräften oder Kollisionen beeinflusst. Dynamische Körper (z. B. Bälle oder Raumschiffe) werden direkt von Kräften und Kollisionen beeinflusst. Kinematische Körper sind wie unsichtbare Körper, die helfen mit anderen Objekten zu interagieren. Diese Körper können andere Körper beeinflussen, jedoch nicht von anderen Körper beeinflusst werden. (\cite{arkit-physics})

