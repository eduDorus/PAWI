\section{Lösungsentwicklung}

\subsection{Prototypen}
\textit{Übersicht über die Prototypen …} % TODO:

In diesem Abschnitt folgen die verschiedenen Prototypen, die während der explorativen Phase dieses Projekts erstellt wurden.

\IfFileExists{51-OverlaySceneKit}{
  \subsubsection{Overlay auf einem Würfel mit SceneKit}\label{subsub:prot-overlay}
\begin{description}
	\item[Fragestellung:] Wie kann ein physischer Würfel (ein cuboro Element) mittels AR mit einer Art Overlay hervorgehoben werden? Eignet sich SceneKit dazu?
	\item[Resultat:] Ein mittels SceneKit modellierter Würfel lässt sich an der Stelle eines physischen Würfels platzieren und kann so als Möglichkeit diesen hervorzuheben eingesetzt werden. % TODO: Verweis auf entsprechenden Prototyp (Xcode Projekt) im Repo, bzw. vollständigen Code im Anhang(?)
	\item[Versuchsaufbau:] Dieser Versuch wurde noch mit ARKit 1.0 umgesetzt. Der Ansatz dieses Prototyps ist es, mittels ARKit einen halbtransparenten SceneKit Würfel an die Stelle eines physischen Elements zu positionieren. Die Positionierung wird in diesem Versuch manuell gemacht. Die Basis dieses Versuchs bildet der Beitrag "`Detecting Planes and Placing Objects"' auf dem Blog Machinethinks (\cite{arkit-dectingplanes-placingobjects}) und dem zugehörigen Projekt auf GitHub. Zuerst soll die richtige Ebene der realen Welt definiert werden und im zweiten Schritt wird dann der Würfel darauf platziert.

	\textbf{Flächenerkennung}\\
	Beim Erstellen eines neuen AR Projekts in Xcode kann die gewünschte Technologie für den Inhalt ausgewählt werden. Zur Verfügung stehen neben SceneKit auch SpriteKit und Metal. Die Basis bildet \texttt{ARSCNView}, ein ViewController, der ARKit mit dem SceneKit vereint. In Code \ref{code:prot-overlay-viewdidload} wird die Konfiguration und der Start der \texttt{ARSession} vorgenommen. Da der Würfel auf eine Fläche gestellt werden soll, wird die horizontale Flächenerkennung (Zeile 2) aktiviert.

	\begin{code}{prot-overlay-viewdidload}{Konfiguration und Start der \texttt{ARSession}}
		configuration = ARWorldTrackingConfiguration()
		configuration!.planeDetection = ARWorldTrackingConfiguration.PlaneDetection.horizontal
		sceneView.session.run(configuration!, options: [ARSession.RunOptions.removeExistingAnchors, ARSession.RunOptions.resetTracking])
	\end{code}

	Sobald ARKit eine Fläche detektiert hat, wird die Methode \texttt{renderer(\_:nodeFor:)} des Protokolls \texttt{ARSCNViewDelegate} aufgerufen, die als Rückgabe eine SceneKit Node (\texttt{SCNNode}) für den gefundenen Anker erwartet. Um die erkannte Fläche in der Szene anzuzeigen, wird eine \texttt{SCNBox} mit den geometrischen Ausmassen der Fläche erstellt. Diese Werte sind als Ausdehnung in den X- und Z-Achsen im Attribut \texttt{planeAnchor.extent} enthalten (Code \ref{code:prot-overlay-renderer}, Zeile 4). 

	%TODO: Maybe remove? Else implement method header
	\begin{code}{prot-overlay-renderer}{Implementation der \texttt{renderer(\_:nodeFor:)} Methode zur Darstellung von Flächen}
		var node:  SCNNode?
		if let planeAnchor = anchor as? ARPlaneAnchor {
			node = SCNNode()
			let planeGeometry = SCNBox(width: CGFloat(planeAnchor.extent.x), height: planeHeight, length: CGFloat(planeAnchor.extent.z), chamferRadius: 0.0)
			planeGeometry.firstMaterial?.diffuse.contents = UIColor.green
			planeGeometry.firstMaterial?.specular.contents = UIColor.white
			planeGeometry.firstMaterial?.transparency = 0.1
			let planeNode = SCNNode(geometry: planeGeometry)
			planeNode.position = SCNVector3Make(planeAnchor.center.x, Float(planeHeight / 2), planeAnchor.center.z)
			node?.addChildNode(planeNode)
			anchors.append(planeAnchor)
		}
		return node
	\end{code}

	\textbf{Würfel modellieren}\\
	Jedes SceneKit Element ist eine \texttt{SCNNode} mit einer bestimmten Geometrie und Position. Für würfelförmige Körper bietet sich als Geometrie eine \texttt{SCNBox} an. Die Definition kann so einfach wie folgt sein:
	\mint[style=xcode,breaklines]{swift}{let node = SCNNode(geometry: SCNBox(width: 0.05, height: 0.05, length: 0.05, chamferRadius: 0.0))}
	Damit der Würfel als Overlay wirken kann, sollen die Seitenflächen halbtransparent sein, die Kanten aber hervorgehoben. Zur einfacheren Kontrolle über diese Eigenschaften implementieren wir daher in diesem Versuch die Klasse \texttt{BasicCube} vom Typ \texttt{SCNNode}, bestehend aus einem halbtransparenten Würfel und einem ChildNode mit nur den Kanten als Textur. Dieser Effekt wird mit einem speziellen Shader Modifier erzielt (von \cite{so-shader-modifier}).

	\textbf{Würfel platzieren}\\
	In einem ersten Test wurde beim Erkennen einer Fläche direkt ein \texttt{BasicCube} in das Zentrum der Fläche (\texttt{planeAnchor}) gestellt, indem dem \texttt{position} Attribut der Node folgendes Statement zugewiesen wurde:
	\mint[style=xcode,breaklines]{swift}{SCNVector3Make(planeAnchor.center.x, Float(cubeSize / 2), planeAnchor.center.z)}
	Der Ausgangspunkt von Nodes befindet sich in ihrem Zentrum, weshalb die Höhenkoordinate Y auf die halbe Würfelhöhe gesetzt werden muss. So kommt der Würfel direkt auf der Fläche zu stehen.

	\textbf{Würfel auswählen}\\
	Aus den so in die Mitte aller erkannten Flächen platzierten Nodes soll eine vom Benutzer ausgewählt werden. Beim Überschreiben der Methode \texttt{touchesBegan(\_:with:)} erhalten wir die Position eines Taps des Benutzers in \texttt{touches.first!.location(in:)} als \texttt{CGPoint}. Mit diesem Wert lässt sich auf die \texttt{ARSCNView} ein SceneKit-Hittest machen:
	\mint[style=xcode,breaklines]{swift}{let hitResults = sceneView.hitTest(location, options: [:])}
	Die Rückgabe enthält alle Nodes der Szene, die sich an diesem 2D-Punkt auf dem Bildschirm befinden, in der Reihenfolge, wie sie dargestellt sind. Der erste \texttt{BasicCube} entspricht also dem ausgewählten Würfel und so kann dessen Farbe verändert werden um die Auswahl zu visualisieren.

	\textbf{Fläche bestimmen}\\
	Um eine alternative Methode der Würfel-Platzierung auszuprobieren soll zunächst durch Benutzereingabge eine bestimmte Fläche bestimmt werden. Dies wird wiederum mittels eines Hittest realisiert, doch diesmal von ARKit:
	\mint[style=xcode,breaklines]{swift}{let hitResults = sceneView.hitTest(location, types: .existingPlaneUsingExtent)}
	Der zweite Parameter vom Typ \texttt{ARHitTestResult.ResultTypes} definiert, worauf der Hittest gemacht wird. Hier sollen bereits detektierte Flächen unter Berücksichtigung ihrer Ausmasse betrachtet werden, damit der Nutzer eine der erkannten Flächen auswählen kann. Von allen anderen Flächenanker werden die Nodes mit \texttt{removeFromParentNode()} der Szene entfernt und die Anker selber mit \texttt{sceneView.session.remove(anchor:)} aus der Session gelöscht, sodass sie nicht mehr weiter von ARKit getrackt werden.

	\textbf{Manuelle Platzierung von Würfeln}\\
	% TODO: Screenshots einfügen
	Sobald eine Fläche bestätigt wurde, wird nun ein Positionierungswürfel in der Mitte des Bildschirms angezeigt. Dieser Würfel befindet sich immer auf der bestimmten horizontalen Ebene und dreht sich mit der Kamera. Ein Tap auf den Bildschirm platziert dann eine Kopie des Würfels fix an der aktuellen Position. Damit kann der Positionierungswürfel genau deckungsgleich mit einem physischen Würfel gebracht und dann eine Kopie davon als Overlay hinterlassen werden. Code \ref{code:prot-overlay-positioningcube} zeigt die beiden dazu verwendeten Methoden \texttt{updatePositioningCube()} und \texttt{placeCube()}. Erstere wird stets in der Methode \texttt{renderer(\_:willRenderScene:atTime:)} aufgerufen, sodass er sich mit der Kamera bewegt. Als erstes wird erneut ein AR-Hittest vom Zentrum des Bildschirms gemacht (Zeile 2, \texttt{screenCenter}, Attribut, dem \texttt{sceneView.center} zugewiesen wurde). Auf die erhaltenen Stelle wird die Position des Würfels gesetzt (Zeilen 4-7). Die Rotation erhält man von \texttt{sceneView.pointOfView} als Euler Winkel (Zeilen 8-9). Jedoch schien dies im Versuch nur für zirka 180° zu funktionieren. Das seitliche Drehen des Geräts und somit des Blickwinkels um mehr als einen Viertelkreis in die eine oder andere Richtung lässt den Würfel dann in die falsche Richtung herum rotieren. Wie sich dies korrigieren lässt, wurde in diesem Versuch nicht mehr herausgefunden.

	\begin{code}{prot-overlay-positioningcube}{Positionierungswürfel in der Mitte des Bildschirms}
		func updatePositioningCube() {
			let hitResults = sceneView.hitTest(screenCenter!, types: .existingPlane)
			if hitResults.count > 0 {
				let result : ARHitTestResult = hitResults.first!
				let coords = result.worldTransform.columns.3
				let newLocation = SCNVector3Make(coords.x, (coords.y + Float(positioningCube!.sidelength/2)), coords.z)
				positioningCube!.position = newLocation
				let cameraRotation = sceneView.pointOfView!.eulerAngles
				positioningCube!.eulerAngles = SCNVector3(0, cameraRotation.y, 0)
			}
		}
		
		func placeCube() {
			if positioningCube != nil {
				let cube = BasicCube(withColor: UIColor.green)
				cube.position = positioningCube!.position
				cube.eulerAngles = positioningCube!.eulerAngles
				sceneView.scene.rootNode.addChildNode(cube)
				boxes.append(cube)
			}
		}
	\end{code}

	Um nun einen Würfel an der gewünschten Stelle zu positionieren, wurde aus der Methode \texttt{touchesBegan(\_:with:)} heraus \texttt{placeCube()} aufgerufen. Dort wird ein neuer \texttt{BasicCube} erstellt (Code \ref{code:prot-overlay-positioningcube}, Zeile 15) und mit den Positions- und Winkeleigenschaften des Positionierungswürfel beschrieben (Zeilen 16-17).

\end{description}

}

\IfFileExists{52-PhysischeWuerfelErfassen}{
  \subsubsection{Physischen Würfel als virtuelles Objekt erfassen}\label{subsub:prot-physische-wuerfel}
\begin{description}
	\item[Fragestellung:] Wie kann ein physischer Würfel mittels den Frameworks ARKit, Vision oder CoreML als virtuelles Objekt erfassen werden?
	\item[Resultat:] ARKit bietet die Erkennung zweidimensionaler Elemente. Es verwendet dabei praktiken wie beim bekannten Bilderkennungsframework OpenCV, wobei ein Referenzbild hinterlegt werden muss.
	Mit Vision und CoreML kann ein beliebig trainiertes neurales Netzwerk verwenden, um die Klassifikation durchzuführen. Es konnte leider keine solide Möglichkeit gefunden werden 3D Objekte zu erfassen, damit Sie für eine spätere Augmentierung verwendet werden können. 
	\item[Versuchsaufbau:] Für den Versuchsaufbau wurden zwei Beispielprojekte von der Apple Developer Dokumentation verwendet. Das erste Projekt "`Recognizing Images in an AR Experience"' \cite{arkit-recognize-images} verspricht bekannte 2D Bilder mittels ARKit zu erkennen. Anschliessend können die erkannten Koordinaten verwendet werden um AR Inhalte zu platzieren.
	Beim zweiten Beispielprojekt handelt es sich um das Thema "`Using Vision in Real Time with ARKit"' \cite{vision-real-time-with-arkit} bei dem die Frameworks Vision und CoreML zum Einsatz kommen.

	\textbf{Beispielprojekt "`Recognizing Images in an AR Experience"'} \\
	Das Beispielprojekt kann von der Apple Developer Website heruntergeladen werden. Anschliessend lässt sich das Projekt in XCode öffnen und muss vor der Verwendung auf dem eigenen Gerät signiert werden. Das Verzeichnis "`Resources"' befinden sich bereits einige Demobilder die als Testversuch verwendet werden können. Um die Genauigkeit und Geschwindigkeit zu testen, wurde ein Versuch gestartet indem die Demobilder am Laptop angezeigt wurden. Anschliessend kann die Kamera des IPhones auf den Laptop ausgerichtet werden um den Erkennungsprozess zu starten. Der Versuch wiederspiegelte, dass die Erkennung schnell und zuverlässig erfolgte. Die erkannte Fläche erhielt eine weiss durchsichtig augmentierte Fläche an der stelle wo sich das Bild befindet. Ebenfalls diese augmentierte Fläche wurde korrekt angezeigt. Es wurde festgestellt, dass beim bewegen des IPhones die Fläche nicht exakt gehalten werden kann.

	Darauf Folgend wurde ein eigenes Bild für die Erkennung eines cuboro Elements hinterlegt. Der Prozess wie ein eigenes Bild beigefügt werden kann wird im README.md des Beispielprojektes "`Using Vision in Real Time with ARKit"' detailliert Erklärt. Beim Versuch wurden folgende Schritte durchlaufen:

	\begin{enumerate}
		\item Ein frontal Bild des cuboro Elements aufnehmen. 
		\item Das Bild bearbeitet dass nur die Würfelfläche beibehalten bleibt.
		\item Das Bild der ressourcen Gruppe im XCode hinzufügen.
		\item Neue Version compilieren und auf das Testgerät geladen.
		\item Erkennung des Elements starten.  
	\end{enumerate}

	\bild{0.4}{cuboro-element-frontal}{Frontal Ansicht vom einem cuboro Element}


	Die Implementation der Erkennung wird in den folgenden Zeilen konfiguriert. ARKit stellt die Erkennung der Referenzbilder zur Verfügung, wobei keine weiteren Implementationsschritte notwendig sind. Es wird zuerst eine Referenz auf das Ressourcenverzeichnis erstellt. Anschliessend wird diese Referenz der \texttt{ARWorldTrackingConfiguration} mitgegeben mittels \texttt{.detectionImages}.
	\begin{code}{arkit-recognition-configuration}{Implementation der Erkennung von Referenzbilder mit ARKit}
	guard let referenceImages = ARReferenceImage.referenceImages(inGroupNamed: "AR Resources", bundle: nil) else {
		fatalError("Missing expected asset catalog resources.")
	}
	
	let configuration = ARWorldTrackingConfiguration()
	configuration.detectionImages = referenceImages
	session.run(configuration, options: [.resetTracking, .removeExistingAnchors])
	\end{code}

	Wenn ein Bild aus dem Ressourcenverzeichnis erkennt wurde, wird ein \texttt{ARImageAnchor} zurückgegeben. Ein \texttt{ARImageAnchor} enthält diverse informationen z.B über die Position im Koordinatensystem. Dies wird in diesem Beispiel verwendet um das augmentierte Fläche zu erzeugen. 

	%TODO: Code muss an der richtigen Stelle platziert werden.
	\begin{code}{augmentierte Fläche-renderer}{Implementation der \texttt{renderer(\_:nodeFor:)} Methode zur Darstellung von Flächen}
		func renderer(_ renderer: SCNSceneRenderer, didAdd node: SCNNode, for anchor: ARAnchor) {
			guard let imageAnchor = anchor as? ARImageAnchor else { return }
			let referenceImage = imageAnchor.referenceImage
			updateQueue.async {
				let plane = SCNPlane(width: referenceImage.physicalSize.width,
									height: referenceImage.physicalSize.height)
				let planeNode = SCNNode(geometry: plane)
				planeNode.opacity = 0.25
				planeNode.eulerAngles.x =  - .pi / 2
				planeNode.runAction(self.imageHighlightAction)
				node.addChildNode(planeNode)
			}

			DispatchQueue.main.async {
				let imageName = referenceImage.name ?? 
				self.statusViewController.cancelAllScheduledMessages()
				self.statusViewController.showMessage("Detected image (imageName)")
			}
		}
	\end{code}


	\textbf{Beispielprojekt "`Using Vision in Real Time with ARKit"'} \\
	Das zweite Beispiel bei diesem Versuch beschäftigt sich mit Vision, CoreML und ARKit. Die Bildaufnahmen von ARKit werden an Vision weitergegeben und anschliessend mittels einem trainierten neuralen Netzwerk(Inception v3, \cite{DBLP:journals/corr/SzegedyVISW15}) in CoreML ausgewertet. 

	Der Code \ref{code:arkit-recognition-session} ausschnitt startet die ARSession sowie den Klassifizierungsprozess. Da dieser Task leistungs intensive ist, können nur eine gewisse Anzahl Bilder analysiert werden. Es wird stets geprüft ob sich bereits ein Bild im Puffer befindet bevor ein neues Bild zur Auswertung freigegeben wird.

	\begin{code}{arkit-recognition-session}{Startet die ARSession und den Klassifizierungsprozess}	
	func session(_ session: ARSession, didUpdate frame: ARFrame) {
		guard currentBuffer == nil, case .normal = frame.camera.trackingState else {
			return
		}
		self.currentBuffer = frame.capturedImage
		classifyCurrentImage()
	}
	\end{code}

	In diesem Code ausschnitt wird eqw bereits trainiertes neurales Netzwerk in das CoreML Framework geladen. Es kann somit auch ein beliebig selbst trainiertes Netwerk verwendet werden.
	\begin{code}{CoreML-request}{Initialisierung der Klassifizierung mittels CoreML}
	private lazy var classificationRequest: VNCoreMLRequest = {
		do {
			let model = try VNCoreMLModel(for: Inceptionv3().model)
			let request = VNCoreMLRequest(model: model, completionHandler: { [weak self] request, error in
				self?.processClassifications(for: request, error: error)
			})
			request.imageCropAndScaleOption = .centerCrop
			request.usesCPUOnly = true
			return request
		} catch {
			fatalError("Failed to load Vision ML model: (error)")
		}
	}()
	\end{code}

	Das Antippen eines klassifizierten Objektes erstellt ein Label mit SpriteKit. Dies erfolgt mittels einem Hit Test der die genauen Koordinaten ausfindig macht und an dieser Stelle ein \texttt{ARAnchor} setzt. Dieser \texttt{ARAnchor} wird dazu verwendet das Label and dieser Stelle einzublenden und im Weltkoordinatensystem zu festigen.
	\begin{code}{hit-test-klassifizierte-objekte}{Erstellt ein Label beim Antippen von klassifizierten Objekten}
	@IBAction func placeLabelAtLocation(sender: UITapGestureRecognizer) {
		let hitLocationInView = sender.location(in: sceneView)
		let hitTestResults = sceneView.hitTest(hitLocationInView, types: [.featurePoint, .estimatedHorizontalPlane])
		if let result = hitTestResults.first {
			
			let anchor = ARAnchor(transform: result.worldTransform)
			sceneView.session.add(anchor: anchor)
			
			// Track anchor ID to associate text with the anchor after ARKit creates a corresponding SKNode.
			anchorLabels[anchor.identifier] = identifierString
		}
	}
	\end{code}

\end{description}

}

\IfFileExists{53-ModellKugelbahn}{
  \subsubsection{Virtuelles Modell einer Kugelbahn projizieren}\label{subsub:prot-kugelbahn}
\begin{description}
	\item[Fragestellung:] Wie kann eine gesamte Kugelbahn modelliert und als Ganzes in Augmented Reality auf eine Fläche projiziert werden?
	\item[Resultat:] Durch einen hierarchischen Aufbau von SceneKit Nodes, kann eine Kugelbahn aus mehreren einzelnen Elementen modelliert und als Ganzes modifiziert werden (bspw. verschieben, drehen). Eine Klasse, die von \texttt{SCNNode} erbt, enthält und verwaltet die Elemente als Kindknoten. Der \texttt{SCNBillboardConstraint} von SceneKit vereinfacht zudem das Orientieren von Objekten zur Kamera hin. % TODO: Verweis auf entsprechenden Prototyp (Xcode Projekt) im Repo, bzw. vollständigen Code im Anhang(?)
	\item[Versuchsaufbau:] Da zum Zeitpunkt dieses Versuchs keine vollständige 3D Modelle von echten cuboro Elemente vorhanden waren, wurde der Versuch mit einer \texttt{BasicCube} Klasse auf der Grundlage von Prototyp \ref{subsub:prot-overlay} aufgebaut. Dieser Versuch bildete auch die Basis für die hier verwendete Flächenerkennung, -auswahl und das Platzieren der Kugelbahn.

	\textbf{Tracking Statusanzeige} \\ % TODO: dieser Abschnitt war zwar Teil dieses Versuchs, gehört aber Thematisch irgendwie nicht hierher?
	Die Methode \texttt{session(\_:cameraDidChangeTrackingState:)} wird bei jeder Änderung des Tracking Status durch das Framework aufgerufen. Das erhaltene \texttt{ARCamera} Objekt hat ein Attribut \texttt{trackingState} mit den möglichen Zuständen \texttt{normal}, \texttt{notAvailable} und \texttt{limited}. Für letzteres gibt es zudem eine der folgenden Begründungen:
	\begin{itemize}
		\item \texttt{initializing}: Die Session hat noch nicht genügend Informationen für das Tracking. Diese Phase dauert auf einem iPhone 6s in der Regel ein paar Sekunden beim Start der Session.
		\item \texttt{insufficientFeatures}: Die sichbare Szene enthält zu wenige Features. Dies ist beispielweise bei einer einfarbigen Tischoberfläche ohne Musterung der Fall. Solche Tischflächen als Grundlage für eine ARKit Szene zu nutzen ist daher schwierig. Sie werden meistens nicht erkannt. Auch eine ungenügende Beleuchtung des Raumes kann Ursache für diesen Zustand sein.
		\item \texttt{excessiveMotion}: Das Gerät bewegt sich zu stark, sodass präzises Feature Tracking nicht möglich ist.
		\item \texttt{relocalizing}: Die Session wurde unterbrochen und es wird versucht den vorherigen Zustand wiederherzustellen.
	\end{itemize}
	Um den Tracking State anzuzeigen, wird ein Label mit entsprechendem Text gesetzt. Damit Labels im Interface Builder zusammen mit der Scene View genutzt werden können, müssen die Elemente gegenüber dem Standard AR Projekt von Xcode in einer gemeinsamen \texttt{UIView} als Container gesetzt werden.

	\textbf{Kugelbahn modellieren}\\
	Eine neue eigene Klasse \texttt{MarbleTrack} hält alle Informationen zu einer Kugelbahn und bietet Methoden zur Manipulation der Kugelbahn an. Die Klasse erbt von \texttt{SCNNode} und kann damit direkt als Node in einer AR Szene genutzt werden. Im Wesentlichen soll die Klasse aus ganzzahligen Koordinaten eine Kugelbahn mit \texttt{BasicCube} Nodes als Kindknoten zusammenstellen. Die Koordinaten eines Elements beschreiben seine Position in Anzahl Würfel, ausgehend von einem Basiselement in (0,0,0). Die Achsenrichtungen korrespondieren mit denen der ARKit Session.

	Ein Array von Tupeln mit den Koordinaten der drei Achsen bildet die Struktur für eine Kugelbahn. Folgender Code beschreibt zwei aufeinander stehende Elemente und ein drittes Element rechts daneben:
	\mint[style=xcode,breaklines]{swift}{let track = [(0,0,0), (0,1,0), (1,0,0)]}

	Beim Iterieren durch das Array wird für jedes Tupel ein \texttt{BasicCube} als Kindknoten hinzugefügt. Zur Positionierung werden die gegebenen Koordinaten mit der Seitenlänge des Elements multipliziert (Code \ref{code:prot-kugelbahn-addcube}, Zeile 3).

	\begin{code}{prot-kugelbahn-addcube}{Methode \texttt{addCube(x:y:z:)} um einen \texttt{BasicCube} als Kindknoten hinzuzufügen}
		func addCube(x: Int, y: Int, z: Int) -> BasicCube {
			let cube = BasicCube()
			let pos = SCNVector3(CGFloat(x) * cube.sidelength, CGFloat(y) * cube.sidelength, CGFloat(z) * cube.sidelength)
			cube.set(position: pos)
			addChildNode(cube)
			return cube
		}
	\end{code}

	\textbf{Kugelbahn projizieren}\\
	Ähnlich dem Vorgehen in Versuch \ref{subsub:prot-overlay}, wird zuerst die Flächenerkennung und -auswahl gemacht. Statt eines Positionierungswürfel wird hier nun aber eine \texttt{MarbleTrack} Node hinzugefügt. Die Verwendung von \texttt{eulerAngle} zur Rotation hatte sich nicht bewährt. Um ein Objekt stets zur Kamera hin zu orientieren, bietet SceneKit den Constraint \texttt{SCNBillboardConstraint}. Indem nur Y als frei bewegliche Achse gesetzt wird (Code \ref{code:prot-kugelbahn-billboardconstraint}, Zeile 3), bleibt die Kugelbahn in der Horizontalen, dreht sich aber zur Kamera. Die Constraints zu entfernen reicht aus, um die automatische Ausrichtung zu beenden (Zeile 9). Die Node bleibt dann in ihrer aktuellen Position stehen. Die beiden Methoden in Code \ref{code:prot-kugelbahn-billboardconstraint} werden aus dem View Controller bei \texttt{touchesBegan(\_:with:)} abwechslungsweise aufgerufen, wodurch zwischen den beiden Zuständen mittels Toucheingabe gewechselt werden kann.

	\begin{code}{prot-kugelbahn-billboardconstraint}{Verwendung von Constraints um Nodes zur Kamera auszurichten}
		func constraintToCamera() {
			let constraint = SCNBillboardConstraint()
			constraint.freeAxes = SCNBillboardAxis.Y
			constraints = [constraint]
		}
    
		func removeConstraints() {
			constraints = []
		}
	\end{code}

	% TODO: Screenshots einfügen der Drehung? Von einer platzierten Kugelbahn?

\end{description}

}

\IfFileExists{54-KugelbahnAufbau}{
  \subsubsection{Schrittweise augmentierte Bauanleitung einer Kugelbahn}\label{subsub:prot-kugelbahnaufbau}
\begin{description}
	\item[Fragestellung:] Wie kann eine schrittweise Bauanleitung für eine Kugelbahn umgesetzt/augmentiert werden? Es soll Würfel-für-Würfel den Aufbau einer einfache Kugelbahn auf einer Fläche angezeigt werden.
	\item[Resultat:] Mit einem Dictionary, das als Schlüssel die Koordinaten und die Kugelbahn-Elemente als Werte verwendet, lassen sich die einzelnen Würfel problemlos verändern. Ein Algorithmus, der anhand der Koordinaten durch die Elemente geht, kann in diesem Setting eine Würfel-für-Würfel Anleitung erzeugen. In diesem Versuch wurde dies mit einem ebenenweisen Aufbau vereinfacht. ARKit hat die Angewohnheit, sich mit der Zeit aus bislang unklaren Gründen zu verschieben. Oftmals betragen die Verschiebungen bloss 1-2 Zentimeter, was bei Würfeln mit 5 cm Seitenlänge beträchtlich ist. Dies müsste mit weiteren Versuchen genauer beobachtet werden.
	\item[Versuchsaufbau:] Dieser Versuch baut direkt auf \ref{subsub:prot-kugelbahn} auf und erweitert diesen. Der View wurde zusätzlich ein Button hinzugefügt, um die verschiedenen Schritte des Aufbaus zu bestätigen. Sobald die Kugelbahn platziert wird, ist der Button aktiviert und die Aufbauanleitung wird mit einem Tap darauf gestartet.

	\textbf{Variante 1: Rekonstruktion der Bahn}\\
	Als erster Versuch soll jede horizontale Ebene der Bahn als Schritt des Aufbauens separat hervorgehoben werden. In der Node Hierarchie gibt es keine Struktur, welche die Positionen der Elemente abbildet. Daher müsste beim Auswählen von bestimmten Würfeln immer durch alle Kindknoten iteriert werden. Beim Start der Bauanleitung werden aber zunächst alle Würfel entfernt. In der Folge wird die Kugelbahn wieder neu aus dem Array von Tuples erstellt. Dabei werden nur die Würfel erstellt, die sich auf der aktuellen Ebene des Aufbaus befinden (Code \ref{code:prot-kugelbahnaufbau-loadcurrentlevel1}, Zeile 3-4). All diese Würfel werden rot hervorgehoben (Zeile 4).

	\begin{code}{prot-kugelbahnaufbau-loadcurrentlevel1}{Hinzufügen der Würfel, die sich auf der aktuellen Aufbau-Ebene befinden}
	private func loadTrackCurrentBuildingLayer() {
		for block in track {
			if block.1 == currentBuildingStep - 1 {
				addCube(x: block.0, y: block.1, z: block.2).set(color: UIColor.red)
			}
		}
	}
	\end{code}

	Ab der zweiten Ebene werden vor dem Hinzufügen alle bisherigen Würfel weiss gefärbt, damit schlussendlich nur die aktuelle Ebene hervorgehoben ist. Dazu wird die \texttt{SCNNode} Methode \texttt{enumerateChildNodes(\_:)} zur Hilfe genommen (Code \ref{code:prot-kugelbahnaufbau-removehighlight1}).

	\begin{code}{prot-kugelbahnaufbau-removehighlight1}{Iteration durch Kindknoten zur Änderung der Farbe}
		private func clearHighlights() {
			enumerateChildNodes { (node, stop) in
				if let cube = node as? BasicCube {
					cube.set(color: UIColor.white)
				}
			}
		}
	\end{code}

	Mit jedem Tap auf den Button wird \texttt{currentBuildingStep} inkrementiert und die beiden Methoden \texttt{clearHighlights()} und \texttt{loadTrackCurrentBuildingLayer()} ausgeführt (Code \ref{code:prot-kugelbahnaufbau-loadcurrentlevel1} und \ref{code:prot-kugelbahnaufbau-removehighlight1}). Ein Ende des Aufbaus gibt es nicht, da die Höhe der Bahn nicht geprüft wird.

	\textbf{Variante 2: Informationen über die Position von Elementen erhalten}\\
	Es wäre eine deutliche Vereinfachung des Vorgehens, wenn man direkt über die Koordinaten auf ein Elemen, einen Würfel, zugreifen könnte. Dazu müssen die Würfel in einer Datenstruktur refernziert werden. Ein dreidimensionales Array ist aufgrund der negativen Koordinatenwerte im momentanen Aufbau nicht geeignet. In diesem Versuch wird probiert dies mit einem Dictionary zu lösen. Hierbei soll ein Koordinaten-Tuple als Schlüssel und die \texttt{BasicCube} als Werte dienen.
	
	Der Schlüssel eines Dictionaries muss das Protokoll \texttt{Hashable} adoptieren, was Swift Tuple nicht machen. Deswegen übernimmt dies ein neues Struct \texttt{Triple<T,U,V>}, mit einem Tuple von drei Werten als Attribut. So kann das Dictionary für die dreidimensionale "`Karte"' der Kugelbahn wie folgt erstellt werden:

	\mint[style=xcode,breaklines]{swift}{var map : [Triple<Int,Int,Int> : BasicCube] = [:]}

	Damit die Operationen an der Karte von übrigen Aktionen auf der gesamten Kugelbahn getrennt ist, verwaltet die Klasse \texttt{TrackMap<E>} das Dictionary als privates Attribut. Sie bietet Methoden an um Elemente hinzuzufügen, zu entfernen und um bestimmte Elemente anhand der Koordinaten und umgekehrt zu erhalten.

	\texttt{MarbleTrack} hält nun eine \texttt{TrackMap<BasicCube>} und nutzt deren Methoden für die Operationen aus der vorherigen Variante 1 (Code \ref{code:prot-kugelbahnaufbau-trackmapoperationen}). Da im Gegensatz zur Iteration über Kindknoten die Elemente im Dictionary bereits als \texttt{BasicCube} Typ definiert sind, entfällt zudem das Downcasting von \texttt{SCNNode} auf \texttt{BasicCube} (wie in Code \ref{code:prot-kugelbahnaufbau-removehighlight1}, Zeile 3 notwendig).

	\begin{code}{prot-kugelbahnaufbau-trackmapoperationen}{Methoden aus Code \ref{code:prot-kugelbahnaufbau-loadcurrentlevel1} und \ref{code:prot-kugelbahnaufbau-removehighlight1} mit \texttt{map : TrackMap<BasicCube>}}
		private func loadTrackCurrentBuildingLayer() {
			map.getElements(atLevel: currentBuildingStep-1).forEach { (_, cube) in
				cube.show()
				cube.set(color: UIColor.red)
			}
		}

		private func clearHighlights() {
			map.forEach { (_, cube) in
				cube.set(color: UIColor.white)
			}
		}
	\end{code}

	Statt dass alle Würfel beim Beginn der Aufbauanleitung entfernt werden, werden sie nun bloss über Verändern der Transparenz ausgeblendet. Der Aufbau der Bahn geht weiterhin Ebene für Ebene.

	\textbf{Genauigkeit und Stabilität der Augmentierung}\\
	Während der Entwicklung fiel auf, dass sich das virtuelle Modell der Kugelbahn gegenüber der realen Welt oftmals verschob. Wenn sich die Szene durch den Aufbau physischer Würfel verändert, könnte das diesen Effekt erheblich verstärken. Zumal ARKit mit einer Kombination aus Beschleunigungssensoren und dem Kamerabild – das sich durch den Aufbau eben verändert – arbeitet.

	In einem Test wurde auf einem Tisch mit vielen deutlichen Features die virtuelle Kugelbahn platziert. Ein einzelner physischer Würfel wurde als Messpunkt genau an eine Stelle der Bahn gesetzt. Die ganze Fläche des Tisches wurde anschliessend vollständig abgedeckt, ohne dass sich die Kugelbahn wesentlich verschob. Nachdem die Abdeckung wieder entfernt wurde, wurde die Verschiebung anhand des gesetzten Würfels sichtbar. In mehreren Durchläufen hat sich die AR Kugelbahn meistens um rund einen Zentimeter verschoben. Oftmals war die Verschiebung vertikal.

	Die beobachteten Verschiebungen passieren teilweise auch ohne Veränderung der Szene. Weitere Tests sind notwendig, um festzustellen, ob mit diesen Ungenauigkeiten auch bei idealen Bedingungen gerechnet werden müssen.

	% TODO: Versuch genauer beschreiben, mit Fotos am besten.
	% Folgende Tests (as a reminder):
	% - der ganze Tisch abgedeckt wird und keine Fläche mehr vorhanden ist.
	% - nahe an das Modell heran gehen


\end{description}

}

\IfFileExists{55-ElementInteraktionen}{
  \subsubsection{Mittels Touchgesten mit virtuellen Objekten interagieren}
\begin{description}
	\item[Fragestellung:] Wie kann mit Touch- und Swipegesten mit virtuellen Objekten interagiert werden?
	\item[Resultat:] Es konnte erfolgrich ein Prototyp entwickelt werden, bei dem es Möglich ist mittels verschiedenen Gesten mit einem virtuellen Objekt zu interagieren. Eine besondere Schwierigkeit bestand darin das Element zu kippen da je nach Kameraperspektive um die Z- oder X Achse gedreht werden muss. 
    \item[Versuchsaufbau:] Für diesen Versuch wurde ein neues XCode Projekt initialisiert. Anschliessend wurde die Logik implementiert um einen Würfel auf einer Fläche zu augmentieren. Es wurde die Funktionalität entwickelt mittels verschiedenen Gesten mit dem Würfel zu interagieren. Zu den funktionalitäten gehören:
    
    
    \begin{itemize}
        \item \texttt{UITapGestureRecognizer} mit einmaligem Tippen
        \item \texttt{UITapGestureRecognizer} mit zweimaligem Tippen
        \item \texttt{UILongPressGestureRecognizer} für langes antippen eines Objektes
        \item \texttt{UISwipeGestureRecognizer} für die Rotationen
    \end{itemize}


    \textbf{\texttt{UITapGestureRecognizer} mit einmaligem Tippen}\label{textbf:tap-gesture-recognizer}\\
    Damit mit Touchgesten gearbeitet werden kann müssen diese der SceneView hinzugefügt werden Code \ref{code:prot-gesture-addTapGestureToSceneView}. Für einzelne Taps können wir den \texttt{UITapGestureRecognizer} initialisieren und als action parameter eine Methode angeben. Wir geben hierfür die Methode \texttt{ViewController.didTap(withGestureRecognizer:)}. Zusätzlich sollte die Anzahl benötigter Taps mit \texttt{tapGestureRecognizer.numberOfTapsRequired} auf Eins gesetzt werden, da wir auch einen DoubleTap erkennen wollen. 
    \begin{code}{prot-gesture-addTapGestureToSceneView}{Methode \texttt{addTapGestureToSceneView()} um das einmalige antippen der \texttt{SceneView} hinzuzufügen}
    func addTapGestureToSceneView() {
        let tapGestureRecognizer = UITapGestureRecognizer(target: self, action: #selector(ViewController.didTap(\_:)))
        tapGestureRecognizer.numberOfTapsRequired = 1
        sceneView.addGestureRecognizer(tapGestureRecognizer)
    }
    \end{code}

    Sollte erfolgreich ein Tap auf der SceneView registriert werden, so wird die Methode \texttt{didTap(\_ recognizer: UIGestureRecognizer)} im Code \ref{code:prot-gesture-didTap} ausgeführt. Bei dieser Methode werden die 2D Koordinaten des Taps ausgelesen und anschliesend an einem \texttt{hitTest(\_:types:)} im 3D Raum weitergegeben. Dieser HitTest prüft ob es sich um ein SceneKit Node handelt. Falls dies der Fall ist wird dieses Element selektiert und als interagierbares Objekt gesetzt. Dies geschied mir der Methode \texttt{updateSelectedObject(node: node)} aus dem Code \ref{code:prot-gesture-updateSelectedObject}. In diesem Code abschnitt werden auch zwei \texttt{SCNTransaction} ausgeführt damit der Benutzer anhand einer Visuellen animation erkennt welches das aktuell selektierte Objekt ist. Mittels \texttt{SCNTransaction} können komplexe Animationen erstellt werden und innerhalb einer Transaktion ausgeführt werden.

    \begin{code}{prot-gesture-didTap}{Methode \texttt{didTap(\_ recognizer: UIGestureRecognizer)} die beim einmaligen antippen der \texttt{SceneView} ausgeführt wird}
    @objc
    func didTap(_ recognizer: UIGestureRecognizer) {
        let tapLocation = recognizer.location(in: sceneView)
        let hitTestOptions: [SCNHitTestOption: Any] = [.boundingBoxOnly: true]
        let hitTestResults = sceneView.hitTest(tapLocation, options: hitTestOptions)

        guard let node = hitTestResults.first?.node else {return}
        
        updateSelectedObject(node: node)
    }
    \end{code}

    \begin{code}{prot-gesture-updateSelectedObject}{Methode \texttt{updateSelectedObject(node: node)} zur aktualisierung des ausgewählten Objekts}
    func updateSelectedObject(node: SCNNode) {
        // reset color of old selected object
        SCNTransaction.begin()
        SCNTransaction.animationDuration = 0.5
        selectedObject?.geometry?.firstMaterial?.diffuse.contents = UIColor(red: 182.0 / 255.0, green: 155.0 / 255.0, blue: 76.0 / 255.0, alpha: 1)
        SCNTransaction.commit()
        
        // Set new selected node
        selectedObject = node
        
        // Set selected color
        SCNTransaction.begin()
        SCNTransaction.animationDuration = 0.5
        selectedObject?.geometry?.firstMaterial?.diffuse.contents = UIColor(red: 130.0 / 255.0, green: 82.0 / 255.0, blue: 1.0 / 255.0, alpha: 1)
        SCNTransaction.commit()
    }
    \end{code}

    \textbf{\texttt{UITapGestureRecognizer} mit zweimaligem Tippen}\\
    Das zweimalige Antippen der SeceneView setzt einen neuen Würfel. Das Prinzip um das zweimalige Antippen zu registriern ist gleich wie im obigen Abschnitt "`\texttt{UITapGestureRecognizer} mit einmaligem Tippen"' ausser das wir die \texttt{doubleTapGestureRecognizer.numberOfTapsRequired} auf zwei erhöhen:
    \mint[style=xcode,breaklines]{swift}{doubleTapGestureRecognizer.numberOfTapsRequired = 2}

    Als auszuführende Methode wird \texttt{didDoubleTap(\_ recognizer: UIGestureRecognizer)} gesetzt. Bei der dieser Methode (Code \ref{code:prot-gesture-didDoubleTap}) wird wie beim einfachen Antippen geprüft on ein SeceneKit objekt beim HitTest getroffen wird. Ist dies nicht der Fall können wir den neuen Würfel platzieren. Dies wird mit der Methode \texttt{addBox(x: translation.x, y: translation.y, z: translation.z)} gemacht. Die Koordinaten des HitTest werden hierfür von einer Matrizen zu x, y, z konvertiert. Dies wurde anhand einer Extension gemacht Code \ref{code:prot-gesture-extension}

    \begin{code}{prot-gesture-didDoubleTap}{Methode \texttt{didDoubleTap(\_ recognizer: UIGestureRecognizer)} die beim zweimaligen Antippen der \texttt{SceneView} ausgeführt wird}
    @objc
    func didDoubleTap(_ recognizer: UIGestureRecognizer) {
        let tapLocation = recognizer.location(in: sceneView)
        let hitTestResults = sceneView.hitTest(tapLocation)
        
        guard let node = hitTestResults.first?.node else {
            let hitTestResultWithFeaturePoints = sceneView.hitTest(tapLocation, types: .estimatedHorizontalPlane)
            if let hitTestResultWithFeaturePoints = hitTestResultWithFeaturePoints.first {
                let translation = hitTestResultWithFeaturePoints.worldTransform.translation
                addBox(x: translation.x, y: translation.y, z: translation.z)
            }
            return
        }
    }
    \end{code}

    \begin{code}{prot-gesture-extension}{Extension funktion einer float4x4 Matrize um die Koordinaten in x, y und z zurückzugeben}
    // Helper function to convert a matrix to float3 -> x,y,z
    extension float4x4 {
        var translation: float3 {
            let translation = self.columns.3
            return float3(translation.x, translation.y, translation.z)
        }
    }
    \end{code}

    \textbf{\texttt{UILongPressGestureRecognizer} für langes antippen eines Objektes}\\
    Die neu erstellten Würfel sollten auch wieder gelöscht werden können. Um diese Verhalten möglichst Nutzerfreundlich zu gestalten kann lange auf ein Würfel getippt werden. Die Umsetzung erfolgt mit dem \texttt{UILongPressGestureRecognizer}. Der \texttt{UILongPressGestureRecognizer} muss wie bei den obigen Beispielen der SceneView hinzugefügt werden.

    \begin{code}{prot-gesture-addLongTapGestureToSceneView}{Methode \texttt{addLongTapGestureToSceneView()} um das lange antippen der \texttt{SceneView} hinzuzufügen}
    func addLongTapGestureToSceneView() {
        let longTapGestureRecognizer = UILongPressGestureRecognizer(target: self, action: #selector(ViewController.didLongPress(\_:)))
        sceneView.addGestureRecognizer(longTapGestureRecognizer)
    }
    \end{code}

    Wird ein \texttt{UILongPressGestureRecognizer} Event registriert so wird die Funktion \texttt{didLongPress(\_ recognizer: UILongPressGestureRecognizer)} (Code \ref{code:prot-gesture-addLongTapGestureToSceneView}) aufgerufen. Hier findet wie bei den obigen Beispielen ein HitTest statt. Sollte sich an der lang angetippten Stelle ein Würfel befinden so wird dieser mit der Methode \texttt{node.removeFromParentNode()} der SceneView entfernt.

    \begin{code}{prot-gesture-didLongPress}{Methode \texttt{didLongPress()}, die bei einem langen Antippen ausgeführt wird}
    @objc
    func didLongPress(\_ recognizer: UILongPressGestureRecognizer) {
        let longPressLocation = recognizer.location(in: sceneView)
        let hitTestResults = sceneView.hitTest(longPressLocation)
        
        guard let node = hitTestResults.first?.node else { return }
        node.removeFromParentNode()
    }
    \end{code}
    
    \textbf{\texttt{UISwipeGestureRecognizer} für die Rotationen}\\
    Damit die Rotation eines Würfels möglichst einfach wird wurde dies mit Wischgeste umgesetzt. Damit die Wischgeste für links, rechts, oben und untern funktioniert benötigt es vier \texttt{UISwipeGestureRecognizer}. Im Code \ref{code:prot-gesture-addSwipeGestureToSceneView} werden die vier \texttt{UISwipeGestureRecognizer} initialisiert. Wichtig hierbei ist es, dass jeweils die Richtig der Wischgeste mittels \texttt{swipeRightGesture.direction} gesetzt wird. Als auszuführende Aktion wird jeweils die Methode \texttt{didSwipe(\_:)} angegeben.

    \begin{code}{prot-gesture-addSwipeGestureToSceneView}{Methode \texttt{addSwipeGestureToSceneView()} um die Wischgesten nach links, rechts, unten und oben der \texttt{SceneView} hinzuzufügen}
        func addSwipeGestureToSceneView() {
            let swipeRightGesture = UISwipeGestureRecognizer(target: self, action: #selector(didSwipe(_:)))
            swipeRightGesture.direction = .right
            
            let swipeLeftGesture = UISwipeGestureRecognizer(target: self, action: #selector(didSwipe(_:)))
            swipeLeftGesture.direction = .left
            
            let swipeUpGesture = UISwipeGestureRecognizer(target: self, action: #selector(didSwipe(_:)))
            swipeUpGesture.direction = .up
            
            let swipeDownGesture = UISwipeGestureRecognizer(target: self, action: #selector(didSwipe(_:)))
            swipeDownGesture.direction = .down
            
            sceneView.addGestureRecognizer(swipeLeftGesture)
            sceneView.addGestureRecognizer(swipeRightGesture)
            sceneView.addGestureRecognizer(swipeUpGesture)
            sceneView.addGestureRecognizer(swipeDownGesture)
        }
    \end{code}

    \textbf{Drehen von Würfeln anhand der akutellen Kameraposition}\\
    Beim Drehen der Würfel bestehn zwei Schwierigkeiten. Einerseits muss der Würfel immer relativ zur Kamera nach ob oder unten gekippt werden. Es muss also entschieden werden ob die Rotation zur X-Achse oder Z-Achse erfolgt. Zusätzlich dreht beim rotieren eines Würfels das relative Koordinatensystem mit. Dies bedeutet dass bedeutet dem Kippen eine links Wischgeste den Würfel ebenfalls kippen würde.

    Das erste Problem kann gelöst werden indem anhand der aktuellen Kameraposition die Y-Achse ausgewertet wird. Je nach Winkel kann somit entschieden werden um welche Achse der Würfel sich drehen sollte. %TODO: HIER NOCH EIN BILD EINFÜGEN

    Das zweite Problem kann mittels Überschreiben des relativen Würfelkoordinatensystems gelöst werden. Nach der Rotation muss das relative Koordinatensystem des Würfels mit dem der Ausrichtig des World Origin überschrieben werden (Code \ref{code:prot-gesture-didSwipe} Zeile 29).

    \begin{code}{prot-gesture-didSwipe}{Methode \texttt{didSwipe(\_ gesture: UISwipeGestureRecognizer)} die beim Wischgesten nach links, rechts, unten und oben der \texttt{SceneView} ausgeführt wird}
    @objc
    func didSwipe(_ gesture: UISwipeGestureRecognizer) {
        let currentAngle = sceneView.session.currentFrame?.camera.eulerAngles.y
        var action: SCNAction!
        
        if gesture.direction == .right {
            action = SCNAction.rotateBy(x: 0, y: CGFloat(Double.pi/2), z: 0, duration: 0.5)
        }
        else if gesture.direction == .left {
            action = SCNAction.rotateBy(x: 0, y: CGFloat(-(Double.pi/2)), z: 0, duration: 0.5)
        }
        else if gesture.direction == .up {
            if (currentAngle! > Float(0.785) && currentAngle! < Float(2.356)) {
                action = SCNAction.rotateBy(x: 0, y: 0, z: CGFloat(Double.pi/2), duration: 0.5)
            }
            else if (currentAngle! > Float(2.356) || currentAngle! < Float(-2.356)) {
                action = SCNAction.rotateBy(x: CGFloat(Double.pi/2), y: 0, z: 0, duration: 0.5)
            }
            else if (currentAngle! > Float(-2.356) && currentAngle! < Float(-0.785)) {
                action = SCNAction.rotateBy(x: 0, y: 0, z: CGFloat(-(Double.pi/2)), duration: 0.5)
            }
            else if (currentAngle! > Float(-0.785) && currentAngle! < Float(0.785)) {
                action = SCNAction.rotateBy(x: CGFloat(-(Double.pi/2)), y: 0, z: 0, duration: 0.5)
            }
        }
        ...
        if action != nil {
            selectedObject?.runAction(action, forKey: "rotate")
            selectedObject?.orientation = SCNQuaternion(x: 0.0, y: 0.0, z: 0.0, w: 1.0)
        }
    }
    \end{code} 

\end{description}    

}

\IfFileExists{56-BoundingBox}{
  \subsubsection{Konzept zur Erkennung von einem physichen Objekt in einer bounding Box}
\begin{description}
    \item[Fragestellung:] Wie kann mittels Hit-Tests festgestellt werden ob sich ein physisches Objekt in einer bounding Box befindet?
	\item[Resultat:] Es wurden verschiedene Vorgehensweisen dokumentiert. 
    \item[Versuchsaufbau:] Um ein physisches Element in einer augmentierten Bounding Box zu erfassen und sicherzustellen dass sich dieses in der vorgegebenen Bounding Box befindet. 

    Es gibt grundsätzlich zwei mögliche Vorgehensweisen. Eine davon beschränkt sich auf das Hit-Testen des Mittelpunkts des Boxes. Eine andere Vorgehensweise wäre die oberen vier Eckpunkte der Bounding Box mit Hit-Tests zu versehen. Dies könnte auch ermitteln ob sich das Element an der Richtigen stelle in der Box befinden würde.
    \bild{1}{hit-test}{Vorgehensweisen}


\end{description}
}

\IfFileExists{57-KugelbahnAufbau2}{
  \subsubsection{Elementweise Bauanleitung einer Kugelbahn}\label{subsub:prot-kugelbahnaufbau2}
\begin{description}
	\item[Fragestellung:] Wie kann kann der Prototyp \ref{subsub:prot-kugelbahnaufbau} erweitert werden, damit die Bauanleitung Würfel-für-Würfel statt Ebenenweise erfolgt?
	\item[Resultat:] Eine spezifische Klasse enthält die Logik zur Wahl des nächsten zu setzenden Elements. Sie nutzt die Methoden einer \texttt{TrackMap}, um die Bahn zu manipulieren und die richtigen Elemente zu verstecken oder hervorzuheben.
	\item[Versuchsaufbau:] Dieser Versuch baut direkt auf Prototyp \ref{subsub:prot-kugelbahnaufbau} auf, der eine ebenenweise Aufbauanleitung einer Kugelbahn ermöglicht. Dabei werden alle Kugelbahn-Elemente auf der selben Horizontalen hervorgehoben. In diesem Versucht soll die Bauanleitung nun einzelne Würfel in einer angemessenen Reihenfolge hervorheben.

	\textbf{Übersicht Klassen}\\
	Bis auf \texttt{TrackBuilder} sind die Klassen im Wesentlichen wie von Prototyp \ref{subsub:prot-kugelbahnaufbau}.
	\begin{itemize}
		\item \texttt{BasicCube}: erbt von \texttt{SCNNode}; erstellt einen halbtransparenten SceneKit Würfel; hat einen Status vom Enum \texttt{BasicCubeState} zugeordnet (normal, planned, active oder built), der die Darstellung der Elements bestimmt
		\item \texttt{MarbleTrack}: erbt von \texttt{SCNNode}, hält Koordinaten verschiedener Kugelbahnen und kann diese mit \texttt{BasicCube} Elementen erstellen, hat Methoden zur Positionierung und Ausrichtung der ganzen Kugelbahn, besitzt Referenzen auf die Elemente via \texttt{TrackMap}.
		\item \texttt{TrackMap}: hält ein Dictionary von Kugelbahn Elementen mit ihren Koordinaten als Schlüssel, bietet Methoden zum Mutieren des Dictionarys und um benachbarte Elemente zu erhalten
		\item \texttt{TrackBuilder}: nutzt eine \texttt{TrackMap} um den Status von Elementen der Kugelbahn zu verändern, implementierte einen Depth-first Algorithmus mittels eines Arrays als Stack um alle Elemente einer Kugelbahn nacheinander hervorzuheben
	\end{itemize}

	\textbf{Algorithmus zur Wahl des nächsten Elements}\\
	Die Klasse \texttt{TrackBuilder} erhält bei der Initialisierung eine Referenz auf die aktuelle \texttt{TrackMap}. Ein leeres Array für \texttt{BasicCube} Elemente dient als LIFO Datenstruktur. Beim Start der Bauanleitung werden alle Elemente der Kugelbahn ausgeblendet und das Element an der Stelle $(0,0,0)$ zum Stack hinzugefügt. Danach läuft jeder Schritt des Aufbaus wie folgt ab:
	\begin{enumerate}
		\item Wenn das Stack leer ist, sind alle Elemente gebaut und der Builder wird gestoppt.
		\item Der Status des zuletzt gebauten Element wird von \texttt{active} zu \texttt{built} geändert.
		\item Das letzte Element des Stacks wird entfernt (\texttt{stack.popLast()}) und als aktuelles Element festgelegt.
		\item Alle Elemente, die an das aktuelle angrenzen und nicht bereits gebaut wurden und nicht schon im Stack sind, werden hinten an das Stack angehängt (\texttt{stack.append(element)}) und deren Status auf \texttt{planned} gesetzt.
		\item Falls das Stack jetzt leer ist, wurden alle Elemente auf dieser Ebene gebaut. Der \texttt{currentLevel} wird erhöht und ein Element der neuen Ebene dem Stack hinzugefügt.
	\end{enumerate}

	Jeder dieser fünf Schritte wird in einer entsprechenden Methode abgehandelt. Ein solcher Algorithmus garantiert gegenüber einer zufälligen Wahl, dass (nach dem Setzen des ersten Elements) nur Elemente, die an bereits gebaute anschliessend, ausgewählt werden. Für die Tiefensuche spricht, dass Elemente neben dem zuletzt gesetzten bevorzugt werden und ganze Abschnitte einer Bahn zuerst fertig gebaut werden, bevor mit einem anderen Teil weitergemacht wird. Ein Algorithmus auf Basis der Breitensuche sprang in Versuchen stark zwischen allen Seiten der Bahn hin und her.

	\textbf{Ablauf der Bauanleitung}\\
	Sobald die Position einer Kugelbahn im \texttt{ViewController} fixiert wurde, wird ein Button aktiviert, mit dem der Benutzer die Aufbauanleitung starten kann. Dabei wird ein \texttt{TrackBuilder} mit der Map vom \texttt{MarbleTrack} initialisiert und gestartet. Der \texttt{TrackBuilder} bietet nach aussen drei Methoden an: \texttt{start()}, \texttt{step()} und \texttt{stop()}. Bei jeder weiteren Betätigung des Buttons ruft der Controller die \texttt{step()} Methode auf, welche einen Schritt des oben beschriebenen Algorithmus ausführt und so das nächste Element hervorhebt. Sind alle Elemente gebaut, gibt \texttt{step()} als Rückgabewert \texttt{false} und der Controller beendet die Bauphase.

\end{description}

}

\IfFileExists{58-KugelbahnErstellen}{
  \subsubsection{Schrittweiser Aufbau einer Bahn durch Benutzer}\label{subsub:prot-kugelbahneditor}
\begin{description}
	\item[Fragestellung:] Wie kann der Schrittweiser Aufbau einer Bahn durch Benutzer erfolgen?
	\item[Resultat:] Eine spezifische Klasse \texttt{TrackEditor} enthält die Logik zum augbauen einer neuen Bahn. Die Klasse nutzt die Methoden von \texttt{MarbleTrack}, um die Bahn zu manipulieren und neue Elemente der Bahn hinzuzufügen.
	\item[Versuchsaufbau:] Der Versuch baut auf dem Prototyp \ref{subsub:prot-kugelbahnaufbau} auf. Es wurde nicht nur versucht einen Editor zu erstellen sondern auch Wissensgewinne und Implementationen des vorherigen Prototyps einzubinden. Grundsätzlich können nur Elemente gebaut werden die direkt an einem bereits gebauten Element angrenzen. Zusätzlich können Elemente auf gebauten Elementen erstellt werden. 

	\textbf{Übersicht Klassen}
	Die Klasse \texttt{TrackEditor} wurde dem Demoprojekt hinzugefügt. Zusätzlich musste der \texttt{ViewController} und \texttt{MarbleTrack} angepasst werden. Die übrigen Klassen wie \texttt{BasicCube}, \texttt{Triple} und \texttt{TrackMap} sind die Klassen im Wesentlichen wie von Prototyp \ref{subsub:prot-kugelbahnaufbau}.
	\begin{itemize}
		\item \texttt{MarbleTrack}: erbt von \texttt{SCNNode}, hält Koordinaten verschiedener Kugelbahnen, kann eine solche Bahn mit \texttt{BasicCube} Elementen erstellen, verfügt über Methoden zur Positionierung und Ausrichtung der gesamten Bahn, besitzt Referenzen auf die Elemente der Bahn in einer \texttt{TrackMap}
		\item \texttt{TrackEditor}: nutzt die Klasse \texttt{MarbleTrack} um den Status von Elementen der Kugelbahn zu verändern oder Elemente der Bahn hinzuzufügen/zu entfernen. Beim hinzufügen von neuen Elementen wird berechnet welche umliegende Elemente als Mögliche Elemente anzuzeigen sind.
	\end{itemize}

	\textbf{Hinzufügen neuer Elemente}\\
	Die Klasse \texttt{TrackEditor} erstellt bei der Initialisierung das Root-Element auf der Referenz (0, 0, 0) in der aktuellen \texttt{MarbleTrack} Klasse. Nun wird folgender Algorithmus verwendet um neue Elemente hinzuzufügen:
	
	\begin{enumerate}
		\item Beim erstellen des Root-Elements werden die Umliegenden Felder geprüft und Elemente mit dem Status \texttt{planned} hinzugefügt.
		\item Mit dem Antippen von einem Element wird ein Hit-Test durchgeführt und prüft ob es sich um ein Element mit dem Status \texttt{planned} handelt. Falls dies der Fall ist wird der Status des Elements auf \texttt{build} gesetzt
		\item Anschliessend wird berechnet wo die nächsten neuen Elemente erstellt werden könnten und diese Werden über die Klasse \texttt{MarbleTrack} mit dem Status \texttt{planned} hinzugefügt.
	\end{enumerate}

	Bei der Besprechung des Prototyps wurde festgestellt, dass das Löschen von Elementen eine ähnliche Überprüfung benötigt. Grundsätzlich dürfen nur Elemente entfernt werden die keine Inseln in der Bahn verursachen bzw. es muss immer eine Verbindung zum Ursprung bestehen.

\end{description}

}

\subsection{Lösungsvarianten}
\textit{Verschiedene Lösungen (2-3) zur Umsetzung der Demo App beschreiben …}

\subsubsection{Bauanleitung}

Basierend auf den Prototypen \ref{subsub:prot-kugelbahnaufbau} und \ref{subsub:prot-kugelbahnaufbau2} kann eine schrittweise Bauanleitung erstellt werden.
Damit kann der Benutzer eine gespeicherte, virtuelle Kugelbahn auf eine Fläche projizieren und erhält dann eine Schritt-für-Schritt Anleitung zum physischen Nachbau der Bahn.
Dabei wird jeweils ein Element hervorgehoben, an dessen Stelle der Benutzer das echte Element hinstellen kann.
Das App erkennt dann selber, ob ein Element an den Ort gestellt wurde und geht über zum nächsten Schritt der Anleitung.
Da in Prototyp \ref{subsub:prot-physische-wuerfel} keine verlässliche Methode für die Erkennung und Nutzung von dreidimensionalen Objekten gefunden wurde, könnte alternativ der Benutzer manuell in der Anleitung vor- und zurückgehen.

\subsubsection{AR Baumodus}

\subsubsection{Simulation virtueller Kugel}

\subsubsection{Kugelbahnerkennung}

\subsection{Lösungswahl}
\textit{Entscheiden und begründen …}
