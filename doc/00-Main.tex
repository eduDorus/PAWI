%----------------------------------------------------------------------------------------
%	PACKAGES AND OTHER DOCUMENT CONFIGURATIONS
%----------------------------------------------------------------------------------------

\documentclass[11pt]{article} % Default font size is 12pt, it can be changed here
\pagestyle{headings}

\usepackage{geometry} % Required to change the page size to A4
\geometry{a4paper, total={6.4in, 9in}} % Set the page size to be A4 as opposed to the default US Letter
\usepackage{graphicx} % Required for including pictures
\usepackage{subcaption} % to use subfigure
\usepackage{wrapfig} % Allows in-line images such as the example fish picture
\usepackage[natbibapa]{apacite}  % APA Standard citing
\usepackage[ngerman]{babel}
\usepackage[utf8x]{inputenc} % utf8x instead of utf8 becuase "breaks" are not recognized in utf8
\usepackage{parskip} % For auto-paragraphs
\usepackage{color}
\usepackage{amsmath}
\usepackage{lscape} % For landscape content
\usepackage[hyphens]{url} % for linebreaking url's in sources MUSS VOR HYPERREF PACKAGE KOMMEN
\usepackage{hyperref} % For clickable table of contents
\usepackage[bottom]{footmisc} % For footnotes at them bottom
\usepackage{enumitem} % no separation between list items
\usepackage{soulutf8}
\usepackage{lastpage}
\usepackage{fancyhdr}
\usepackage{listings}
\usepackage{multicol}
\usepackage{multirow}
\usepackage{longtable}
\usepackage{tocloft} % allows for sources on figures in the list of figures
\usepackage{pdfpages}
\setlength{\columnsep}{0.6cm}

\pagestyle{fancy}
\renewcommand{\headrulewidth}{0.3pt}
\renewcommand{\footrulewidth}{0.3pt}

\usepackage{titlesec}
\newcommand{\sectionbreak}{\clearpage} % have sections start a new page
\newcommand{\tabitem}{\textbullet~~} % allows for lists inside tabular environment

\hypersetup{
    colorlinks,
    citecolor=black,
    filecolor=black,
    linkcolor=black,
    urlcolor=black
}

\oddsidemargin = 0pt

\makeatletter % Quellen in Abbildungsverzeichnis
\newcommand{\figsourcefont}{\footnotesize Quelle: }
\newcommand{\figsource}[1]{%
  \addtocontents{lof}{%
    {\leftskip\cftfigindent
     \advance\leftskip\cftfignumwidth
     \rightskip\@tocrmarg
     \figsourcefont#1\protect\par}%
  }%
 }
\makeatother

% alle Figures automatisch \centering
\makeatletter
\g@addto@macro\@floatboxreset\centering
\makeatother

% alle Figures und Tabellen automatisch an Position [htb]
\makeatletter
\renewcommand*{\fps@figure}{htb}
\renewcommand*{\fps@table}{htb}
\makeatother

% command für einfacheres Einfügen von einzelnen Bildern
\newcommand{\bild}[4][]{%
  \begin{figure}
    \centering
    \includegraphics[width=#2\textwidth]{#3}%
    \caption{#4}%
    \ifthenelse{\equal{#1}{}}{}{\figsource{\url{#1}}}%
    \label{fig:#3}%
  \end{figure}
}%

\addto{\captionsngerman}{	% Renew some Captions
% \renewcommand*{\contentsname}{Inhalt}
  \renewcommand*{\listfigurename}{Abbildungen}
  \renewcommand*{\listtablename}{Tabellen}
  \renewcommand*{\listoflistingscaption}{Codeblöcke}
  \renewcommand*{\figurename}{Abb.}
  \renewcommand*{\tablename}{Tab.}
  \renewcommand*{\listingscaption}{Code}
}

\usepackage[export]{adjustbox}
\usepackage{caption}
%\usepackage{fixltx2e} % for upper / subscript

\usepackage[scaled]{beramono}
\usepackage[T1]{fontenc}

\usepackage[breakwords]{truncate}


% Swift syntax highlighting code snippets
\usepackage{minted}
\newminted{swift}{ % define a custom minted environment, called with "swiftcode"
  autogobble,     % automatically removes unnecessary indentation
  breaklines,     % automatically breaks long lines on whitespace
  fontsize=\footnotesize, % self explanatory, it's the code's font size
  frame=single,   % add a single line for a frame around the code
  % gobble=2,       % remove first level of indentation (activate in case autogobble fails)
  linenos,        % show line numbers
  numbersep=6pt,  % smaller distance between linenumbers and frame
  style=xcode,    % use Xcode colorscheme
  tabsize=4,      % custom (narrower) indentation
  xleftmargin=15pt, % set left margin, so that the numbers don't hang outside
}
% styling of line number (gray, small, ttfamily)
\renewcommand{\theFancyVerbLine}{\textcolor[rgb]{0.6,0.6,0.6}{\scriptsize \ttfamily{\arabic{FancyVerbLine}}}}
% make inserting code even easier and enable label and caption
\newenvironment{code}[2]
{\VerbatimEnvironment
  \begin{listing}
    \caption{#2}\label{code:#1}
    \begin{swiftcode}%
}{
    \end{swiftcode}
  \end{listing}
}


\newenvironment{italicquotes} % Long quotes italic
{\begin{quote}\itshape}
{\end{quote}}

\linespread{1.2} % Line spacing
\setlength\parindent{0pt} % Uncomment to remove all indentation from paragraphs "Umfang ca..."
% \setcounter{tocdepth}{2} % only show 2 levels of section hierarchy in table of contents

\graphicspath{{images/}} % Specifies the directory where pictures are stored

\begin{document}

%----------------------------------------------------------------------------------------
%	TITLE PAGE
%----------------------------------------------------------------------------------------
\begin{titlepage}

\newcommand{\HRule}{\rule{\linewidth}{0.5mm}} % Defines a new command for the horizontal lines, change thickness here

\center

\textsc{\LARGE Hochschule Luzern\\Informatik}\\[1.5cm] % Name of your university/college
\textsc{\Large Informatikprojekt PAWI, FS 2018}\\[1.5cm] % Major heading such as course name

\HRule \\[1cm]
{\huge \bfseries iOS App mit Augmented Reality}\\[0.6cm] % Title of your document
\HRule \\[1cm]

\vspace{30pt}

\begin{minipage}{0.4\textwidth}
  \begin{flushleft} \large
    \emph{Autoren:}\\
    Dorus \textsc{Janssens}\\
    Lucas \textsc{Schnüriger}\\
    ~\\~ % for vertical alignment
  \end{flushleft}
\end{minipage}
\begin{minipage}{0.4\textwidth}
  \begin{flushleft} \large
    \emph{Dozent:}\\
    Prof. Dr. Ruedi \textsc{Arnold}\\~\\
    \emph{Experte:}\\
    Prof. Dr. Thomas \textsc{Koller}
  \end{flushleft}
\end{minipage}\\[1cm]


\vspace{30pt}

{\large \today }\\[1cm] % Date, change the \today to a set date if you want to be precise

\end{titlepage}


%----------------------------------------------------------------------------------------
%	Vor-Zeugs
%----------------------------------------------------------------------------------------
\pagenumbering{Roman}
\pagestyle{plain}

\IfFileExists{01-Selbststaendigkeitserklaerung}{
  \section*{Selbstständigkeitserklärung}
Hiermit erkläre ich, dass ich die vorliegende Arbeit selbstständig angefertigt und keine anderen als die angegebenen Hilfsmittel verwendet habe. Sämtliche verwendeten Textausschnitte, Zitate oder Inhalte anderer Verfasser wurden ausdrücklich als solche gekennzeichnet.

\vspace{1cm}

Rotkreuz, 8. Juni 2018 

\vspace{2cm}

\parbox{\textwidth}{
  \parbox{7cm}{
    \rule{6cm}{.5pt}\\
    Dorus Janssens
  }
  \hfill
  \parbox{7cm}{
    \rule{6cm}{.5pt}\\
    Lucas Schnüriger
  }
}

}

\newpage
\IfFileExists{02-Abstract}{
  \section*{Abstract}

\subsection*{Deutsch}

% Relevanz
Anlässlich der Entwicklerkonferenz WWDC 2017 hat der Technologiekonzern Apple neue Frameworks zur Nutzung von Augmented Reality, Machine Learning und Computer Vision auf mobilen Geräten veröffentlicht.
Die Forschungsgruppe Algorithmic Business (ABIZ) der Hochschule Luzern - Informatik hat verschiedene Forschungsprojekte und viel Know-How in diesen Bereichen.
% Thema / Problemstellung
Am Beispiel von Holzkugelbahnen soll aufgezeigt werden, welche Möglichkeiten die neuen Frameworks von Apple bieten.
% Vorgehen Methode
Diese Arbeit erforscht Zusammenspiel und Grenzen dieser Technologien mit dem Bau von Prototypen und veranschaulicht die Ergebnisse in einer Demo-App. 
% Ergebnisse
Anhand den Prototypen konnten die Limitationen der Frameworks aufgezeigt werden.
Mit der iOS Demo-App "`ARMarbleRun"' können durch die Nutzung von ARKit 1.5 virtuelle Kugelbahnen in Augmented Reality erstellt und bearbeitet werden.
Weiter kann eine physische Kugelbahn mit Hilfe einer virtuellen Bauanleitung aufgebaut werden.
% Folgerungen
Die Demo-App demonstriert, dass ein produktiver Einsatz von ARKit durchaus möglich ist, aber im Funktionsumfang noch limitiert ist.
Eine direkte Interaktion mit dreidimensionalen Objekten ist mit dem verwendeten ARKit 1.5 nicht gegeben.

\subsection*{English}

At the developer conference WWDC 2017, technology company Apple released new frameworks for the use of augmented reality, machine learning and computer vision on mobile devices.
The research group Algorithmic Business (ABIZ) of the Lucerne School of Information Technology has various research projects and a lot of know-how in these areas.
The possibilities offered by the new frameworks from Apple should be demonstrated by the example of wooden marble runs.
This work explores the interaction and limitations of these technologies with the creation of prototypes and illustrates the results in a demo app.
Based on the prototypes, the boundaries of the frameworks could be demonstrated.
The iOS demo app ``ARMarbleRun'' can be used to create and edit virtual marble runs in augmented reality.
Furthermore, a physical marble run can be built with the aid of a virtual guide.
The demo app demonstrates that a productive use of ARKit is possible, but its amount of functions is still limited.
A direct interaction with three-dimensional objects is not given with ARKit 1.5.

}

\newpage
\IfFileExists{03-Begriffe}{
  \section*{Begriffe \& Abkürzungen}

\begin{table}[htb!]
	\begin{tabular}{@{} p{.18\textwidth} p{.78\textwidth} @{}}
		\hline
		\textbf{Begriff} & \textbf{Erklärung} \\
		\hline
		ABIZ	& Forschungsgruppe Algorithmic Business der Hochschule Luzern - Informatik \\
		AR 		& Augmented Reality, computergestützte – primär visuelle – Erweiterung der Realität. Siehe Kapitel \ref{sub:augmented-reality}. \\
		Bounding Box & Hüllkörper, ein Quader oder Rechteck, das ein Objekt vollständig umschliesst \\
		cuboro	& Schweizer Unternehmen, das Kugelbahnen aus Holz herstellt \\
		CV		& Computer Vision, maschinelles Sehen zur Interpretation von Bildern und Erkennung von Objekten. Siehe Kapitel \ref{sub:computer-vision}. \\
		Deep Learning & Eine Methodik der Maschine Learning Disziplin \\
		Engine	& Ein Programm, das spezifische komplexe Berechnungen oder Simulationen durchführt \\
		Framework	& Ein Programmiergerüst, das dem Entwickler Funktionen und Strukturen bereitstellt \\
		HSLU	& Hochschule Luzern \\
		IDE 	& Integrated Development Environment, unterstützt den Entwickler bei der Softwareentwicklung \\
		Interface Builder & Grafischer Editor in Xcode zur Erstellung von Benutzeroberflächen \\
		iOS		& Apples mobiles Betriebssystem für iPhone und iPad \\
		ML		& Machine Learning, maschinelles Lernen und Gewinnung von Wissen aus Erfahrungen und Lerndaten durch den Computer \\
		PAWI	& Informatikprojekt and der HSLU Informatik \\
		Storyboard & Dateiformat für iOS Benutzeroberflächen, können im Interface Builder grafisch editiert werden \\
		Swift	& Programmiersprache, Open Source, von Apple für deren Betriebssysteme. Siehe Kapitel \ref{sub:swift}. \\
		Tracking & Verfolgung der Bewegung und Position eines Objektes \\
		UI & Grafische Benutzeroberfläche. \\
		VR		& Virtual Reality, eine interaktive virtuelle Umgebung, die in Echtzeit generiert wird. Siehe Kapitel \ref{sub:virtual-reality}. \\
		WWDC	& Worldwide Developers Conference, jährliche Konferenz von Apple für Software-Entwickler, bei der der Konzern oft neue Produkte vorstellt \\
		Xcode	& Entwicklungsumgebung von Apple zur Entwicklung von Programmen für Apples Betriebssysteme. Siehe Kapitel \ref{sub:xcode}. \\
		\hline
	\end{tabular}
	\caption{Übersicht Begriffe \& Abkürzungen}
\end{table}

}

%----------------------------------------------------------------------------------------
%	TOC
%----------------------------------------------------------------------------------------
\newpage
\tableofcontents % Include a table of contents

%----------------------------------------------------------------------------------------
%	Content
%----------------------------------------------------------------------------------------
\newpage % Begins the essay on a new page instead of on the same page as the table of contents
\pagenumbering{arabic}

\pagestyle{fancy}
\lhead{I.BA\_PAWI.F18}
\chead{Hochschule Luzern – Informatik}
\rhead{\nouppercase{\leftmark}}
\lfoot{\today}
\cfoot{}
\rfoot{\thepage\ von \pageref{LastPage}}

\newpage
\IfFileExists{04-Einleitung}{
  \section{Einleitung}

Mobile Plattformen entwickeln sich stetig weiter und bieten auf der Hardware-Seite mit schnelleren Chips, besseren Sensoren (z.B. Kameras, Gyroskopen usw.) neue Möglichkeiten. Dazu folgen auf der Software-Seite mit neuen Anwendungen und erweiterten Programmierschnittstellen für die Endanwender bzw. die Software-Entwickler. Für letztere hat beispielsweise Apple an der letztjährigen Entwicklerkonferenz WWDC 2017 mit Core ML, Vision und dem ARKit drei neue Frameworks für maschinelles Lernen, Computer Vision und Augmented Reality vorgestellt für das mobile Betriebssystem iOS.


Die Forschungsgruppe Algorithmic Business (ABIZ) der Hochschule Luzern - Informatik hat verschiedene Forschungsprojekte und viel Know-How im Umfeld von maschinellem Lernen, Computer Vision und Augmented Reality. Im Rahmen von diesem Projekt sollen die Möglichkeiten von den drei oben genannten (und allenfalls weitern) iOS-Frameworks an einer exemplarischen Anwendung ausgelotet und dann in Form einer attraktiven Demo-App illustriert werden.


Die vorgegebene Anwendungsdomäne sind Holzkugelbahnen, ABIZ stellt dazu den Studierenden entsprechende Bausätze der Marke Cuboro zur Verfügung. Diese Firma stellt ebenfalls virtuelle Kugelbahnen inkl. Ball-Simulationen zur Verfügung, evtl. kann daran angeknüpft oder aufgebaut werden. Die zu erstellende App soll in einer Kombination von Objekt-Erkennung mit AR/VR eine physisch aufgebaute Kugelbahn auch digital erfassen, diese virtuell darstellen können (VR) und auch das physische Modell durch Überlagerungen anreichern (AR).
}

\IfFileExists{05-Problemstellung}{
  \section{Problemstellung}
In diesem Projekt soll eine iOS-App erstellt werden, welche es erlaubt mit einer physischen Kugelbahn zu interagieren. Einerseits soll es die zu erstellende App erlauben, physische Kugelbahnen digital zu erfassen. Diese Erfassung kann beispielsweise parallel zum Aufbau einer Kugelbahn geschehen, indem die App Bauteil um Bauteil diese zuerst erkennt (mit passenden Methoden aus dem Bereich Bilderkennung/Computer Vision bzw. dem maschinellen Lernen) und dann dessen relative Position innerhalb von der Kugelbahn aufzeichnet. Es sollen zwingend mehrere unterschiedliche Kugelbahnen in der App erfasst sein, ggf. werden diese in der Software fix hinterlegt bzw. können auf einem anderen Weg als mithilfe der App erfasst werden, z.B. durch einen Import aus dem Cuboro-Webkit.


Erfasste Kugelbahnen sollen nach Möglichkeit in AR-Manier als Echtzeit-Überblendung auf Kamerabildern der physischen Kugelbahn projiziert werden können, z.B. indem Bauteile eingefärbt werden, in Form von angezeigten "`Bounding-Boxes"' oder indem virtuelle Kugeln auf der physischen Kugelbahn simuliert werden können. Oder weiter könnten erfasste Kugelbahnen mittels AR beispielsweise auf einen physischen Tisch "`gestellt"' werden und darauf könnte das Rollen von virtuellen Kugeln simuliert werden. 


Das Hauptziel ist es, dass am Schluss eine attraktive und lauffähige Demo-App zur Verfügung steht. Die App soll möglichst intuitiv angewendet werden können und schön die Möglichkeiten (und Grenzen) dieser neuen Technologien illustrieren.


In dieser Arbeit gibt es einige Freiheitsgrade, sprich viele funktionale und technische Aspekte gilt es zu erarbeiten, evaluieren und fest zu legen. Damit das Endresultat den Erwartungen von Auftraggeber und Betreuer entspricht, sollen Evaluationen und Entscheidungsfindungen in enger Rücksprache mit dem Auftraggeber und dem Betreuer stattfinden. Wichtige Entscheide (wie z.B. Fixierung von funktionalen Anforderungen, Wahl von Frameworks, usw.) müssen vom Auftraggeber bzw. dem Betreuer genehmigt werden, entsprechend sollen diese möglichst proaktiv in Evaluationen, Bau von Prototypen usw. miteinbezogen, sowie über Vor- und Nachteile, Alternativen usw. informiert werden.

\subsection{Ausgangslage}

\subsection{Ziele}

\subsection{Abgrenzung (Scope)}

\subsection{Fragestellungen}

\subsection{Anforderungen}
M = Muss (Festanforderung), S = Soll (Wunschanforderung), K = Kann (Zusatzanforderung).

\subsubsection{Funktionale Anforderungen}

\begin{table}
	\begin{tabular}{l c p{5cm} p{9cm}}
		\hline
		Nr. & Prio. & Bezeichnung & Erläuterungen \\
		\hline
		01 & M & Erkennen einer Kugelbahn & Die App muss eine physisch aufgebaute Kugelbahn as solche erkennen. \\
		02 & S & Digitale Erfassung einer Kugelbahn & Mit der App kann eine physisch aufgebaute Kugelbahn erfasst und so digitalisiert werden. \\
		\hline
	\end{tabular}
	\caption{Funktionale Anforderungen}
	\label{tab:funktionale-anforderungen}
\end{table}

\subsubsection{Nichtfunktionale Anforderungen}

\begin{table}
	\begin{tabular}{l c p{5cm} p{9cm}}
		\hline
		Nr. & Prio. & Bezeichnung & Erläuterungen \\
		\hline
		01 & M & Lauffähige Demo & Die App ist lauffähig. \\
		02 & M & iOS und Swift & Die App ist in Swift für iOS programmiert. \\
		03 & M & Recherche ARKit, Core ML, Vision & Die Dokumentation enthält Informationen zu den iOS Frameworks ARKit, Core ML und Vision. \\
		\hline
	\end{tabular}
	\caption{Nichtfunktionale Anforderungen}
	\label{tab:nichtfunktionale-anforderungen}
\end{table}

\subsection{Lösungsskizze}

}

\IfFileExists{06-Stand-der-Technik}{
  \section{Stand der Technik}

\subsection{Swift}
https://developer.apple.com/swift

\subsection{Core ML}
https://developer.apple.com/documentation/coreml

\subsection{Vision}
https://developer.apple.com/documentation/vision

\subsection{ARKit}
https://developer.apple.com/documentation/arkit

\subsection{SceneKit}
https://developer.apple.com/documentation/scenekit

\subsection{SpriteKit}
https://developer.apple.com/documentation/arkit/arskview/providing\_2d\_virtual\_content\_with\_spritekit
}

\IfFileExists{07-iOS-Frameworks}{
  \section{iOS Frameworks}

\subsection{Core ML}
https://developer.apple.com/documentation/coreml

\subsection{Vision}
https://developer.apple.com/documentation/vision

\subsection{ARKit}
https://developer.apple.com/documentation/arkit

\subsection{SceneKit}
https://developer.apple.com/documentation/scenekit

\subsection{SpriteKit}
https://developer.apple.com/documentation/spritekit

}

\IfFileExists{08-Loesungsentwicklung}{
  \section{Lösungsentwicklung}

\subsection{Prototypen}
\textit{Beschreibung der einzelnen Prototypen …}

In diesem Abschnitt folgen die verschiedenen Prototypen, die während der explorativen Phase dieses Projekts erstellt wurden.

\IfFileExists{51-OverlaySceneKit}{
  \subsubsection{Overlay auf einem Würfel mit SceneKit}\label{subsub:prot-overlay}
\begin{description}
	\item[Fragestellung:] Wie kann ein physischer Würfel (ein cuboro Element) mittels AR mit einer Art Overlay hervorgehoben werden? Eignet sich SceneKit dazu?
	\item[Resultat:] Ein mittels SceneKit modellierter Würfel lässt sich an der Stelle eines physischen Würfels platzieren und kann so als Möglichkeit diesen hervorzuheben eingesetzt werden. % TODO: Verweis auf entsprechenden Prototyp (Xcode Projekt) im Repo, bzw. vollständigen Code im Anhang(?)
	\item[Versuchsaufbau:] Dieser Versuch wurde noch mit ARKit 1.0 umgesetzt. Der Ansatz dieses Prototyps ist es, mittels ARKit einen halbtransparenten SceneKit Würfel an die Stelle eines physischen Elements zu positionieren. Die Positionierung wird in diesem Versuch manuell gemacht. Die Basis dieses Versuchs bildet der Beitrag "`Detecting Planes and Placing Objects"' auf dem Blog Machinethinks (\cite{arkit-dectingplanes-placingobjects}) und dem zugehörigen Projekt auf GitHub. Zuerst soll die richtige Ebene der realen Welt definiert werden und im zweiten Schritt wird dann der Würfel darauf platziert.

	\textbf{Flächenerkennung}\\
	Beim Erstellen eines neuen AR Projekts in Xcode kann die gewünschte Technologie für den Inhalt ausgewählt werden. Zur Verfügung stehen neben SceneKit auch SpriteKit und Metal. Die Basis bildet \texttt{ARSCNView}, ein ViewController, der ARKit mit dem SceneKit vereint. In Code \ref{code:prot-overlay-viewdidload} wird die Konfiguration und der Start der \texttt{ARSession} vorgenommen. Da der Würfel auf eine Fläche gestellt werden soll, wird die horizontale Flächenerkennung (Zeile 2) aktiviert.

	\begin{code}{prot-overlay-viewdidload}{Konfiguration und Start der \texttt{ARSession}}
		configuration = ARWorldTrackingConfiguration()
		configuration!.planeDetection = ARWorldTrackingConfiguration.PlaneDetection.horizontal
		sceneView.session.run(configuration!, options: [ARSession.RunOptions.removeExistingAnchors, ARSession.RunOptions.resetTracking])
	\end{code}

	Sobald ARKit eine Fläche detektiert hat, wird die Methode \texttt{renderer(\_:nodeFor:)} des Protokolls \texttt{ARSCNViewDelegate} aufgerufen, die als Rückgabe eine SceneKit Node (\texttt{SCNNode}) für den gefundenen Anker erwartet. Um die erkannte Fläche in der Szene anzuzeigen, wird eine \texttt{SCNBox} mit den geometrischen Ausmassen der Fläche erstellt. Diese Werte sind als Ausdehnung in den X- und Z-Achsen im Attribut \texttt{planeAnchor.extent} enthalten (Code \ref{code:prot-overlay-renderer}, Zeile 4). 

	%TODO: Maybe remove? Else implement method header
	\begin{code}{prot-overlay-renderer}{Implementation der \texttt{renderer(\_:nodeFor:)} Methode zur Darstellung von Flächen}
		var node:  SCNNode?
		if let planeAnchor = anchor as? ARPlaneAnchor {
			node = SCNNode()
			let planeGeometry = SCNBox(width: CGFloat(planeAnchor.extent.x), height: planeHeight, length: CGFloat(planeAnchor.extent.z), chamferRadius: 0.0)
			planeGeometry.firstMaterial?.diffuse.contents = UIColor.green
			planeGeometry.firstMaterial?.specular.contents = UIColor.white
			planeGeometry.firstMaterial?.transparency = 0.1
			let planeNode = SCNNode(geometry: planeGeometry)
			planeNode.position = SCNVector3Make(planeAnchor.center.x, Float(planeHeight / 2), planeAnchor.center.z)
			node?.addChildNode(planeNode)
			anchors.append(planeAnchor)
		}
		return node
	\end{code}

	\textbf{Würfel modellieren}\\
	Jedes SceneKit Element ist eine \texttt{SCNNode} mit einer bestimmten Geometrie und Position. Für würfelförmige Körper bietet sich als Geometrie eine \texttt{SCNBox} an. Die Definition kann so einfach wie folgt sein:
	\mint[style=xcode,breaklines]{swift}{let node = SCNNode(geometry: SCNBox(width: 0.05, height: 0.05, length: 0.05, chamferRadius: 0.0))}
	Damit der Würfel als Overlay wirken kann, sollen die Seitenflächen halbtransparent sein, die Kanten aber hervorgehoben. Zur einfacheren Kontrolle über diese Eigenschaften implementieren wir daher in diesem Versuch die Klasse \texttt{BasicCube} vom Typ \texttt{SCNNode}, bestehend aus einem halbtransparenten Würfel und einem ChildNode mit nur den Kanten als Textur. Dieser Effekt wird mit einem speziellen Shader Modifier erzielt (von \cite{so-shader-modifier}).

	\textbf{Würfel platzieren}\\
	In einem ersten Test wurde beim Erkennen einer Fläche direkt ein \texttt{BasicCube} in das Zentrum der Fläche (\texttt{planeAnchor}) gestellt, indem dem \texttt{position} Attribut der Node folgendes Statement zugewiesen wurde:
	\mint[style=xcode,breaklines]{swift}{SCNVector3Make(planeAnchor.center.x, Float(cubeSize / 2), planeAnchor.center.z)}
	Der Ausgangspunkt von Nodes befindet sich in ihrem Zentrum, weshalb die Höhenkoordinate Y auf die halbe Würfelhöhe gesetzt werden muss. So kommt der Würfel direkt auf der Fläche zu stehen.

	\textbf{Würfel auswählen}\\
	Aus den so in die Mitte aller erkannten Flächen platzierten Nodes soll eine vom Benutzer ausgewählt werden. Beim Überschreiben der Methode \texttt{touchesBegan(\_:with:)} erhalten wir die Position eines Taps des Benutzers in \texttt{touches.first!.location(in:)} als \texttt{CGPoint}. Mit diesem Wert lässt sich auf die \texttt{ARSCNView} ein SceneKit-Hittest machen:
	\mint[style=xcode,breaklines]{swift}{let hitResults = sceneView.hitTest(location, options: [:])}
	Die Rückgabe enthält alle Nodes der Szene, die sich an diesem 2D-Punkt auf dem Bildschirm befinden, in der Reihenfolge, wie sie dargestellt sind. Der erste \texttt{BasicCube} entspricht also dem ausgewählten Würfel und so kann dessen Farbe verändert werden um die Auswahl zu visualisieren.

	\textbf{Fläche bestimmen}\\
	Um eine alternative Methode der Würfel-Platzierung auszuprobieren soll zunächst durch Benutzereingabge eine bestimmte Fläche bestimmt werden. Dies wird wiederum mittels eines Hittest realisiert, doch diesmal von ARKit:
	\mint[style=xcode,breaklines]{swift}{let hitResults = sceneView.hitTest(location, types: .existingPlaneUsingExtent)}
	Der zweite Parameter vom Typ \texttt{ARHitTestResult.ResultTypes} definiert, worauf der Hittest gemacht wird. Hier sollen bereits detektierte Flächen unter Berücksichtigung ihrer Ausmasse betrachtet werden, damit der Nutzer eine der erkannten Flächen auswählen kann. Von allen anderen Flächenanker werden die Nodes mit \texttt{removeFromParentNode()} der Szene entfernt und die Anker selber mit \texttt{sceneView.session.remove(anchor:)} aus der Session gelöscht, sodass sie nicht mehr weiter von ARKit getrackt werden.

	\textbf{Manuelle Platzierung von Würfeln}\\
	% TODO: Screenshots einfügen
	Sobald eine Fläche bestätigt wurde, wird nun ein Positionierungswürfel in der Mitte des Bildschirms angezeigt. Dieser Würfel befindet sich immer auf der bestimmten horizontalen Ebene und dreht sich mit der Kamera. Ein Tap auf den Bildschirm platziert dann eine Kopie des Würfels fix an der aktuellen Position. Damit kann der Positionierungswürfel genau deckungsgleich mit einem physischen Würfel gebracht und dann eine Kopie davon als Overlay hinterlassen werden. Code \ref{code:prot-overlay-positioningcube} zeigt die beiden dazu verwendeten Methoden \texttt{updatePositioningCube()} und \texttt{placeCube()}. Erstere wird stets in der Methode \texttt{renderer(\_:willRenderScene:atTime:)} aufgerufen, sodass er sich mit der Kamera bewegt. Als erstes wird erneut ein AR-Hittest vom Zentrum des Bildschirms gemacht (Zeile 2, \texttt{screenCenter}, Attribut, dem \texttt{sceneView.center} zugewiesen wurde). Auf die erhaltenen Stelle wird die Position des Würfels gesetzt (Zeilen 4-7). Die Rotation erhält man von \texttt{sceneView.pointOfView} als Euler Winkel (Zeilen 8-9). Jedoch schien dies im Versuch nur für zirka 180° zu funktionieren. Das seitliche Drehen des Geräts und somit des Blickwinkels um mehr als einen Viertelkreis in die eine oder andere Richtung lässt den Würfel dann in die falsche Richtung herum rotieren. Wie sich dies korrigieren lässt, wurde in diesem Versuch nicht mehr herausgefunden.

	\begin{code}{prot-overlay-positioningcube}{Positionierungswürfel in der Mitte des Bildschirms}
		func updatePositioningCube() {
			let hitResults = sceneView.hitTest(screenCenter!, types: .existingPlane)
			if hitResults.count > 0 {
				let result : ARHitTestResult = hitResults.first!
				let coords = result.worldTransform.columns.3
				let newLocation = SCNVector3Make(coords.x, (coords.y + Float(positioningCube!.sidelength/2)), coords.z)
				positioningCube!.position = newLocation
				let cameraRotation = sceneView.pointOfView!.eulerAngles
				positioningCube!.eulerAngles = SCNVector3(0, cameraRotation.y, 0)
			}
		}
		
		func placeCube() {
			if positioningCube != nil {
				let cube = BasicCube(withColor: UIColor.green)
				cube.position = positioningCube!.position
				cube.eulerAngles = positioningCube!.eulerAngles
				sceneView.scene.rootNode.addChildNode(cube)
				boxes.append(cube)
			}
		}
	\end{code}

	Um nun einen Würfel an der gewünschten Stelle zu positionieren, wurde aus der Methode \texttt{touchesBegan(\_:with:)} heraus \texttt{placeCube()} aufgerufen. Dort wird ein neuer \texttt{BasicCube} erstellt (Code \ref{code:prot-overlay-positioningcube}, Zeile 15) und mit den Positions- und Winkeleigenschaften des Positionierungswürfel beschrieben (Zeilen 16-17).

\end{description}

}

\IfFileExists{52-PhysischeWuerfelErfassen}{
  \subsubsection{Physischen Würfel als virtuelles Objekt erfassen}\label{subsub:prot-physische-wuerfel}
\begin{description}
	\item[Fragestellung:] Wie kann ein physischer Würfel mittels den Frameworks ARKit, Vision oder CoreML als virtuelles Objekt erfassen werden?
	\item[Resultat:] ARKit bietet die Erkennung zweidimensionaler Elemente. Es verwendet dabei praktiken wie beim bekannten Bilderkennungsframework OpenCV, wobei ein Referenzbild hinterlegt werden muss.
	Mit Vision und CoreML kann ein beliebig trainiertes neurales Netzwerk verwenden, um die Klassifikation durchzuführen. Es konnte leider keine solide Möglichkeit gefunden werden 3D Objekte zu erfassen, damit Sie für eine spätere Augmentierung verwendet werden können. 
	\item[Versuchsaufbau:] Für den Versuchsaufbau wurden zwei Beispielprojekte von der Apple Developer Dokumentation verwendet. Das erste Projekt "`Recognizing Images in an AR Experience"' \cite{arkit-recognize-images} verspricht bekannte 2D Bilder mittels ARKit zu erkennen. Anschliessend können die erkannten Koordinaten verwendet werden um AR Inhalte zu platzieren.
	Beim zweiten Beispielprojekt handelt es sich um das Thema "`Using Vision in Real Time with ARKit"' \cite{vision-real-time-with-arkit} bei dem die Frameworks Vision und CoreML zum Einsatz kommen.

	\textbf{Beispielprojekt "`Recognizing Images in an AR Experience"'} \\
	Das Beispielprojekt kann von der Apple Developer Website heruntergeladen werden. Anschliessend lässt sich das Projekt in XCode öffnen und muss vor der Verwendung auf dem eigenen Gerät signiert werden. Das Verzeichnis "`Resources"' befinden sich bereits einige Demobilder die als Testversuch verwendet werden können. Um die Genauigkeit und Geschwindigkeit zu testen, wurde ein Versuch gestartet indem die Demobilder am Laptop angezeigt wurden. Anschliessend kann die Kamera des IPhones auf den Laptop ausgerichtet werden um den Erkennungsprozess zu starten. Der Versuch wiederspiegelte, dass die Erkennung schnell und zuverlässig erfolgte. Die erkannte Fläche erhielt eine weiss durchsichtig augmentierte Fläche an der stelle wo sich das Bild befindet. Ebenfalls diese augmentierte Fläche wurde korrekt angezeigt. Es wurde festgestellt, dass beim bewegen des IPhones die Fläche nicht exakt gehalten werden kann.

	Darauf Folgend wurde ein eigenes Bild für die Erkennung eines cuboro Elements hinterlegt. Der Prozess wie ein eigenes Bild beigefügt werden kann wird im README.md des Beispielprojektes "`Using Vision in Real Time with ARKit"' detailliert Erklärt. Beim Versuch wurden folgende Schritte durchlaufen:

	\begin{enumerate}
		\item Ein frontal Bild des cuboro Elements aufnehmen. 
		\item Das Bild bearbeitet dass nur die Würfelfläche beibehalten bleibt.
		\item Das Bild der ressourcen Gruppe im XCode hinzufügen.
		\item Neue Version compilieren und auf das Testgerät geladen.
		\item Erkennung des Elements starten.  
	\end{enumerate}

	\bild{0.4}{cuboro-element-frontal}{Frontal Ansicht vom einem cuboro Element}


	Die Implementation der Erkennung wird in den folgenden Zeilen konfiguriert. ARKit stellt die Erkennung der Referenzbilder zur Verfügung, wobei keine weiteren Implementationsschritte notwendig sind. Es wird zuerst eine Referenz auf das Ressourcenverzeichnis erstellt. Anschliessend wird diese Referenz der \texttt{ARWorldTrackingConfiguration} mitgegeben mittels \texttt{.detectionImages}.
	\begin{code}{arkit-recognition-configuration}{Implementation der Erkennung von Referenzbilder mit ARKit}
	guard let referenceImages = ARReferenceImage.referenceImages(inGroupNamed: "AR Resources", bundle: nil) else {
		fatalError("Missing expected asset catalog resources.")
	}
	
	let configuration = ARWorldTrackingConfiguration()
	configuration.detectionImages = referenceImages
	session.run(configuration, options: [.resetTracking, .removeExistingAnchors])
	\end{code}

	Wenn ein Bild aus dem Ressourcenverzeichnis erkennt wurde, wird ein \texttt{ARImageAnchor} zurückgegeben. Ein \texttt{ARImageAnchor} enthält diverse informationen z.B über die Position im Koordinatensystem. Dies wird in diesem Beispiel verwendet um das augmentierte Fläche zu erzeugen. 

	%TODO: Code muss an der richtigen Stelle platziert werden.
	\begin{code}{augmentierte Fläche-renderer}{Implementation der \texttt{renderer(\_:nodeFor:)} Methode zur Darstellung von Flächen}
		func renderer(_ renderer: SCNSceneRenderer, didAdd node: SCNNode, for anchor: ARAnchor) {
			guard let imageAnchor = anchor as? ARImageAnchor else { return }
			let referenceImage = imageAnchor.referenceImage
			updateQueue.async {
				let plane = SCNPlane(width: referenceImage.physicalSize.width,
									height: referenceImage.physicalSize.height)
				let planeNode = SCNNode(geometry: plane)
				planeNode.opacity = 0.25
				planeNode.eulerAngles.x =  - .pi / 2
				planeNode.runAction(self.imageHighlightAction)
				node.addChildNode(planeNode)
			}

			DispatchQueue.main.async {
				let imageName = referenceImage.name ?? 
				self.statusViewController.cancelAllScheduledMessages()
				self.statusViewController.showMessage("Detected image (imageName)")
			}
		}
	\end{code}


	\textbf{Beispielprojekt "`Using Vision in Real Time with ARKit"'} \\
	Das zweite Beispiel bei diesem Versuch beschäftigt sich mit Vision, CoreML und ARKit. Die Bildaufnahmen von ARKit werden an Vision weitergegeben und anschliessend mittels einem trainierten neuralen Netzwerk(Inception v3, \cite{DBLP:journals/corr/SzegedyVISW15}) in CoreML ausgewertet. 

	Der Code \ref{code:arkit-recognition-session} ausschnitt startet die ARSession sowie den Klassifizierungsprozess. Da dieser Task leistungs intensive ist, können nur eine gewisse Anzahl Bilder analysiert werden. Es wird stets geprüft ob sich bereits ein Bild im Puffer befindet bevor ein neues Bild zur Auswertung freigegeben wird.

	\begin{code}{arkit-recognition-session}{Startet die ARSession und den Klassifizierungsprozess}	
	func session(_ session: ARSession, didUpdate frame: ARFrame) {
		guard currentBuffer == nil, case .normal = frame.camera.trackingState else {
			return
		}
		self.currentBuffer = frame.capturedImage
		classifyCurrentImage()
	}
	\end{code}

	In diesem Code ausschnitt wird eqw bereits trainiertes neurales Netzwerk in das CoreML Framework geladen. Es kann somit auch ein beliebig selbst trainiertes Netwerk verwendet werden.
	\begin{code}{CoreML-request}{Initialisierung der Klassifizierung mittels CoreML}
	private lazy var classificationRequest: VNCoreMLRequest = {
		do {
			let model = try VNCoreMLModel(for: Inceptionv3().model)
			let request = VNCoreMLRequest(model: model, completionHandler: { [weak self] request, error in
				self?.processClassifications(for: request, error: error)
			})
			request.imageCropAndScaleOption = .centerCrop
			request.usesCPUOnly = true
			return request
		} catch {
			fatalError("Failed to load Vision ML model: (error)")
		}
	}()
	\end{code}

	Das Antippen eines klassifizierten Objektes erstellt ein Label mit SpriteKit. Dies erfolgt mittels einem Hit Test der die genauen Koordinaten ausfindig macht und an dieser Stelle ein \texttt{ARAnchor} setzt. Dieser \texttt{ARAnchor} wird dazu verwendet das Label and dieser Stelle einzublenden und im Weltkoordinatensystem zu festigen.
	\begin{code}{hit-test-klassifizierte-objekte}{Erstellt ein Label beim Antippen von klassifizierten Objekten}
	@IBAction func placeLabelAtLocation(sender: UITapGestureRecognizer) {
		let hitLocationInView = sender.location(in: sceneView)
		let hitTestResults = sceneView.hitTest(hitLocationInView, types: [.featurePoint, .estimatedHorizontalPlane])
		if let result = hitTestResults.first {
			
			let anchor = ARAnchor(transform: result.worldTransform)
			sceneView.session.add(anchor: anchor)
			
			// Track anchor ID to associate text with the anchor after ARKit creates a corresponding SKNode.
			anchorLabels[anchor.identifier] = identifierString
		}
	}
	\end{code}

\end{description}

}

\subsection{Lösungsvarianten}
\textit{Verschiedene Lösungen (2-3) zur Umsetzung der Demo App beschreiben …}

\subsection{Lösungswahl}
\textit{Entscheiden und begründen …}

}

\IfFileExists{09-Umsetzung}{
  \section{Umsetzung}


\subsection{Konzept}

Zur Illustration der Möglichkeiten der AR Technologie soll eine lauffähige iOS Demo-App entstehen.
Im Abschnitt zur Lösungswahl (\ref{sub:loesungswahl}) wurde anhand der erarbeiteten Prototypen entschieden, dass die App primär zwei Anwendungsfälle enthalten soll:

\begin{itemize}
	\item Eine interaktive Bauanleitung, bei der eine virtuelle Kugelbahn in Augmented Reality als Anleitung für den Benutzer zum Bau der physischen Kugelbahn verwendet wird.
	\item Der Bau und die Bearbeitung von virtuellen Kugelbahnen in Augmented Reality.
\end{itemize}

Eine ausführliche Architekturdokumentation findet sich im Anhang \ref{appendix:architekturdokumentation}.
Darin enthalten sind die kompletten Funktionalen Anforderungen als User Stories mit Akzeptanzkriterien (\ref{appendix:funktionale-anforderungen}) und Nichtfunktionalen Anforderungen (\ref{appendix:nichtfunktionale-anforderungen}).

\bild{1}{mockup-flow}{Mockups der Bildschirme und deren Zusammenhänge}

\textit{Beschreibung der Funktionalitäten und des Aufbaus. Übersicht der wesentlichen Use Cases (mit Verweis auf User Stories im Anhang), Mockups usw.}
% TODO:

\subsection{Softwarearchitektur}

\subsubsection{VIPER Architektur}

In einem Artikel auf der Webseite obj.io von \cite{viper-objcio} wird als Alternative zu der Architektur MVC (Model-View-Controller) das Modell VIPER vorgestellt.
Die Architektur verfolgt einerseits das Ziel sogenannte "`Massive View Controllers"' zu vermeiden, bei denen zu viel Logik in die Controller von MVC gesteckt wird.
Andererseits ist es ein Versuch die von Robert C. Martin vorgeschlagene Clean Architecture in iOS umzusetzen (\cite{clean-architecture}).
Der Name VIPER ist ein Backronym, das für folgende Komponenten steht (Auflistung frei aus dem Artikel von \cite{viper-objcio} übersetzt):

\begin{itemize}
	\item \textbf{View:} zeigt an, was vom Presenter mitgeteilt wird und leitet Benutzerinteraktionen an diesen weiter
	\item \textbf{Interactor:} enthält die eigentliche Businesslogik
	\item \textbf{Presenter:} beinhaltet Logik für die View, um die Daten vom Interactor aufzubereiten und reagiert auf Benutzereingaben
	\item \textbf{Entities:} sind grundlegende Datenmodelle, die primär durch den Interactor genutzt werden
	\item \textbf{Router/Wireframe:} enthält die Navigationslogik und ist Verbindungsglied zwischen einzelnen Modulen/Bildschirmen
\end{itemize}

\bild[https://www.objc.io/issues/13-architecture/viper/]{0.8}{viper-diagram-objc}{VIPER Diagramm der Komponenten}

Abbildung \ref{fig:viper-diagram-objc} zeigt den Zusammenhang der Komponenten in der Grundidee von VIPER.
Pro Bildschirm wird grundsätzlich ein Modul erstellt, das aus den VIPER Komponenten besteht.
Über die Wireframes werden die entsprechenden Module aufgerufen.
Das erlaubt eine starke Trennung der verschiedenen Module und innerhalb der Module die Trennung von Verantwortlichkeiten.

% TODO: Wahl für VIPER begründen
% TODO: Verweis auf Architekturdokumentation Teil zum Stil und Sichten

\subsubsection{Aufbau} % TODO: bessere Bezeichnung finden

Es gibt Module (nach VIPER), Common, Entities, Nodes und die unabhängige About Screen.

\subsection{Module}

\textit{Übersicht der Module in einem Komponentendiagramm oder so. Der Ablauf und Zusammenhang der Module muss hier gezeigt werden, bevor sie im Detail erläutert werden. Pro Modul dann die Klassenübersicht zeichnen und jede Klasse beschreiben.}
% TODO:

Im Folgenden werden die einzelnen Module der Demo-App einzeln beschrieben.
Die App besteht aus folgenden VIPER-Modulen:

\begin{itemize}
	\item \textbf{SelectMode:} Startbildschirm mit der Auswahl zwischen Editor und Guide
	\item \textbf{MarbleRunList:} Auflistung der gespeicherten Kugelbahnen zur Auswahl durch den Benutzer, für den Editor wird zusätzlich die Option zum Erstellen einer neuen Bahn angeboten
	\item \textbf{AREditor:} AR Bildschirm für den Editor
	\item \textbf{ARGuide:} AR Bildschirm für die Bauanleitung
	\item \textbf{ElementList:} Auflistung der verfügbaren Elementtypen zur Auswahl durch den Benutzer (vom Editor verwendet)
\end{itemize}

Das Zusammenspiel der Module ist in Abbildung \ref{fig:viper-modules} ersichtlich.
Sie zeigt, von wo man zu welchen Modulen gelangt.

\bild{0.6}{viper-modules}{Module der Demo-App}

\subsubsection{Select Mode}

\subsubsection{Marble Run List}

\subsubsection{AR Guide}

\subsubsection{AR Editor}

\subsubsection{Element List}


\subsection{Persistenz}

Informationen zum Local Data Manager und den Entities.

}

\IfFileExists{10-Validierung}{
  \section{Validierung}

Die detailierten Anforderungen sind im Kapitel \ref{appendix:user-stories} verfügbar.

\begin{longtable}{l l p{10cm} l}
	\hline
	\textbf{Nr.} & \textbf{Prio.} & \textbf{Beschreibung} & \textbf{Status} \\
	\hline
	\textbf{1} & & \textbf{Allgemein} & \\
	\hline
	1.1 & M & Die erkannte Fläche wird angezeigt. Durch antippen der Fläche wird diese augewählt. Die Position der neu zu setzenden Kugelbahn orientiert sich an der ausgewählten Fläche. & Erfüllt \\
	1.2 & M & Die Kugelbahn wird auf der ausgewählten Fläche angezeigt. Die Kugelbahn befindet sich zental im Kamerabild. Mittels Tap kann die Bahn fixiert werden. & Erfüllt \\
	1.3 & M & Die Bahn ist parallel zur Kamera ausgerichtet und behält dies auch beim drehen des Gerätes bei. & Erfüllt \\
	1.4 & M & Auf dem AR Bildschirm wird oben links in schwarzer Schrift der aktuelle Modul aufgeführt. & Erfüllt \\
	1.5 & S & Diese Funktionalität wurde nicht implementiert. Als möglicher Workaround kann der Modul neu gestartet werden. & Nicht Erfüllt \\
	1.6 & K & Es besteht die Möglichkeit den Moduls im Menu direkt zu wechseln. Beim Wechsel bleibt die aktuelle Szene nicht bestehen und muss somit neu ausgerichtet werden. & Teilweise Erfüllt \\
	\hline
	\textbf{2} & & \textbf{Guide} & \\
	\hline
	2.1 & M & Alle persistierten Bahnen werden die Namen der Kugelbahnen in der View MarbleRuns angezeigt. Diese Bahnen können durch antippen ausgewählt werden. Nach erfolgreicher selektion wird die Bahn der AR View bereitgestellt. & Erfüllt \\
	2.2 & M & Beim Aufbau der Bahn wird das nächte physische zu platzierende Element rot transparent Angezeigt. Die restlichen Elemente werden weiss transparent Angezeigt. Anhand der transparenten Darstellung wird verdeutlicht welches Element für den Aufbau verwendet werden muss und in welcher Position es sich befinden muss  & Erfüllt \\
	2.3 & M & Falls nächste Schritte in der Bauanleitung zur Verfügung stehen, ist auf der AR View unten rechts ein oranger Knopf mit einem Pfeil nach rechts verfügbar. Mit diesem Knopf kann der nächste Schritt eingeleitet werden kann. & Erfüllt \\
	2.4 & S & Falls vorherige Schritte in der Bauanleitung zur Verfügung stehen, ist auf der AR View unten links ein oranger Knopf mit einem Pfeil nach links verfügbar. Mit diesem Knopf kann der vorherige Schritt angezeigt werden. & Erfüllt \\
	2.5 & K & Die Bauanleitung kann über das Menu oben rechts neu gestartet werden. Zuoberst im Menu befindet sich das neustarten des Baumodus unter "'Restart Guide`"'. & Erfüllt \\
	\hline
	\textbf{3} & & \textbf{Builder} & \\
	\hline
	3.1 & M & Zur Auswahl stehen 13 cuboro Elemente. Diese können mittels dem orangen Plus-Knopf in der unteren Mitte selektiert werden. & Erfüllt \\
	3.2 & M & Wird auf dem Startbildschirm der Editor-Modus ausgewählt so wird eine Liste von gespeicherten Bahnen angezeigt. Das erste Element dieser Liste ist ein Plus bei dem eine neue Bahn erstellt werden kann. Beim selektiern einer neuen Bahn kann der Namen der Bahn mittels einem Popover eingegeben werden. Die neue Bahn wird unter dem eingegeben Namen persistiert und steht ab diesem Zeitpunkt in der Liste zur Verfügung. & Erfüllt \\
	3.3 & M & Nach dem Mutieren einer Kugelbahn kann diese über das Menu oben links gespeichert werden. Dies geschieht über den Knopf "'Save`". & Erfüllt \\
	3.4 & M & & Erfüllt \\
	3.5 & M & Valide Position werden nach der Selektion eines Elementes auf der Kugelbahn als Bouding Boxes dargestellt. Die Bounding Boxes werden mittels einem Algorithmus nur an Stellen angezeigt, die auf der ersten Ebene direkt an einem bereits erstellten Element angrenzen oder oberhalb platziert werden können. & Erfüllt \\
	3.6 & M & Ein Element kann mittels dem antippen einer Bounding Box hinzugefügt werden. Es ist nur möglich Element an den Stellen einer Bounding Box hinzuzufügen. & Erfüllt \\
	3.7 & M & Durch das antippen von einem Element kann dieses selektiert werden. Bei erfolgreicher Selektion wird dieses rot transparent hervorgehoben. Anschliessend kann das Element durch Streichgesten in einer der drei Achsen in richtung der Geste um 90° rotiert werden. & Erfüllt \\
	3.8 & M & Durch das antippen und halten kann ein Kugelbahn element entfehrnt werden. Dies kann jedoch nur gemacht wenn die Kugelbahn zusammenhängend bleibt und physikalisch möglich ist. Es ist nicht Möglich element unterhalb von Elementen zu entfehrnen (keine schwebende Elemente). & Erfüllt \\
	3.9 & S & & Nicht Erfüllt \\
\end{longtable}
}

\IfFileExists{11-Schlussfolgerungen}{
  \section{Schlussfolgerungen}

\subsection{Lessons Learned}

Für die Erstellung der VIPER Architektur wurde kostbare Zeit aufgewendet.
Diese Architektur ist eher für grössere Projekte geeignet und womöglich wäre für die Demo-App ein Bottom-Up Ansatz angemessener gewesen.
Die Architektur hat uns aber geholfen, verschiedene Funktionalitäten und Aufgaben zu trennen und eine über alle Module konsistente Struktur einzuhalten.

Augmented Reality ist ein Schlagwort, das immer häufiger anzutreffen ist.
Die Arbeit mit ARKit hat im Voraus viele Erwartungen an deren Möglichkeiten geweckt, die leider nicht ganz erfüllt wurden (bspw. die 3D Objekterkennung).
Gerade die WWDC 2018, die in der letzten Projektwoche anstand, zeigte mit Apples Präsentation von ARKit 2, dass es noch eine junge Technologie ist und viele (rasche) Fortschritte und Verbesserungen noch vor uns stehen.
Es ist bereits klar, dass der selbe Projektauftrag mit ARKit 2 ganz neue Möglichkeiten haben würde.

\subsubsection{Persönliches Fazit von Dorus Janssens}
Das Informatikprojekt war ein spannender Einblick in die iOS Welt. Die Programmiersprache Swift und Xcode als Entwicklungsumgebung kennenzulernen war eindrücklich und hat meinen technischen Horizont erweitert. Wie bei jeder fremden Technologie ist der Anfang schwierig und hat sicherlich Zeit und Nerven gebraucht. Ich konnte jedoch immer auf die Hilfe von Lucas Schnüriger oder dem betreuenden Dozenten zählen.

Die Verwendung der VIPER Architektur im Gegensatz zum MVC war lehrreich und wird in zukünftigen Projekten ihre Spuren hinterlassen.

In meinen Augen bietet Augmented Reality ein Werkzeug Probleme anders und innovativ anzugehen. Die Entwicklerkonferenz WWDC 2018 und der Ausblick auf ARKit 2 verdeutlichten dieses Potenzial. Mit der neuen Version ist es möglich Objekte zu erkennen, wobei wir in diesem Projekt diese Limitation in der aktuellen Version aufzeigen konnte. 

Unter dem Strich hat das Projekt mit Lucas Schnüriger Freude bereitet und hat den gewünschten Lerneffekt erzielt.

\subsubsection{Persönliches Fazit von Lucas Schnüriger}

Das Projekt klang von Beginn an spannend und interessant.
Sich mit den verschiedenen sehr neuen iOS Technologien zu beschäftigen hatte gleich einen grossen Reiz.
Dank meinen Vorkenntnissen in der iOS Entwicklung fand ich mich zwar schnell zurecht, dennoch konnte ich durch die vertiefte Arbeit mit Swift und den iOS Frameworks sehr viel Neues lernen. % TODO: konkret??

Es war das erste Mal, dass ich ein Projekt im explorativen Vorgehen durchführte.
Die zweiwöchigen Zyklen zur Entwicklung unterschiedlicher Prototypen boten einen guten zeitlichen Rahmen und erlaubte es sich mit dem betreuenden Dozenten regelmässig abzusprechen.

Das Modul SAQT, das ich parallel zum Projekt besuchte, motivierte mich, mich etwas genauer mit der Architektur einer Software auseinander zu setzen.
Ich fand es interessant zu sehen, dass man auch für iOS Alternativen hat und durchaus vom vorgegebenen Standardmodell Model-View-Controller abweichen kann.
Dabei muss man jedoch den einen oder anderen Umweg in Kauf nehmen. % TODO: konkret??
Die VIPER Architektur hat uns angespornt, die Aufgaben innerhalb der Module klar zu trennen, war aber auch ein zusätzlicher Overhead, den wir etwas unterschätzt hatten.
Ich bin der Ansicht, dass es sich aber gelohnt hat und wir sonst eine wesentlich unübersichtlichere Struktur erhalten hätten.

\subsection{Ausblick}

Anbei werden mögliche Weiterführungen dieser Arbeit beschrieben.

\subsubsection{Objekterkennung}
Im Kapitel \ref{subsub:prot-boundingbox} wird ein Konzept vorgestellt, welches die Erkennung von einem Element innerhalb einer Bounding Box beinhaltet. Diese automatische Erkennung könnte den Ablauf beim Aufbau der Kugelbahn verbessern. Um diese Funktionalität umzusetzen muss recherchiert werden, wie solche Hit-Tests durchgeführt werden können. Zusätzlich muss die Genauigkeit der Hit-Tests evaluiert werden damit das Element in der Bounding Box richtig erkannt wird.

\subsubsection{Kugelsimulation}
Im Kapitel \ref{sub:scene-kit} wird erläutert, dass das SceneKit eine eingebaute physik Engien besitzt. Diese könnte als weiterführende Arbeit an der Demo-App implementiert werden. Dazu müssen die Beschaffenheiten und Geometrie der Elemente hinterlegt werden. Anschliessend kann die Funktionalität entwickelt werden eine virtuelle Kugel auf die Bahn zu platzieren. Mit der Interaktion einer virtuellen Kugel kann die Bahn die vollständige Funktionalität der physischen Kugelbahn abbilden.

\subsubsection{ARKit 2.0 und iOS 12}
Kurz vor Ende dieser Arbeit wurde die Entwicklerkonferenz WWDC2018 abgehalten. An dieser Konferenz war Augmented Reality ein ausgeprägtes Thema. Mit dem neuen iOS 12 Release wurde ebenfalls die neue ARKit Version 2.0 vorgestellt. ARKit 2.0 bringt eine eingebaute Funktionalität um dreidimensionale Objekte zu erkennen, bis zu vier Personen simultan mit der gleichen augmentierten Welt zu interagieren und eine verbesserte Leistung bei der Erkennung von horizontalen sowie vertikalen Flächen. Zusätzlich wurde ein neues Dateiformat vorgestellt, dass das Einfügen, Bearbeiten und Austauschen von dreidimensionalen Objekten vereinfacht. 

Mit der neuen ARKit Version 2.0 könnte das setzen der virtuellen Bahn mit dem Erkennen eines physischen Elements ersetzt werden. Dies würde zwar bedeuten, dass es notwendig ist eine Bahn zu besitzen jedoch kann der Grundstein oder die Bahn physikalisch ausgerichtet werden und denn Prozess für den Endanwender interaktiver gestalten.

}


%----------------------------------------------------------------------------------------
%	Bibliography
%----------------------------------------------------------------------------------------
\newpage
\section{Quellen und Verzeichnisse}
\bibliographystyle{apacite}
\bibliography{Quellen}
%------------------------------------------------
\newpage
%\section{Abbildungsverzeichnis}
\listoffigures
\addtocontents{lof}{\protect\thispagestyle{headings}}
%------------------------------------------------
\newpage
%\section{Tabellenverzeichnis}
\listoftables
\addtocontents{lot}{\protect\thispagestyle{headings}}
%------------------------------------------------
\newpage
%\section{Codeverzeichnis}
\listoflistings
\addtocontents{lol}{\protect\thispagestyle{headings}}


%----------------------------------------------------------------------------------------
%	Appendix
%----------------------------------------------------------------------------------------
\newpage
\pagestyle{headings}
\appendix
\setcounter{figure}{0}
\setcounter{table}{0}
\renewcommand{\thefigure}{\thesection\arabic{figure}}
\renewcommand{\thetable}{\thesection\arabic{table}}
\renewcommand{\thesubsection}{\thesection\arabic{subsection}}


\IfFileExists{90-Appendix}{
  \section{Anhang}

\subsection{Projektmanagement}

\bild{1}{timeline}{Meilenstein Planung}

\begin{longtable}{l l l}
	\hline
	\textbf{MS} & \textbf{Datum} & \textbf{Inhalt} \\
	\hline

	1	& 27. Februar 	& 
	\begin{tabular}[t]{@{} l @{}}
		\tabitem Kick-Off Meeting \\
		\tabitem Besprechung von Aufgabenstellung \\
		\tabitem Projektvorgehen und Meetings besprechen \\
	\end{tabular} \\
	\hline

	2	& 13. März 		& 
	\begin{tabular}[t]{@{} l @{}}
		\tabitem Projekt- und Risikomanagement \\
		\tabitem Grobplanung Projektablauf (Meilensteine) \\
		\tabitem Grobkonzept (grobe Erfassung Fragestellungen, Scope, Anforderungen) \\
		\tabitem Grundaufbau Dokumentation \\
	\end{tabular} \\
	\hline

	3	& 8.  Mai 		& 
	\begin{tabular}[t]{@{} l @{}}
		\tabitem Besprechung Ideenfindung \\
		\tabitem Recherche von Swift und den relevanten Frameworks \\
		\tabitem Technische Grenzen definiert \\
		\tabitem Mehrere Problemlösungszyklen durchlaufen \\
		\tabitem Prototypen zu den Problemlösungszyklen \\
		\tabitem (funktionale) Anforderungen festgelegt \\
		\tabitem Planung der Umsetzung, Arbeitspakete \\
	\end{tabular} \\
	\hline

	4	& 22. Mai 		& 
	\begin{tabular}[t]{@{} l @{}}
		\tabitem Testatsitzung \\
		\tabitem Besprechung erreichter Ziele \\
		\tabitem funktionierende App, die min. alle Muss-Anforderungen erfüllt \\
		\tabitem Entwurf der abzugebenden Dokumentation \\
	\end{tabular} \\
	\hline

	5	& 8.  Juni 		& 
	\begin{tabular}[t]{@{} l @{}}
		\tabitem Abgabe Projekt \\
		\tabitem fertige Dokumentation \\
		\tabitem fertige App \\
		\tabitem Entwurf Schlusspräsentation \\
	\end{tabular} \\

	\hline
	\caption{Auflistung der Meilensteine}
	\label{tab:meilensteine}
\end{longtable}

\subsection{Aufgabenstellung}
\includepdf[pages=-]{appendix/Aufgabenstellung_PAWI_FS_2018_Janssens_Schnueriger.pdf}

\IfFileExists{99-Protokolle}{
	\newpage
	\subsection{Protokolle}

\setlist[itemize]{noitemsep}

\subsubsection*{22.05.2018 - Sitzung}

Anwesend: Ruedi Arnold, Lucas Schnüriger, Dorus Janssens

\textbf{Fragen}
\begin{itemize}
	\item Demo-App: Rotation um nur eine Achse auch akzeptabel?
	\begin{itemize}
		\item Noch einmal eine Stunde investieren, um zu sehen, ob die Rotation um alle drei Achsen nicht doch geht, bspw. mit einer Transformationsmatrix
	\end{itemize}
	\item Dokumentation:
	\begin{itemize}
		\item Wie sollen die (funktionalen) Anforderungen festgehalten werden: Teil des Hauptteil (Problemstellung) oder in Anhang (Architekturdokumentation)?
		\item Anforderungen können im Anhang gehandhabt werden. Sie sollen aber nicht doppelt aufgeführt werden.
	\end{itemize}
	\item Prototypen: die Fragestellungen und Resultate in einer Übersicht sammeln oder Teil der Prototypen Beschreibung oder gar an beiden Stellen?
	\begin{itemize}
		\item Kurze Liste mit Beschreib der Unterfangen, allenfalls Erfüllungsgrad visualisieren (bspw. mit Ampelsystem)
		\item Der bisherige Aufbau zur Beschreibung der Prototypen sonst ist ok
	\end{itemize}
	\item Umfang des Testing: Testplan und Testprotokoll oder einfach Anforderungen validieren? Wie dokumentieren: in Hauptteil/Anhang?
	\begin{itemize}
		\item Anforderungen validieren anhand der Akzeptanzkriterien, Erfüllungsgrad prüfen und Unvollständigkeiten begründen
		\item Festhalten im Hauptteil unter "Validierung" passt
		\item Unit Tests nicht notwendig
	\end{itemize}
\end{itemize}

\textbf{Weitere Arbeitsschritte / zu bearbeitende Fragestellungen}
\begin{itemize}
	\item Demo-App fertigstellen:
	\begin{itemize}
		\item Rotation nochmals überarbeiten und allenfalls darlegen warum es nicht funktioniert
		\item Löschalgorithmus fertig implementieren, sodass keine "Inseln" gebildet werden können
		\item Einen "About"-Bereich ergänzen (mit Name, Version)
		\item Das Schachbrettmuster der Flächenerkennung nach Start Editor/Guide nicht mehr anzeigen, oder transparenter
		\item Methoden der Klasse zur Persistierung statisch machen, nicht als Objekt benötigt
	\end{itemize}
	\item Dokumentation fertigstellen:
	\begin{itemize}
		\item Wichtig: Dokumentation muss eine objektive Bewertung und ein persönliches Fazit (pro Student) enthalten
		\item Die Projektplanung (Grobplan/Meilensteine) kann direkt in den Hauptteil der Dokumentation unter "Problemstellung"
		\item Protokolle, Arbeitsjournale, Risikomanagement im Anhang der Dokumentation mit abgeben aber nicht ausdrucken
		\item Aufgabestellung in den Anhang, Einleitung/Ausgangslage mit eigener Formulierung
		\item Source Dateien im Anhang (nicht nur von Prototypen/App, sondern auch von Bildern, Grafiken und Diagrammen)
		\item Den Backlog wenn möglich anhängen (vom GitHub Project nach Möglichkeit exportieren)
	\end{itemize}
	\item Demo Video erstellen
\end{itemize}

\textbf{Varia}
\begin{itemize}
	\item Zum letzten Statusprotokoll: der 1. und 3. Arbeitsschritt aus "Stand der Arbeit" würde in "Ausgeführte Arbeiten" kommen, "Stand der Arbeit" sollte SOLL-IST vergleichen
	\item Demo-App wurde vorgeführt, die App wurde zudem auf Herr Arnolds Gerät installiert
	\item Abgabe nicht auf CD notwendig, kann bspw. auch als Dropbox Link sein
	\item Arnold ist bis nächste Woche noch per Mail erreichbar, in Kalenderwoche 23 (4.-8. Juni) abwesend
	\item erhalten, Termin für Präsentation haben wir anschliessend an die Sitzung erhalten: 25. Juni, 14.00 Uhr)
\end{itemize}


\subsubsection*{08.05.2018 - Sitzung}

Anwesend: Ruedi Arnold, Lucas Schnüriger, Dorus Janssens

\textbf{Weitere Arbeitsschritte / zu bearbeitende Fragestellungen}
\begin{itemize}
	\item Demo App umsetzen
	\begin{itemize}
		\item englische Benutzeroberfläche, Mehrsprachigkeit nicht notwendig
		\item Architektur auf Basis VIPER ist ok, einfach nicht zuviel "Overhead" erzeugen, ein bottom-up Ansatz wäre auch ausreichend -> am Schluss ist die Funktionalität wichtig
	\end{itemize}
	\item Anforderungen in Dokumentation fertigstellen / Korrekturen anbringen:
	\begin{itemize}
		\item User Stories fortlaufend, bzw. eindeutig nummerieren
		\item bei Bauanleitung von "Schritten" nicht "Elementen" reden
		\item Anzahl integrierte Elemente: "mindestens" 4 schreiben
		\item Erstellen neuer Kugelbahn: ergänzen, dass man die Kugelbahn benennen kann, unter einem Namen speichert
		\item Platzieren neuer Element im Editor: präzisieren, wo das Platzieren möglich ist (nur valide Positionen)
		\item zusätzliche Soll oder Kann Anforderung zur Umbenennung von gespeicherten Kugelbahnen
		\item Ändern der Ausrichtung eines Element im Editor: spezifizieren, dass nur um 90° entlang der drei Achsen gedreht wird
		\item allg. Formulierungen kontrollieren und präzisieren wo nötig, darauf achten, dass sie unmissverständlich sind 
	\end{itemize}
	\item Mockups (UI):
	\begin{itemize}
		\item AR Screen überdenken: bspw. eine Navigationsbar oben inkl. zurück-Button und deutlicher Anzeige welcher Modus aktiv ist, was der nächste Schritt für den User ist (bspw. "Fläche auswählen")
		\item im Build Mode das aktuelle Element anzeigen, damit klar ist, welcher virtueller Würfel als nächstes gesetzt wird
	\end{itemize}
	\item Konzept für Demo Video erstellen
\end{itemize}

\textbf{Varia}
\begin{itemize}
	\item Dokumentation:
	\begin{itemize}
		\item Adäquate Deutsche Begriffe Anglizismen vorziehen (bspw. Speicher statt Memory oder Datei statt File)
		\item Begriffe konsistent/einheitlich verwenden
	\end{itemize}
	\item Die beiden Modi der App passender benennen:
	\begin{itemize}
		\item bisherigen Editier Modus als Build Modus, da in diesem Modus virtuelle Bahnen gebaut werden
		\item bisheriger Build Modus als Guide (oder ähnlich), da dies eine Anleitung ist
	\end{itemize}
	\item Laut Herrn Arnold hatte Herr Koller kürzlich wieder Kontakt mit cuboro. Sollten wir in den nächsten Tagen nichts hören, soll um ein Statusupdate bei Herr Koller nachgefragt werden
\end{itemize}


\subsubsection*{23.04.2018 - Sitzung}

Anwesend: Ruedi Arnold, Lucas Schnüriger, Dorus Janssens

\textbf{Weitere Arbeitsschritte / zu bearbeitende Fragestellungen}
\begin{itemize}
	\item Mockups fertigstellen
	\item PDF von den neuen Mockups Herr Arnold zukommen lassen
	\item Min. 4-5 Würfeltypen in die App integrieren
	\item Fertigstellen Konzept Demo-App
	\item Start Umsetzung Demo-App
\end{itemize}

\textbf{Varia}
\begin{itemize}
	\item Constraint setzen beim Löschen der Würfel im Editiermodus, damit keine Inseln gemacht werden können
	\item Demo an Herr Koller zeigen und Nachfragen ob eine Rückmeldung von cuboro eingetroffen ist
	\item Build- und Editiermodus mit visuellen Elementen für den Benutzer eindeutig identifizierbar machen (z.B Farbe)
	\item Mögliche Anschlusspunkte der Würfel hinterlegen und Anzeigen ob die Elemente zusammenpassen
	\item Für das Demo Video kann die Kugel in X- oder Z-Richtung angeschubst werden
	\item Mocks realistischer darstellen ohne Platzhalter  
	\item Simulation mit virtueller Kugel vorsehen (SceneKit Physikengine)
\end{itemize}

\subsubsection*{10.04.2018 - Sitzung}

Anwesend: Ruedi Arnold, Lucas Schnüriger, Dorus Janssens

\textbf{Weitere Arbeitsschritte / zu bearbeitende Fragestellungen}
\begin{itemize}
	\item AR Bauanleitung einer Kugelbahn Würfel-für-Würfel (Fortsetzung vom Prototyp zur Bauanleitung)
	\item Schrittweiser Aufbau einer augmentierten Bahn durch den Benutzer
	\item Korrektur der Position durch manuelles Verschieben der cuboro Bahn und Würfeln
	\item Konzept der Demo-Applikation erarbeiten
\end{itemize}

\textbf{Varia}
\begin{itemize}
	\item If-Else Konstrukt zum Drehen der Würfel in ein Enum oder einer Klasse verpacken (Strategie Pattern)
	\item Hit-Testing dokumentieren und im Ausblick als ein mögliches weiteres Projekt festhalten 
	\item Herr Koller wird von Dorus Janssens im Modul WebLab eine Demo der aktuellen Prototypen erhalten
	\item Für die Demo soll eine kleine Bahn verwendet werden (ca. 4 x 4 Grundfläche)
	\item in 2 Wochen geplantes Ende der explorativen Phase
	\item Neue Anforderung: Demofilm für die Endabgabe
\end{itemize}


\subsubsection*{27.03.2018 - Sitzung}

Anwesend: Ruedi Arnold, Lucas Schnüriger, Dorus Janssens

\textbf{Weitere Arbeitsschritte / zu bearbeitende Fragestellungen}
\begin{itemize}
	\item Vorschlag: Mehrere cuboro Elemente modellieren und eine einfache Bahn vordefinieren und augmentieren. Zweiter Schritt besteht darin mit der augmentierten Bahn zu interagieren.
	\item zu bearbeitende Themen:
	\begin{itemize}
		\item Würfel/Bahn modellieren: lässt sich cuboro-Webkit Export nutzen? Ansonsten cuboro anfragen (via Herr Koller), oder dann selber Wege zur Modellierung finden.
		\item virtuelles Modell einer Bahn auf einen Tisch projizieren
		\item AR Bauanleitung einer Bahn mit schrittweisem Aufbau
		\item erkennen ob sich innerhalb einer Bounding Box ein physischer Würfel befindet
		\item mittels Touchgesten virtuelle Objekte drehen
	\end{itemize}
\end{itemize}

\textbf{Varia}
\begin{itemize}
	\item Herr Arnold stets in CC nehmen bei Mails
	\item Herr Koller anfragen wegen Kontakt zu cuboro: für 3D Modelle der Würfel
	\item Versuche konkret dokumentieren: detailliert die Erkenntnisse, Versuche und Zwischenschritte festhalten; mit Screenshots, Codeausschnitten, Verweise auf Doku; im Hinterkopf behalten, dass damit jemand daran weiterarbeiten könnte, nachvollziehbar machen wieso welche Schritte genommen wurden; was wäre sonst noch möglich, Ausblick, alternative Möglichkeiten die nicht weiterverfolgt wurden
	\item Der Haupttreiber dieser Arbeit ist technisch, \textit{nicht} Business-Case
	\item Mail an Herr Arnold, wenn im GitHub Projects die Versuche erfasst sind
	\item Verweise auf die Projekte, die als Grundlage dienten, sollen im Repo enthalten sein; an passender Stelle (Readme, direkt im Code) erwähnen
\end{itemize}


\subsubsection*{13.03.2018 - Sitzung}

Anwesend: Ruedi Arnold, Lucas Schnüriger, Dorus Janssens

\textbf{Fragen}
\begin{itemize}
	\item Können wir die Einleitung und Aufgabenstellung in der erhaltenen Form verwenden?
	\begin{itemize}
		\item Die Einleitung und Aufgabenstellung können aktuell so beibehalten werden und gegen Ende des Projekts genauer spezifiziert werden.
	\end{itemize}
	\item Wie werden die funktionalen Anforderungen im explorativen Verfahren gehandhabt?
	\begin{itemize}
		\item Z.B. Overlays als Anforderung definieren und im Verlauf des Problemlösungszyklus einzelne Funktionalitäten näher festlegen.
		\item Somit ergeben sich grobe Anforderungen zu Beginn des Projektes und genauer spezifizierte Anforderungen nach dem Erkenntnisgewinn.
	\end{itemize}
	\item Planung / Vorgehen und Dokumentation bei explorativem Verfahren?
	\begin{itemize}
		\item Die Phasen Ideenfindung und Ideenauswahl zusammenlegen, sie laufen in kleineren Zeitrahmen immer wieder ab.
		\item Problemlösungszyklen einzeln dokumentieren und den Erkenntnisgewinn ausführlich formulieren.
	\end{itemize}
	\item In welcher Phase läuft der Problemlösungszyklus?
	\begin{itemize}
		\item Da die Ideenfindung und Ideenauswahl zusammengelegt wird ist dies ständig der Fall.
	\end{itemize}
\end{itemize}

\textbf{Weitere Arbeitsschritte / zu bearbeitende Fragestellungen}
\begin{itemize}
	\item Wie kann ein Overlay auf einem physischen Würfel erzeugt werden?
	\begin{itemize}
		\item Das SceneKit eignet sich für 3D Objekte und das SpriteKit für 2D Flächen. Welches ist passender?
	\end{itemize}
	\item Wie können physische Körper als virtuelle Objekte modeliert werden?
\end{itemize}

\textbf{Varia}
\begin{itemize}
	\item Öfters Prototypen oder ähnliches kommunizieren
	\item Ein Overlay auf einem Würfel erzeugen ist aktuell priorisiert
	\item Die Technischenaskpekte und Umsetzung stehen im Vordergrund
	\item Die Dokumentation soll vor allem die technische Funktionsweise (relevanter, wesentlicher Code) und neue Erkenntnisse aufzeigen
\end{itemize}

\subsubsection*{27.02.2018 - Kick-Off Meeting}

Anwesend: Ruedi Arnold, Lucas Schnüriger, Dorus Janssens

\begin{itemize}
	\item Herr Arnold hat uns den Sachverhalt des Projekts erklärt.
	\item Die Aufgabenstellung ist bewusst offen, es sollen die Möglichkeiten der AR Technologie mit Apples ARKit ausgelotet werden.
	\item Alle zwei Wochen wird ein Meeting abgehalten. Das Meeting findet jeweils 10:00 bis 11:00 statt.
	\item Falls benötigt kann ein IOS Gerät zur Verfügung gestellt werden.
	\item Das Projekt wird im explorativem Verfahren gehandhabt.
	\item Ein Set einer Cuboro Kugelbahn steht bei Herrn Thomas Koller zur Verfügung.
\end{itemize}


}

\IfFileExists{99-Risikomanagement}{
	\newpage
	\subsection{Risikomanagement}

\begin{longtable}{lp{3.5cm}lllp{3.5cm}p{3.5cm}}
	\hline
	\textbf{Nr.} & \textbf{Beschreibung} & \textbf{W} & \textbf{A} & \textbf{W*A} & \textbf{Prävention} & \textbf{Reaktion} \\ 
	\hline
	01  & Ausfall eines Projektmitarbeiters & 1 & 2 & 2 & Regelmässiger Wissensaustausch und Projektmeetings, zentrale Dokumentation aller Vorgänge (GitHub) & Betreuenden Dozenten informieren und Anforderungen einschränken \\
	\hline
	02  & Projektdaten gehen verloren & 0 & 3 & 0 & Daten regelmässig sichern, Daten versionisiert auf GitHub & Daten wiederherstellen \\
	\hline
	03  & GitHub fällt aus & 1 & 1 & 1 & lokale Kopien aller Daten & auf lokalen Kopien arbeiten und auf HSLU GitLab migrieren \\
	\hline
	04  & Anforderungen werden nicht erfüllt & 1 & 2 & 2 & regelmässiges Überprüfen / Meetings mit betreuendem Dozenten, realistische Ziele setzen & Rechtzeitige Information an Projektbeteiligte und Anpassung des Projektplans \\
	\hline
	05  & Zeitaufwand und Komplexität für Entwicklung zu hoch & 1 & 3 & 3 & eingehende Technologierecherche und Versuche & Betreuenden Dozenten informieren, um Hilfe bei Dozenten suchen, Anforderungen einschränken \\
	\hline
	06  & Frameworks weisen nicht die funktionalen Anforderungen auf die im Marketing versprochen werden. & 1 & 3 & 3 & Mittels Problemlösungszyklen die Funktionalität überprüfen und verifizieren & Rücksprache mit dem Auftraggeber und dem Betreuer \\
	\hline
	07  & Beschränkte Erfahrung mit iOS und Swift erhöht die Zeit der Evaluation und gefärdet somit den tiefe Grad der Problemlösungszyklen. & 1 & 3 & 3 & Genügend Zeit bei den ersten Problemlösungszyklen definieren & Anpassung der Anforderungen und Rücksprache mit dem Auftraggeber und dem Betreuer \\
	\hline
	08  & Betas von ARKit 1.5 oder iOS 11.3 sind instabil oder ändern genutzte Funktionen & 0 & 2 & 0 & Known Issues recherchieren, Verwendung von neuen Funktionen bewusst wählen und Alternativen berücksichtigen & Issue an Apple reporten, an anderen (gleich wichtigen) Aufgaben arbeiten, alternative Methoden suchen, auf stabile Funktionen zurückgreifen \\
	\hline
	09  & Leistung der (Test)Geräte nicht ausreichend & 1 & 2 & 2 & Technologische Anforderungen (welche Generation von iPhones/iPads) in der Recherche berücksichtigen, beim Bau von Prototypen Zeit einplanen zu Performance Tests & Code/Methoden optimieren, alternative Lösungen weiterverfolgen \\
	\hline
	10  & Präzision des Tracking zu niedrig um das Cuboro Element sicher zu Tracken & 2 & 2 & 4 & Anforderungen mit einem Tolleranzbereich definieren. & Marker anbringen um die präzision des Trackings zu erhöhen. Third party Software zur erkennung des Würfels verwenden. \\
	\hline
	11  & Präzision der Hit-Tests zu niedrig um zu ermitteln ob sich ein physisches Objekt in einer augmentierten Bounding Box befindet & 2 & 2 & 4 & Bestätigen mittels einer Tapgeste oder einem Button klick. & Verschiedene Versuche durchführen um die Genauigkeit zu ermitteln. \\
	\hline
	12  & Erhalten keine oder nicht verwendbar 3D Daten von cuboro Elemente & 1 & 2 & 2 & Kontakt mit cuboro herstellen & Einfache Würfel als Platzhalter, eigene einfache Elemente mit SceneKit oder Drittanbieter Software (bspw. Blender) erstellen \\
	\hline
	13  & Präzision des Worldtrackings ist zu instabil um die Bahn auf der ausgewählten Fläche zu halten & 2 & 2 & 4 & ARKit 1.5 nutzen & Funktion um die Bahm manuell zu korrigieren \\
	\hline
	14  & Korrekte Rotation der Elemente zu komplex & 2 & 2 & 4 & - & Anzahl Rotationsachsen verringern \\
	\hline
	15  & Worldtracking bei Unterbrüchen zu instabil & 2 & 2 & 4 & Unterbrüche der Szene vermeiden & Popup statt Modal verwenden um keinen Unterbruch zu erzeugen, Benutzer über Relokalisierung informieren \\
	\hline
\end{longtable}

\textbf{Legende}
\begin{itemize}
	\item \textbf{W:} Wahrscheinlichkeit
	\item \textbf{A:} Auswirkungen
\end{itemize}

\subsubsection*{Änderungsprotokoll}

\begin{table}
	\begin{tabular}{llp{10cm}}
		\hline
		\textbf{Datum} & \textbf{Autor} & \textbf{Bemerkungen} \\
		\hline
		06.03.2018 & Lucas & Erstellung der Tabelle mit Risiken 01-05 \\
		10.03.2018 & Dorus & Erweiterung um Risiken 06-07 \\
		12.03.2018 & Lucas & Erweiterung um Risiken 08-09 \\
		26.03.2018 & Lucas \& Dorus & Erweiterung um Risiko 10 \\
		26.03.2018 & Dorus & Risiko 07 W heruntergestuft von 2 auf 1 \\
		09.04.2018 & Dorus & Erweiterung um Risiko 11 \\
		09.04.2018 & Lucas & Risiko 08 W heruntergestuft von 1 auf 0, iOS 11.3 / ARKit 1.5 sind veröffentlicht \\
		22.04.2018 & Lucas & Erweiterung um Risiko 12 \\
		22.04.2018 & Lucas \& Dorus & Erweiterung um Risiko 13 \\
		07.05.2018 & Lucas & Risiko 10 heruntergestuft von 3 auf 2 \\
		07.05.2018 & Lucas & Risiko 12 heruntergestuft von 2 auf 1 \\
		21.05.2018 & Lucas \& Dorus & Erweiterung um Risiken 14-15 \\
		\hline
	\end{tabular}
\end{table}


}

\IfFileExists{99-Arbeitsjournale}{
	\newpage
	\subsection{Arbeitsjournale}

\subsubsection{Arbeitsjournal von Dorus Janssens}

% TODO:

\subsubsection{Arbeitsjournal von Lucas Schnüriger}

% TODO:

}

\IfFileExists{99-Architektur}{
	\newpage
	\input{99-Architektur}
}




}

\end{document}
