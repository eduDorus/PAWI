%----------------------------------------------------------------------------------------
%	PACKAGES AND OTHER DOCUMENT CONFIGURATIONS
%----------------------------------------------------------------------------------------

\documentclass[11pt]{article} % Default font size is 12pt, it can be changed here
\pagestyle{headings}

\usepackage{geometry} % Required to change the page size to A4
\geometry{a4paper, total={6.4in, 9in}} % Set the page size to be A4 as opposed to the default US Letter
\usepackage{graphicx} % Required for including pictures
\usepackage{subcaption} % to use subfigure
\usepackage{wrapfig} % Allows in-line images such as the example fish picture
\usepackage[natbibapa]{apacite}  % APA Standard citing
\usepackage[ngerman]{babel}
\usepackage[utf8x]{inputenc} % utf8x instead of utf8 becuase "breaks" are not recognized in utf8
\usepackage{parskip} % For auto-paragraphs
\usepackage{color}
\usepackage{amsmath}
\usepackage{lscape} % For landscape content
\usepackage[hyphens]{url} % for linebreaking url's in sources MUSS VOR HYPERREF PACKAGE KOMMEN
\usepackage{hyperref} % For clickable table of contents
\usepackage[bottom]{footmisc} % For footnotes at them bottom
\usepackage{enumitem} % no separation between list items
\usepackage{soulutf8}
\usepackage{lastpage}
\usepackage{fancyhdr}
\usepackage{listings}
\usepackage{multicol}
\setlength{\columnsep}{0.6cm}

\pagestyle{fancy}
\renewcommand{\headrulewidth}{0.3pt}
\renewcommand{\footrulewidth}{0.3pt}

\usepackage{titlesec}
\newcommand{\sectionbreak}{\clearpage} % have sections start a new page

\hypersetup{
    colorlinks,
    citecolor=black,
    filecolor=black,
    linkcolor=black,
    urlcolor=black
}

\oddsidemargin = 0pt

\makeatletter % Quellen in Abbildungsverzeichnis
\newcommand{\figsourcefont}{\footnotesize Quelle: }
\newcommand{\figsource}[1]{%
  \addtocontents{lof}{%
    {\leftskip\cftfigindent
     \advance\leftskip\cftfignumwidth
     \rightskip\@tocrmarg
     \figsourcefont#1\protect\par}%
  }%
 }
\makeatother

% alle Figures automatisch \centering
\makeatletter
\g@addto@macro\@floatboxreset\centering
\makeatother

% alle Figures und Tabellen automatisch an Position [htb]
\makeatletter
\renewcommand*{\fps@figure}{htb}
\renewcommand*{\fps@table}{htb}
\makeatother

% command für einfacheres Einfügen von einzelnen Bildern
\newcommand{\bild}[4]{%
  \begin{figure}
    \centering
    \includegraphics[width=#1\textwidth]{#2}%
    \caption{#3}%
    \label{#4}%
  \end{figure}
}%

\addto{\captionsngerman}{	% Renew some Captions
% \renewcommand*{\contentsname}{Inhalt}
  \renewcommand*{\listfigurename}{Abbildungen}
  \renewcommand*{\listtablename}{Tabellen}
  \renewcommand*{\figurename}{Abb.}
  \renewcommand*{\tablename}{Tab.}
}

\usepackage[export]{adjustbox}
\usepackage{caption}
%\usepackage{fixltx2e} % for upper / subscript

\usepackage[scaled]{beramono}
\usepackage[T1]{fontenc}

\usepackage[breakwords]{truncate}
\usepackage{tcolorbox}
\newtcolorbox[auto counter, number within=section, list inside=lernziel.toc]{lernziel}[2][]{%
  colback=black!5!white, colframe=black!50!white,
  fonttitle=\bfseries, fontupper=\bfseries #2 \tcblower,
  boxrule=0.3mm,
  boxsep=2mm, left=1mm, right=1mm,
  title=Lernziel~\thetcbcounter, list text=#2, #1
}

\newenvironment{italicquotes} % Long quotes italic
{\begin{quote}\itshape}
{\end{quote}}

\linespread{1.2} % Line spacing
\setlength\parindent{0pt} % Uncomment to remove all indentation from paragraphs "Umfang ca..."
\setcounter{tocdepth}{2}

\graphicspath{{images/}} % Specifies the directory where pictures are stored

\begin{document}

%----------------------------------------------------------------------------------------
%	TITLE PAGE
%----------------------------------------------------------------------------------------
\begin{titlepage}

\newcommand{\HRule}{\rule{\linewidth}{0.5mm}} % Defines a new command for the horizontal lines, change thickness here

\center

\textsc{\LARGE Hochschule Luzern\\Informatik}\\[1.5cm] % Name of your university/college
\textsc{\Large Informatikprojekt PAWI, FS 2018}\\[1.5cm] % Major heading such as course name

\HRule \\[1cm]
{\huge \bfseries iOS App mit Machine Learning und Augmented Reality}\\[0.6cm] % Title of your document
\HRule \\[1cm]

\vspace{30pt}

\begin{minipage}{0.4\textwidth}
  \begin{flushleft} \large
    \emph{Autoren:}\\
    Dorus \textsc{Janssens}\\
    Lucas \textsc{Schnüriger}\\
    ~\\~ % for vertical alignment
  \end{flushleft}
\end{minipage}
\begin{minipage}{0.4\textwidth}
  \begin{flushleft} \large
    \emph{Dozent:}\\
    Prof. Dr. Ruedi \textsc{Arnold}\\~\\
    \emph{Projektpartner:}\\
    Prof. Dr. Thomas \textsc{Koller}
  \end{flushleft}
\end{minipage}\\[1cm]


\vspace{30pt}

{\large \today }\\[1cm] % Date, change the \today to a set date if you want to be precise

\end{titlepage}


%----------------------------------------------------------------------------------------
%	Vor-Zeugs
%----------------------------------------------------------------------------------------
\pagenumbering{Roman}
\pagestyle{plain}

\IfFileExists{01-Selbststaendigkeitserklaerung}{
  \section*{Selbstständigkeitserklärung}
Hiermit erkläre ich, dass ich die vorliegende Arbeit selbstständig angefertigt und keine anderen als die angegebenen Hilfsmittel verwendet habe. Sämtliche verwendeten Textausschnitte, Zitate oder Inhalte anderer Verfasser wurden ausdrücklich als solche gekennzeichnet.

\vspace{1cm}

Rotkreuz, 8. Juni 2018 

\vspace{2cm}

\parbox{\textwidth}{
  \parbox{7cm}{
    \rule{6cm}{.5pt}\\
    Dorus Janssens
  }
  \hfill
  \parbox{7cm}{
    \rule{6cm}{.5pt}\\
    Lucas Schnüriger
  }
}

}

\newpage
\IfFileExists{02-Abstract}{
  \section*{Abstract}


}

\newpage
\IfFileExists{03-Begriffe}{
  \section*{Begriffe \& Abkürzungen}
\begin{table}
	\begin{tabular}{@{} p{.12\textwidth} p{.85\textwidth} @{}}
		\hline
		\textbf{Begriff} & \textbf{Erklärung} \\
		\hline
		ABIZ	& Forschungsgruppe Algorithmic Business der Hochschule Luzern - Informatik \\
		AR 		& Augmented Reality, computergestützte – primär visuelle – Erweiterung der Realität \\
		cuboro	& Schweizer Unternehmen, das Kugelbahnen aus Holz herstellt \\
		iOS		& Apples mobiles Betriebssystem für iPhones und iPads \\
		ML		& Machine Learning, maschinelles Lernen und Gewinnung von Wissen aus Erfahrungen und Lerndaten durch den Computer \\
		HSLU	& Hochschule Luzern \\
		VR		& Virtual Reality, eine interaktive virtuelle Umgebung, die in Echtzeit generiert wird \\
		WWDC	& Worldwide Developers Conference, jährliche Konferenz von Apple für Software-Entwickler, bei der der Konzern oft neue Produkte vorstellt \\
		Xcode	& Entwicklungsumgebung von Apple zur Entwicklung von Programmen für Apples Betriebssysteme (macOS, iOS, tvOS und watchOS) \\
		\hline
	\end{tabular}
\end{table}

}

%----------------------------------------------------------------------------------------
%	TOC
%----------------------------------------------------------------------------------------
\newpage
\tableofcontents % Include a table of contents

%----------------------------------------------------------------------------------------
%	Content
%----------------------------------------------------------------------------------------
\newpage % Begins the essay on a new page instead of on the same page as the table of contents
\pagenumbering{arabic}

\pagestyle{fancy}
\lhead{I.BA\_PAWI.F18}
\chead{Hochschule Luzern – Informatik}
\rhead{\nouppercase{\leftmark}}
\lfoot{\today}
\cfoot{}
\rfoot{\thepage\ von \pageref{LastPage}}

\newpage
\IfFileExists{04-Einleitung}{
  \section{Einleitung}

In dieser Arbeit sollen die technischen Grenzen von Augmented Reality mit iOS Frameworks aufgezeigt werden. Die Projektmethodik findet im explorativem Verfahren stattfinden. 
Die Arbeit setzt sich zuerst mit der Problemstellung, Ziele und Abgrenzung auseinander. Anschliessend werden die für diese Arbeit relevanten Themen beschrieben und den Stand der Technik in deren Rubrik aufgeführt. Im nächsten Kapitel werden die iOS Frameworks CoreML, Vision, ARKit etc. veranschaulicht. Nach der theoretischen Einleitung werden im Kapitel Lösungsentwicklung die Prototypen vorgestellt, die in der explorativen Phase erstellt wurden. Nach einer Lösungswahl wird im Kapitel Umsetzung ist die erarbeitete Demo-App detailliert dokumentiert. Im anschliessenden Kapitel Validierung wird die erarbeitete Lösung mit dem Anforderungskatalog verglichen und ein technisches Fazit gezogen. Im letzten Kapitel Schlussfolgerung werden die Lessons Learned, persönliche Fazit und ein Ausblick auf weitere Arbeiten gegeben.
% Forschungskonzept, Fragestellung, Method, Aufbau der Arbeit

}

\section{Problemstellung}
\subsection{Ausgangslage}
\subsection{Ziele}
\subsection{Abgrenzung (Scope)}
\subsection{Anforderungen}
\subsection{Lösungsskizze}

\section{Stand der Technik}

\section{Lösungsentwicklung}

\section{Umsetzung}

\section{Validierung}

\section{Schlussfolgerungen}
\subsection{Lessons Learned}
\subsection{Ausblick}


%----------------------------------------------------------------------------------------
%	Bibliography
%----------------------------------------------------------------------------------------
\newpage
\section{Quellen und Verzeichnisse}
\bibliographystyle{apacite}
\bibliography{Quellen}
%------------------------------------------------
\newpage
%\section{Abbildungsverzeichnis}
\listoffigures
\addtocontents{lof}{\protect\thispagestyle{headings}}
%------------------------------------------------
\newpage
%\section{Tabellenverzeichnis}
\listoftables
\addtocontents{lot}{\protect\thispagestyle{headings}}

%----------------------------------------------------------------------------------------
%	Appendix
%----------------------------------------------------------------------------------------
\newpage
\appendix
\setcounter{figure}{0}
\setcounter{secnumdepth}{0}
\newgeometry{
 total={170mm,257mm},
 left=25mm,
 top=25mm,}
\IfFileExists{20_Appendix}{
\input{20_Appendix}
}

\end{document}
