%----------------------------------------------------------------------------------------
%	PACKAGES AND OTHER DOCUMENT CONFIGURATIONS
%----------------------------------------------------------------------------------------

\documentclass[11pt]{article} % Default font size is 12pt, it can be changed here
\pagestyle{headings}

\usepackage{geometry} % Required to change the page size to A4
\geometry{a4paper, total={6.4in, 9in}} % Set the page size to be A4 as opposed to the default US Letter
\usepackage{graphicx} % Required for including pictures
\usepackage{subcaption} % to use subfigure
\usepackage{wrapfig} % Allows in-line images such as the example fish picture
\usepackage[natbibapa]{apacite}  % APA Standard citing
\usepackage[ngerman]{babel}
\usepackage[utf8x]{inputenc} % utf8x instead of utf8 becuase "breaks" are not recognized in utf8
\usepackage{parskip} % For auto-paragraphs
\usepackage{color}
\usepackage{amsmath}
\usepackage{lscape} % For landscape content
\usepackage[hyphens]{url} % for linebreaking url's in sources MUSS VOR HYPERREF PACKAGE KOMMEN
\usepackage{hyperref} % For clickable table of contents
\usepackage[bottom]{footmisc} % For footnotes at them bottom
\usepackage{enumitem} % no separation between list items
\usepackage{soulutf8}
\usepackage{lastpage}
\usepackage{fancyhdr}
\usepackage{listings}
\usepackage{multicol}
\usepackage{multirow}
\usepackage{longtable}
\setlength{\columnsep}{0.6cm}

\pagestyle{fancy}
\renewcommand{\headrulewidth}{0.3pt}
\renewcommand{\footrulewidth}{0.3pt}

\usepackage{titlesec}
\newcommand{\sectionbreak}{\clearpage} % have sections start a new page
\newcommand{\tabitem}{~~\llap{\textbullet}~~}

\hypersetup{
    colorlinks,
    citecolor=black,
    filecolor=black,
    linkcolor=black,
    urlcolor=black
}

\oddsidemargin = 0pt

\makeatletter % Quellen in Abbildungsverzeichnis
\newcommand{\figsourcefont}{\footnotesize Quelle: }
\newcommand{\figsource}[1]{%
  \addtocontents{lof}{%
    {\leftskip\cftfigindent
     \advance\leftskip\cftfignumwidth
     \rightskip\@tocrmarg
     \figsourcefont#1\protect\par}%
  }%
 }
\makeatother

% alle Figures automatisch \centering
\makeatletter
\g@addto@macro\@floatboxreset\centering
\makeatother

% alle Figures und Tabellen automatisch an Position [htb]
\makeatletter
\renewcommand*{\fps@figure}{htb}
\renewcommand*{\fps@table}{htb}
\makeatother

% command für einfacheres Einfügen von einzelnen Bildern
\newcommand{\bild}[3]{%
  \begin{figure}
    \centering
    \includegraphics[width=#1\textwidth]{#2}%
    \caption{#3}%
    \label{fig:{#2}}%
  \end{figure}
}%

\addto{\captionsngerman}{	% Renew some Captions
% \renewcommand*{\contentsname}{Inhalt}
  \renewcommand*{\listfigurename}{Abbildungen}
  \renewcommand*{\listtablename}{Tabellen}
  \renewcommand*{\figurename}{Abb.}
  \renewcommand*{\tablename}{Tab.}
}

\usepackage[export]{adjustbox}
\usepackage{caption}
%\usepackage{fixltx2e} % for upper / subscript

\usepackage[scaled]{beramono}
\usepackage[T1]{fontenc}

\usepackage[breakwords]{truncate}
\usepackage{tcolorbox}
\newtcolorbox[auto counter, number within=section, list inside=lernziel.toc]{lernziel}[2][]{%
  colback=black!5!white, colframe=black!50!white,
  fonttitle=\bfseries, fontupper=\bfseries #2 \tcblower,
  boxrule=0.3mm,
  boxsep=2mm, left=1mm, right=1mm,
  title=Lernziel~\thetcbcounter, list text=#2, #1
}

\newenvironment{italicquotes} % Long quotes italic
{\begin{quote}\itshape}
{\end{quote}}

\linespread{1.2} % Line spacing
\setlength\parindent{0pt} % Uncomment to remove all indentation from paragraphs "Umfang ca..."
\setcounter{tocdepth}{2}

\graphicspath{{images/}} % Specifies the directory where pictures are stored

\begin{document}

%----------------------------------------------------------------------------------------
%	TITLE PAGE
%----------------------------------------------------------------------------------------
\begin{titlepage}

\newcommand{\HRule}{\rule{\linewidth}{0.5mm}} % Defines a new command for the horizontal lines, change thickness here

\center

\textsc{\LARGE Hochschule Luzern\\Informatik}\\[1.5cm] % Name of your university/college
\textsc{\Large Informatikprojekt PAWI, FS 2018}\\[1.5cm] % Major heading such as course name

\HRule \\[1cm]
{\huge \bfseries iOS App mit Machine Learning und Augmented Reality}\\[0.6cm] % Title of your document
\HRule \\[1cm]

\vspace{30pt}

\begin{minipage}{0.4\textwidth}
  \begin{flushleft} \large
    \emph{Autoren:}\\
    Dorus \textsc{Janssens}\\
    Lucas \textsc{Schnüriger}\\
    ~\\~ % for vertical alignment
  \end{flushleft}
\end{minipage}
\begin{minipage}{0.4\textwidth}
  \begin{flushleft} \large
    \emph{Dozent:}\\
    Prof. Dr. Ruedi \textsc{Arnold}\\~\\
    \emph{Projektpartner:}\\
    Prof. Dr. Thomas \textsc{Koller}
  \end{flushleft}
\end{minipage}\\[1cm]


\vspace{30pt}

{\large \today }\\[1cm] % Date, change the \today to a set date if you want to be precise

\end{titlepage}


%----------------------------------------------------------------------------------------
%	Vor-Zeugs
%----------------------------------------------------------------------------------------
\pagenumbering{Roman}
\pagestyle{plain}

\IfFileExists{01-Selbststaendigkeitserklaerung}{
  \section*{Selbstständigkeitserklärung}
Hiermit erkläre ich, dass ich die vorliegende Arbeit selbstständig angefertigt und keine anderen als die angegebenen Hilfsmittel verwendet habe. Sämtliche verwendeten Textausschnitte, Zitate oder Inhalte anderer Verfasser wurden ausdrücklich als solche gekennzeichnet.

\vspace{1cm}

Rotkreuz, 8. Juni 2018 

\vspace{2cm}

\parbox{\textwidth}{
  \parbox{7cm}{
    \rule{6cm}{.5pt}\\
    Dorus Janssens
  }
  \hfill
  \parbox{7cm}{
    \rule{6cm}{.5pt}\\
    Lucas Schnüriger
  }
}

}

\newpage
\IfFileExists{02-Abstract}{
  \section*{Abstract}


}

\newpage
\IfFileExists{03-Begriffe}{
  \section*{Begriffe \& Abkürzungen}
\begin{table}
	\begin{tabular}{@{} p{.12\textwidth} p{.85\textwidth} @{}}
		\hline
		\textbf{Begriff} & \textbf{Erklärung} \\
		\hline
		ABIZ	& Forschungsgruppe Algorithmic Business der Hochschule Luzern - Informatik \\
		AR 		& Augmented Reality, computergestützte – primär visuelle – Erweiterung der Realität \\
		cuboro	& Schweizer Unternehmen, das Kugelbahnen aus Holz herstellt \\
		iOS		& Apples mobiles Betriebssystem für iPhones und iPads \\
		ML		& Machine Learning, maschinelles Lernen und Gewinnung von Wissen aus Erfahrungen und Lerndaten durch den Computer \\
		HSLU	& Hochschule Luzern \\
		VR		& Virtual Reality, eine interaktive virtuelle Umgebung, die in Echtzeit generiert wird \\
		WWDC	& Worldwide Developers Conference, jährliche Konferenz von Apple für Software-Entwickler, bei der der Konzern oft neue Produkte vorstellt \\
		Xcode	& Entwicklungsumgebung von Apple zur Entwicklung von Programmen für Apples Betriebssysteme (macOS, iOS, tvOS und watchOS) \\
		\hline
	\end{tabular}
\end{table}

}

%----------------------------------------------------------------------------------------
%	TOC
%----------------------------------------------------------------------------------------
\newpage
\tableofcontents % Include a table of contents

%----------------------------------------------------------------------------------------
%	Content
%----------------------------------------------------------------------------------------
\newpage % Begins the essay on a new page instead of on the same page as the table of contents
\pagenumbering{arabic}

\pagestyle{fancy}
\lhead{I.BA\_PAWI.F18}
\chead{Hochschule Luzern – Informatik}
\rhead{\nouppercase{\leftmark}}
\lfoot{\today}
\cfoot{}
\rfoot{\thepage\ von \pageref{LastPage}}

\newpage
\IfFileExists{04-Einleitung}{
  \section{Einleitung}

In dieser Arbeit sollen die technischen Grenzen von Augmented Reality mit iOS Frameworks aufgezeigt werden. Die Projektmethodik findet im explorativem Verfahren stattfinden. 
Die Arbeit setzt sich zuerst mit der Problemstellung, Ziele und Abgrenzung auseinander. Anschliessend werden die für diese Arbeit relevanten Themen beschrieben und den Stand der Technik in deren Rubrik aufgeführt. Im nächsten Kapitel werden die iOS Frameworks CoreML, Vision, ARKit etc. veranschaulicht. Nach der theoretischen Einleitung werden im Kapitel Lösungsentwicklung die Prototypen vorgestellt, die in der explorativen Phase erstellt wurden. Nach einer Lösungswahl wird im Kapitel Umsetzung ist die erarbeitete Demo-App detailliert dokumentiert. Im anschliessenden Kapitel Validierung wird die erarbeitete Lösung mit dem Anforderungskatalog verglichen und ein technisches Fazit gezogen. Im letzten Kapitel Schlussfolgerung werden die Lessons Learned, persönliche Fazit und ein Ausblick auf weitere Arbeiten gegeben.
% Forschungskonzept, Fragestellung, Method, Aufbau der Arbeit

}

\IfFileExists{05-Problemstellung}{
  \section{Problemstellung}

\subsection{Ausgangslage}
% Welche Problematik führt zu diesem Projekt? Was ist am derzeitigen Zustand unbefriedigend?
Für die PAWI Arbeit im Frühlingssemester 2018 soll eine iOS-App erstellt werden, welche es erlaubt mit einer physischen cuboro Kugelbahn zu interagieren. Einerseits soll es die zu erstellende App erlauben, physische Kugelbahnen digital zu erfassen. Diese Erfassung kann beispielsweise parallel zum Aufbau einer Kugelbahn geschehen, indem die App Bauteil um Bauteil diese zuerst erkennt (mit passenden Methoden aus dem Bereich Bilderkennung/Computer Vision bzw. dem maschinellen Lernen) und dann dessen relative Position innerhalb von der Kugelbahn aufzeichnet. Es sollen zwingend mehrere unterschiedliche Kugelbahnen in der App erfasst sein, ggf. werden diese in der Software fix hinterlegt bzw. können auf einem anderen Weg als mithilfe der App erfasst werden, z.B. durch einen Import aus dem Cuboro-Webkit.

Erfasste Kugelbahnen sollen nach Möglichkeit in AR-Manier als Echtzeit-Überblendung auf Kamerabildern der physischen Kugelbahn projiziert werden können, z.B. indem Bauteile eingefärbt werden, in Form von angezeigten "`Bounding-Boxes"' oder indem virtuelle Kugeln auf der physischen Kugelbahn simuliert werden können. Oder weiter könnten erfasste Kugelbahnen mittels AR beispielsweise auf einen physischen Tisch "`gestellt"' werden und darauf könnte das Rollen von virtuellen Kugeln simuliert werden. 

In dieser Arbeit gibt es einige Freiheitsgrade, sprich viele funktionale und technische Aspekte gilt es zu erarbeiten, evaluieren und fest zu legen. Damit das Endresultat den Erwartungen von Auftraggeber und Betreuer entspricht, sollen Evaluationen und Entscheidungsfindungen in enger Rücksprache mit dem Auftraggeber und dem Betreuer stattfinden. Wichtige Entscheide (wie z.B. Fixierung von funktionalen Anforderungen, Wahl von Frameworks, usw.) müssen vom Auftraggeber bzw. dem Betreuer genehmigt werden, entsprechend sollen diese möglichst proaktiv in Evaluationen, Bau von Prototypen usw. miteinbezogen, sowie über Vor- und Nachteile, Alternativen usw. informiert werden.

\subsection{Ziele}
% Was soll mit dem Projekt erreicht werden? Was soll nach dem Projekt „besser“ sein als vorher?
Das Hauptziel ist es, dass am Schluss eine attraktive und lauffähige Demo-App zur Verfügung steht. Die App soll möglichst intuitiv angewendet werden können und die Möglichkeiten und Grenzen dieser neuen Technologien illustrieren. Weitere Ziele sind das erfolgreiche Einarbeiten in die neue Programmiersprache Swift und den iOS Frameworks Core ML, Vision und ARKit. Die Frameworks sollen anhand von Problemlösungszyklen genauer auf einzelne Aspekte geprüft werden.

\subsection{Abgrenzung (Scope)}
% Welche Sachverhalte sind nicht Bestandteil des Projektes. Welche Restriktionen gibt es im Projekt?
Nichtbestandteil des Projektes sind Betriebssysteme / Frameworks ausserhalb von iOS. Die iOS-App wird ausschliesslich mit Swift entwickelt. Es werden ausschliesslich native iOS UI Komponente für die Applikation verwendet.

\subsection{Fragestellungen}
\begin{itemize}
	\item Das SceneKit eignet sich für 3D Objekte und das SpriteKit für 2D Flächen. Welche der beiden Frameworks kann ein virtuelles Overlay auf einem physischen Würfel erzeugen?
	\item Wie können physische Körper als virtuelle Objekte modelliert werden?
	\item Welche Möglichkeiten bestehen mit augmentierten Objekten zu interagieren und diese zu mutieren?
	\item Wie können virtuelle Objekte ausgerichtet werden?
	\item Wie werden virtuelle Körper definiert, beschrieben, persistiert?
	\item Wie stabil verhält sich der AR Referenzpunkt?
	\item Unter welchen Voraussetzungen kann der Referenzpunkt zuverlässig aufgebaut werden?
\end{itemize}

\subsection{Anforderungen}
Im diesem Kapitel sind alle funktionalen und nichtfunktionalen Anforderungen des Projekts festgehalten. Jede Anforderung ist in eine der folgenden drei Kategorien priorisiert:
\begin{itemize}
	\item Muss (M): Dies sind Festanforderungen, die zwingend vom Projekt erfüllt werden müssen.
	\item Soll (S): Dies sind Wunschanforderungen, die umgesetzt werden sollen, sobald die Muss-Anforderungen erfüllt sind oder erfüllt werden können.
	\item Kann (K): Diese Zusatzanforderungen haben die niedrigste Priorität und können umgesetzt werden, sofern keine anderen (wichtigeren) Anforderungen beeinträchtigt werden und der erfolgreiche Projektablauf nicht gefährdet wird.
\end{itemize}

Die App soll folgende \textbf{Epics} erfüllen:
\begin{enumerate}
	\item \textbf{Bauanleitung:} Der Benutzer kann von einer gewählten Kugelbahn eine Augmented Reality Bauanleitung erhalten, um die Bahn physisch aufbauen zu können.
	\item \textbf{Editor:} Der Benutzer kann in Augmented Reality eine virtuelle Kugelbahn neu erstellen oder eine bestehende verändern, damit diese als Bauanleitung zur Verfügung stehen.
\end{enumerate}

\subsubsection{Funktionale Anforderungen}
In Tabelle \ref{tab:funktionale-anforderungen} sind die funktionalen Anforderungen an die zu entwickelnde App aufgelistet. Diese Anforderungen beschreiben die gewünschten Funktionen und Leistungen der fertigen Anwendung.

\begin{longtable}{l l p{4.7cm} p{8cm}}
	\hline
	\textbf{Nr.} & \textbf{Prio.} & \textbf{Bezeichnung} & \textbf{Erläuterungen} \\
	\hline
	01 & M & Virtuelle Darstellung einer Kugelbahn & Die App kann eine Kugelbahn virtuell darstellen (VR). \\
	02 & M & AR Bahn erstellen & Der Benutzer kann eine virtuelle Kugelbahn in AR erstellen. \\
	03 & M & AR Bahn verändern & Der Benutzer kann eine virtuelle Kugelbahn in AR verändern. \\
	04 & M & AR Bauanleitung & Der Benutzer kann eine physische Kugelbahn mit Hilfe einer AR Bauanleitung Element für Element aufbauen. \\
	05 & M & Mehrere Kugelbahnen erfassen & Die App kann mehrere unterschiedliche Kugelbahnen erfasst, bzw. hinterlegt, haben. \\
	06 & M & Kugelbahn virtuell platzieren & Erfasste virtuelle Kugelbahnen können mittels AR auf eine physische Fläche platziert und betrachtet werden. \\
	07 & S & Virtuelle Überlagerungen einer Kugelbahn & Ein physisches Modell einer Kugelbahn kann in Echtzeit mit virtuellen Überlagerungen auf der App angereichert werden (AR). \\
	08 & K & Virtuelle Kugeln simulieren & Es können virtuelle Kugeln auf einer physischen Kugelbahn simuliert werden. \\
	\hline
	\caption{Funktionale Anforderungen}
	\label{tab:funktionale-anforderungen}
\end{longtable}

\subsubsection{Nichtfunktionale Anforderungen}
In Tabelle \ref{tab:nichtfunktionale-anforderungen} sind die nichtfunktionalen Anforderungen an das Projekt aufgelistet. Nichtfunktionale Anforderungen betreffen einerseits die Qualität (während und ausserhalb der Laufzeit) sowie die Einschränkungen (aus der Technologie und aus Randbedingungen des Moduls PAWI) des Projekts. Dies betrifft sowohl die Applikation als solches, als auch die Dokumentation.

\begin{longtable}{l l p{4.7cm} p{8cm}}
	\hline
	\textbf{Nr.} & \textbf{Prio.} & \textbf{Bezeichnung} & \textbf{Erläuterungen} \\
	\hline
	01 & M & iOS und Swift & Die App ist in Swift für iOS programmiert. \\
	02 & M & Lauffähige App & Die App läuft auf einem iOS Gerät. \\
	03 & M & Recherche von iOS Frameworks & Die Dokumentation enthält Informationen zu den iOS Frameworks ARKit, Core ML und Vision. \\
	04 & M & Technologie aufzeigen & Die App zeigt die Möglichkeiten von AR mittels ARKit auf. \\
	05 & M & Nützlicher Code & Der Sourcecode soll für an der verwendeten Technik interessiert aufschlussreich und hilfreich sein. \\
	\hline
	\caption{Nichtfunktionale Anforderungen}
	\label{tab:nichtfunktionale-anforderungen}
\end{longtable}

\subsection{Lösungsskizze}

}

\IfFileExists{06-Stand-der-Technik}{
  \section{Stand der Technik}

\subsection{Augmented Reality} \label{sub:augmented-reality}
Um zu Verstehen wobei es sich bei Augemented Reality handelt müssen wir den Ursprung des Wortes analysieren. Da augmented soviel wie hinzufügen oder erweitern bedeutet und reality die physische Wahrnehmung der Umgebung wiedergibt kann das Word auf deutsch als erweiterte Realität bezeichnet werden.

Augmented Reality kann wie folgt beschrieben werden: Eine erweiterte Version der Realität, in der direkte oder indirekte die Ansichten physischer Realitätsumgebungen oder Objekte mit überlagerten computergenerierten Bildern über die Sicht eines Benutzers auf die reale Welt projeziert werden, wodurch die aktuelle Wahrnehmung der Realität erweitert wird. (\cite{reality-technologies})

Die AR Technologie hat viele Anwendungszwecke und es wird in der Zukunft eine grosse Rolle spielen. Anbei einige Beispiele damit der Verwendungszweck der Technologie ein klares Bild erhält:
\begin{itemize}
    \item Unterhaltung: Spiele wie Tetris können auf jeder beliebigen Unterlage gespielt werden und von diversen Blichwinkel erlebt werden.
    \item Gesundheitswesen: Dem Arzt werden vitalwerte des Patienten eingeblendet und informationen zur aktuellen operation sowie mögliche Schnittmuster etc. angezeigt.
    \item Wartung: Bei der Wartung von speziellen Turbinensystemen wird der Wartungsarbeiter auf die Herangehensweise aufmerksamgemacht und welche Teile in welcher Reinenfolge abmontiert, gewartet und wiedermontiert werden müssen.
    \item Navigation: Der Navigationsweg wird auf die Strasse projeziert, wobei der Lenker sein Blick nicht von der Strasse wenden muss.
    \item Bildung: Der Dozent kann Objekte aus dem Unterrichtsstoff anzeigen und in seine Einzelteile erklären ohne dabei ein physisches Examplar zu besitzen. Dies kann z.B ein menschlicher Körper bei Medizinstudenten sein, damit jedes Organ betrachet werden kann.
    \item Einkaufen: Möbel können in der Wohnung virtuell platziert werden. Dies Unterstützt den Käufer beim kauf der Wohnungseinrichtung
\end{itemize}

Augmented Reality bietet, wie oben beschrieben, unzählige anwendungsgebiete. Mit dem erscheinen der Frameworks ARKit für iOS und ARCore für Android Smartphones wird die Technologie für den Massenmarkt zugänglich.

\subsection{Virtual Reality} \label{sub:virtual-reality}
Eine Unterscheidung muss zu Virtual Reality (VR) gemacht werden. Bei VR wird die gesamte realität mit computergenerierten Bildern wiedergegeben. Dies ist eine hervorragende Technologie wenn es darum geht virtuelle Welten, Filme oder neue Gebäude zu erleben. Eine wesentliche Herausforderung bei VR stellt das Manövrieren dar. Es ist schwierig in einer komplett virtuellen Welt zu agieren, ohne dass in der realen Welt ein Hindernis im weg steht.

\subsection{Computer Vision} \label{sub:computer-vision}
Bei Computer Vision handelt es sich Bild- und Videoerkennung. Aus aufgenommen Bildern sollen Informationen extrahiert werden und zur interpretierung genutzt werden. Diese Interpretierung kann sich in meherern Aufgaben unterscheiden wie z.B das Erkennen von Objekten, das Beschreiben eines Bildes, die Möglichen nächsten Bilder eines Films voraus sagen etc. Computer Vision wird heutzutage in vielen Bereichen genutzt und hat mit Deep Learning neue Massstäbe angenommen. In gewissen Subsets von Bildern können Computer diese Bereits besser Kategorisieren denn Menschen.

\subsection{Swift}
Swift ist die neue Programmiersprache für macOS, iOS, watchOS und tvOS. Die Programmiersprache wurde so aufgebaut, dass sie möglichst intuitive für Entwichler ist aber trotzdem ein mächtiges arsenal an Features besitzt. Swift wird mittels dem LLVM Compiler zu optimiertem nativem Code transformiert. Swift ist ein Open Source Projekt unter Swift.org und kann vom jedem mitgestaltet werden. Für unser Projekt wird die vierte Version von Swift verwendet.

% TODO: kleine Einführung zu Datentypen und Collections

\subsection{Xcode}
XCode ist die von Apple zur Verfügung gestellte IDE um iOS oder macOS Software herzustellen. Es bietet durchdachte Features wie Code vervollständigung, Refactoring hinweise und Storyboards.

Mit dem Storyboad kann das UI einer App schnell und sauber erstellt werden. Es werden UI Komponenten für fast jeden erdenklichen Fall angeboten. Diese Komponenten können bequem über drag und drop auf dem Interface platziert werden. Anschliessen besteht die Möglichkeit die Komponente genauer zu konfigurieren und mit dem View Controller zu verbinden.

}

\IfFileExists{07-iOS-Frameworks}{
  \section{iOS Frameworks}
In den folgenden Abschnitten wird ein Überblick über die wesentlichen iOS Frameworks gegeben, die für dieses Projekt relevant sind. Alle Informationen stammen, sofern nicht anders angegeben, aus den offiziellen Dokumentationen von Apple. % TODO: Quellen hier einsetzen


\subsection{Core ML} \label{subsub:core-ml}
Mit dem Aufschwung von neuronalen Netzen insbesondere Deep Learning wollen Entwickler diese Technologie in ihren Apps nutzen. Mit dem CoreML Framework gibt Apple den Entwicklern diese Möglichkeit und erlaubt es selbst trainierte Netze in den Apps zu integrieren. CoreML \cite{core-ml} ist dabei für On-Device Performance optimiert und verspricht weniger Arbeitsspeicher zu belegen und dabei den Energie konsum zu minimieren. CoreML unterstützt die Frameworks Vision zur Bildverarbeitung, Foundation zum verarbeiten von Text und GameplayKit zum erstellen von Entscheidungsbäume.

\bild[https://developer.apple.com/documentation/coreml]{0.4}{core-ml-architecture}{CoreML Architektur}

\subsection{Vision}
Apple bietet mit dem Vision \cite{vision} Framework eine gute Lösung für Computer Vision Probleme. Es enthält Gesichts-, Text-, Barcodeerkennung und Feature Tracking, wobei bereits ein grossteil der Probleme damit gelöst werden kann. Zusätzlich kann das in der Section \ref{subsub:core-ml} beschriebene CoreML verwendet werden, um spezifische Probleme mittels neuronalen Netzen zu lösen. Die integration mit dem CoreML Framework soll ohne Probleme funktionieren da es, wie im Bild \ref{fig:core-ml-architecture}, eine API Schicht über dem CoreML Framework befindet.

\subsection{ARKit}

\subsubsection{Überblick}
Zusammen mit iOS 11 präsentierte Apple im Juni 2017 ARKit als neues Framework für Augmented Reality Anwendungen, das alle iOS Geräten mit mindestens einem Apple A9 Prozessor unterstützt. Für das World Tracking nutzt ARKit die "`visual-inertial odometry"' Technik. Dabei werden Informationen aus den Bewegungssensoren mit denen aus den Kamerabildern kombiniert. Das Framework errechnet damit ein Modell der realen Welt und die Position und Ausrichtung des Geräts. Um in diesem Modell virtuelle Objekte zu platzieren bietet ARKit einerseits \texttt{ARHitTestResult} um Punkte auf dem Kamerabild einer Oberfläche/Stelle der realen Welt zuordnen, und andererseits \texttt{planeDetection}, das ebene Flächen sucht. Ab ARKit Version 1.5 werden neben horizontalen auch vertikale Flächen erkannt und die Geometrie der Flächen wird statt nur rechteckig neu auch polygonal angegeben. Die eigentliche Beschreibung und Konstruktion virtueller Objekte wird von den Frameworks SpriteKit, SceneKit oder Metal übernommen.

\subsubsection{Sessions}
Die Klasse \texttt{ARSession} koordiniert die wichtigsten Bestandteile der ARKit Funktionalität, darunter die Kamera und die Bewegungssensoren aber auch die Bildverarbeitung. Um das ARKit zu nutzen wird mindestens eine \texttt{ARSession} benötigt, die beiden ViewController \texttt{ARSCNView} und \texttt{ARSKView} beinhalten gleich eine Session-Instanz.

Die Session benötigt eine Konfiguration mit \texttt{ARConfiguration} oder einer ihrer Subklassen. Damit können verschiedene Eigenschaften festgelegt werden: wird nur das Drehen (die Ausrichtung) nicht aber die Position des Geräts berücksichtigt? Werden die Koordinaten nach der ursprünglichen oder aktuellen Lage des Gerät oder gar nach dem Kompass ausgerichtet? Wird die Frontkamera und somit Gesichtserkennung/-tracking verwendet?

Bei der standardmässigen Verwendung von \texttt{ARWorldTrackingConfiguration} als Session Konfiguration wird die Position und Ausrichtung des Geräts beim Start der Session als Nullpunkt des dreidimensionalen Koordinatensystems definiert ("`World Coordinate Space"'). Abbildung \ref{fig:arkit-worldalignment-gravity} aus der Apple Dokumentation zeigt diese Einstellung. Es entspricht der manuellen Konfiguration der Session-Option \texttt{worldAlignment} auf \texttt{gravity}.

\bild[https://developer.apple.com/documentation/arkit/arconfiguration.worldalignment/2873778-gravity]{0.4}{arkit-worldalignment-gravity}{AR-Koordinatensystem mit Gravity als World Alignment}

\subsubsection{Anker}
Dem Modell können sogenannte Anker vom Typ \texttt{ARAnchor} hinzugefügt werden, die genutzt werden können, um Objekte zu platzieren. Wenn die Flächendetektion mit \texttt{planeDetection} aktiviert ist, fügt ARKit der Session automatisch \texttt{ARPlaneAnchor}s hinzu. Bei der Gesichtserkennung werden \texttt{ARFaceAnchor} verwendet und bei Bilderkennung \texttt{ARImageAnchor}. Jeder Anker definiert sein eigenes lokales Koordinatensystem, das zur Platzierung von Objekten verwendet werden kann.


\subsection{SpriteKit}
SpriteKit wurde mit iOS 7 eingeführt und bietet im Wesentlichen Werkzeuge für 2D Animationen. Es umfasst zudem eine Physik-Engine und Eventhandling, sodass es für Spiele genutzt werden kann.


\subsection{SceneKit}
Auf SpriteKit folgend, fügte Apple in iOS 8 mit SceneKit ein high-level Framework für 3D Grafiken hinzu. Das Framework war zuvor bereits in macOS im Einsatz. Wie SpriteKit beinhaltet es eine Physik-Engine, Eventhandling und ein Partikelsystem. Szenen können mit dreidimensionalen Geometrien, Materialien, Lichtern, Animationen und Kameras beschrieben werden. Die Elemente werden in \texttt{SCNScene} in einem Szenengraph hierarchisch verwaltet, mit der \texttt{rootNode} als Wurzelknoten.

% TODO: zur Physics Engine was schreiben

}


\section{Lösungsentwicklung}

\section{Umsetzung}

\section{Validierung}

\section{Schlussfolgerungen}
\subsection{Lessons Learned}
\subsection{Ausblick}


%----------------------------------------------------------------------------------------
%	Bibliography
%----------------------------------------------------------------------------------------
\newpage
\section{Quellen und Verzeichnisse}
\bibliographystyle{apacite}
\bibliography{Quellen}
%------------------------------------------------
\newpage
%\section{Abbildungsverzeichnis}
\listoffigures
\addtocontents{lof}{\protect\thispagestyle{headings}}
%------------------------------------------------
\newpage
%\section{Tabellenverzeichnis}
\listoftables
\addtocontents{lot}{\protect\thispagestyle{headings}}

%----------------------------------------------------------------------------------------
%	Appendix
%----------------------------------------------------------------------------------------
\newpage
\pagestyle{headings}
\appendix
\setcounter{figure}{0}
\setcounter{table}{0}
\renewcommand{\thefigure}{A\arabic{figure}}
\renewcommand{\thetable}{A\arabic{table}}
\renewcommand{\thesubsection}{A\arabic{subsection}}


\IfFileExists{99-Appendix}{
  \section{Anhang}

\subsection{Projektmanagement}

\bild{1}{timeline}{Meilenstein Planung}

\begin{table}
	\begin{tabular}{l l l}
		\hline
		\textbf{MS} & \textbf{Datum} & \textbf{Inhalt} \\
		\hline
		1	& 27. Februar 	& Kick-Off Meeting \\
		2	& 13. März 		& Start Ideenfindung \\
		3	& 10. April 	& Besprechung Ideenfindung und Start Ideenauswahl \\
		4	& 8.  Mai 		& Start Umsetzung \\
		5	& 22. Mai 		& Besprechung erreichter Ziele \\
		6	& 8.  Juni 		& Nachbearbeitung, Abgabe Projekt \\
		\hline
	\end{tabular}
	\caption{Auflistung der Meilensteine}
	\label{tab:meilensteine}
\end{table}

}

\end{document}
