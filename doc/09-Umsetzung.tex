\section{Umsetzung}

In diesem Kapitel wird die Demo-App \textbf{ARMarbleRun} beschrieben, die im Rahmen dieses Projekts erstellt wurde.
Der Code der App, genauso wie der Prototypen, ist auf GitHub unter \url{https://github.com/eduDorus/PAWI} frei verfügbar.

\subsection{Konzept}

Zur Illustration der Möglichkeiten der AR Technologie soll eine lauffähige iOS Demo-App entstehen.
Im Abschnitt zur Lösungswahl (\ref{sub:loesungswahl}) wurde anhand der erarbeiteten Prototypen entschieden, dass die App primär zwei Anwendungsfälle enthalten soll:

\begin{itemize}
	\item Eine interaktive Bauanleitung, bei der eine virtuelle Kugelbahn in Augmented Reality als Anleitung für den Benutzer zum Bau der physischen Kugelbahn verwendet wird.
	\item Der Bau und die Bearbeitung von virtuellen Kugelbahnen in Augmented Reality.
\end{itemize}

Eine ausführliche Architekturdokumentation findet sich im Anhang \ref{appendix:architekturdokumentation}.
Darin enthalten sind die kompletten Funktionalen Anforderungen als User Stories mit Akzeptanzkriterien (\ref{appendix:funktionale-anforderungen}) und Nichtfunktionalen Anforderungen (\ref{appendix:nichtfunktionale-anforderungen}).

\bild{1}{mockup-flow}{Mockups der Bildschirme und deren Zusammenhänge}

\textit{Beschreibung der Funktionalitäten und des Aufbaus. Übersicht der wesentlichen Use Cases (mit Verweis auf User Stories im Anhang), Mockups usw.}
% TODO:

\subsection{Softwarearchitektur}

\subsubsection{VIPER Architektur}

In einem Artikel auf der Webseite obj.io von \cite{viper-objcio} wird als Alternative zu der Architektur MVC (Model-View-Controller) das Modell VIPER vorgestellt.
Die Architektur verfolgt einerseits das Ziel sogenannte "`Massive View Controllers"' zu vermeiden, bei denen zu viel Logik in die Controller von MVC gesteckt wird.
Andererseits ist es ein Versuch die von Robert C. Martin vorgeschlagene Clean Architecture in iOS umzusetzen (\cite{clean-architecture}).
Der Name VIPER ist ein Backronym, das für folgende Komponenten steht (Auflistung frei aus dem Artikel von \cite{viper-objcio} übersetzt):

\begin{itemize}
	\item \textbf{View:} zeigt an, was vom Presenter mitgeteilt wird und leitet Benutzerinteraktionen an diesen weiter
	\item \textbf{Interactor:} enthält die eigentliche Businesslogik
	\item \textbf{Presenter:} beinhaltet Logik für die View, um die Daten vom Interactor aufzubereiten und reagiert auf Benutzereingaben
	\item \textbf{Entities:} sind grundlegende Datenmodelle, die primär durch den Interactor genutzt werden
	\item \textbf{Router/Wireframe:} enthält die Navigationslogik und ist Verbindungsglied zwischen einzelnen Modulen/Bildschirmen
\end{itemize}

\bild[https://www.objc.io/issues/13-architecture/viper/]{0.7}{viper-diagram-objc}{VIPER Diagramm der Komponenten}

Abbildung \ref{fig:viper-diagram-objc} zeigt den Zusammenhang der Komponenten in der Grundidee von VIPER.
Pro Bildschirm wird grundsätzlich ein Modul erstellt, das aus den VIPER Komponenten besteht.
Wireframes kennen die jeweils für sie relevanten Wireframes anderer Module, um diese aufzurufen.
Das erlaubt eine starke Trennung der verschiedenen Module und innerhalb der Module die Trennung von Verantwortlichkeiten.

Basierend darauf wurde die Grundarchitektur für das Projekt wie in Abbildung \ref{fig:project-viper-architecture} dargestellt aufgebaut.
Für den Wechsel zwischen Modulen ruft das aktive Wireframe die statische \texttt{createModule()} Methode auf dem entsprechenden Ziel-Wireframe auf.
Das aufgerufene Wireframe instantiiert alle konkreten Implementationen der Komponenten des Moduls (inklusive sich selber) und injiziert die notwendigen Abhängigkeiten.
Schlussendlich gibt sie die View als Rückgabewert.
Das aufrufende Wireframe präsentiert die erhaltene View und gibt so die Kontrolle ab.
Die \texttt{weak} Referenz vom Presenter zur View vermeidet eine zyklische Abhängigkeit, wodurch beide Objekte im Speicher bleiben würden, da der Referenzenzähler bei 1 bliebe (\cite{automatic-reference-counting}).

\bild{1}{project-viper-architecture}{Projekt Architektur nach VIPER}

% TODO: Wahl für VIPER begründen
% TODO: Verweis auf Architekturdokumentation Teil zum Stil und Sichten

\subsubsection{Bestandteile} % TODO: bessere Bezeichnung finden

\bild{0.3}{xcode-projekt-struktur}{Projektstruktur von ARMarbleRun in Xcode}

In Xcode wurden das Projekt wie in Abbildung \ref{fig:xcode-projekt-struktur} ersichtlich in verschiedenen Gruppen organisiert.
Pro Bildschirm wird grundsätzlich ein VIPER \textbf{Modul} erstellt, diese werden folgend in Abschnitt \ref{sub:umsetzung-module} im Detail erläutert.
Weitere Gruppen sind \textbf{Nodes} und \textbf{Entities}.
In ersteren sind die in den AR Views benötigten Nodes für die Kugelbahn, Elemente, Bounding Boxen und Flächen enthalten.
In letzterem sind die beiden zu persistierenden Entitäten für die Kugelbahn und die Elemente.
In \textbf{Commons} sind Klassen, Protokolle und Enumerationen enthalten, die an mehreren Orten verwendet werden.

\subsection{Module}\label{sub:umsetzung-module}

Im Folgenden werden die einzelnen Module der Demo-App einzeln beschrieben.
Die App besteht aus folgenden VIPER-Modulen:

\begin{itemize}
	\item \textbf{SelectMode:} Startbildschirm mit der Auswahl zwischen Editor und Guide
	\item \textbf{MarbleRunList:} Auflistung der gespeicherten Kugelbahnen zur Auswahl durch den Benutzer, für den Editor wird zusätzlich die Option zum Erstellen einer neuen Bahn angeboten
	\item \textbf{AREditor:} AR Bildschirm für den Editor Modus
	\item \textbf{ARGuide:} AR Bildschirm für die Bauanleitung Modus
	\item \textbf{ElementList:} Auflistung der verfügbaren Elementtypen zur Auswahl durch den Benutzer (vom Editor verwendet)
\end{itemize}

Das Zusammenspiel der Module ist in Abbildung \ref{fig:viper-modules} ersichtlich.
Sie zeigt, von wo man zu welchen Modulen gelangt.

\bild{0.6}{viper-modules}{Module der Demo-App}

\subsubsection{Select Mode}

Das \texttt{SelectMode} Modul ist der Startbildschirm der App.
Abbildung \ref{fig:classes-selectmode} zeigt das Klassendiagramm des Moduls.

\bild{1}{classes-selectmode}{Klassendiagramm des Moduls "`Select Mode"'}

Der Bildschirm besteht im Wesentlichen aus den beiden Buttons "`Editor"' und "`Guide"' für die entsprechenden Modi.
Das Drücken auf einen der Buttons wird direkt an den Presenter weitergegeben, welcher die \texttt{presentMarbleRunListView(from:with:)} Methode des Wireframes aufruft.
Der Parameter \texttt{with} ist vom Enum Typ \texttt{ARInteractionMode}, das zwei Fälle für Editor und Guide hat (mehr dazu unter \ref{sub:umsetzung-module}) und entsprechend dem gedrückten Button mitgegeben wird.

In der Navigationbar des Bildschirms befindet sich ausserdem ein Info-Button, der direkt im Storyboard mit der "`About View"' verbunden ist und diese modal präsentiert. % TODO: ref zu AboutView Abschnitt

\subsubsection{Marble Run List}

Im Modul \texttt{MarbleRunList} werden alle gespeicherten Kugelbahnen abgerufen und angezeigt.
Abbildung \ref{fig:classes-marblerunlist} zeigt das Klassendiagramm des Moduls.

\bild{1}{classes-marblerunlist}{Klassendiagramm des Moduls "`Marble Run List"'}

\subsubsection{AR Guide}

\texttt{ARGuide} ist eines der beiden AR Module und damit Kernstück der App.
Es beinhaltet die Bauanleitung.
Abbildung \ref{fig:classes-arguide} zeigt das Klassendiagramm des Moduls.

\bild{1}{classes-arguide}{Klassendiagramm des Moduls "`AR Guide"'}

\subsubsection{AR Editor}

Im \texttt{AREditor} Modul wird die gewählte Kugelbahn in Augmented Reality bearbeitbar gemacht.
Abbildung \ref{fig:classes-areditor} zeigt das Klassendiagramm des Moduls.

\bild{1}{classes-areditor}{Klassendiagramm des Moduls "`AR Editor"'}

\subsubsection{Element List}

Das \texttt{ElementList} Modul listet alle verfügbaren Elementtypen auf, die der aktuell bearbeiteten Kugelbahn hinzugefügt werden können.
Es ist ähnlich dem \texttt{MarbleRunList} Modul.
Abbildung \ref{fig:classes-elementlist} zeigt das Klassendiagramm des Moduls.

\bild{1}{classes-elementlist}{Klassendiagramm des Moduls "`Element List"'}


\subsection{Persistenz}

\subsubsection{Data Manager}

Über den \texttt{MarbleRunDataManager}.

\subsubsection{Entitäten}

Über die beiden Entitäten \texttt{MarbleRunEntity} und \texttt{ElementEntity} und ihre Codierung/Decodierung.


\subsection{Nodes}

Zu den SceneKit Nodes \texttt{BoundingBoxNode}, \texttt{ElementNode}, \texttt{MarbleRunNode} und \texttt{PlaneNode}. Inklusive dem \texttt{ElementProtocol}.


\subsection{Storyboard}

Abbildung \ref{fig:project-storyboard} zeigt das Storyboard des Projekts.

\bild{1}{project-storyboard}{Das Storyboard des Projekts}


\subsection{Weitere Bestandteile}

\subsubsection{Triple}

\subsubsection{Enums} \label{subsub:umsetzung-enums}

Über \texttt{ARModeState}, \texttt{ARInteractionMode}

\subsubsection{DepthFirstSort}

\subsubsection{ARViewController}

\subsubsection{Initializer}

\subsubsection{Informationsbildschirm}

Kurzer Text zur \texttt{AboutView}.
