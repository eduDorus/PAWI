\section{Umsetzung}

In diesem Kapitel wird die Demo-App \textbf{ARMarbleRun} beschrieben, die im Rahmen dieses Projekts erstellt wurde.
Der Code der App, genauso wie der Prototypen, ist auf GitHub unter \url{https://github.com/eduDorus/PAWI} frei verfügbar.

\subsection{Konzept}

Zur Illustration der Möglichkeiten der AR Technologie soll eine lauffähige iOS Demo-App entstehen.
Im Abschnitt zur Lösungswahl (\ref{sub:loesungswahl}) wurde anhand der erarbeiteten Prototypen entschieden, dass die App primär zwei Anwendungsfälle enthalten soll:

\begin{itemize}
	\item Eine interaktive Bauanleitung, bei der eine virtuelle Kugelbahn in Augmented Reality als Anleitung für den Benutzer zum Bau der physischen Kugelbahn verwendet wird.
	\item Der Bau und die Bearbeitung von virtuellen Kugelbahnen in Augmented Reality.
\end{itemize}

Eine ausführliche Architekturdokumentation findet sich im Anhang \ref{appendix:architekturdokumentation}.
Darin enthalten sind die kompletten Funktionalen Anforderungen als User Stories mit Akzeptanzkriterien (\ref{appendix:funktionale-anforderungen}) und Nichtfunktionalen Anforderungen (\ref{appendix:nichtfunktionale-anforderungen}).

\bild{1}{mockup-flow}{Mockups der Bildschirme und deren Zusammenhänge}

\textit{Beschreibung der Funktionalitäten und des Aufbaus. Übersicht der wesentlichen Use Cases (mit Verweis auf User Stories im Anhang), Mockups usw.}
% TODO:

\subsection{Softwarearchitektur}

\subsubsection{VIPER Architektur}

In einem Artikel auf der Webseite obj.io von \cite{viper-objcio} wird als Alternative zu der Architektur MVC (Model-View-Controller) das Modell VIPER vorgestellt.
Die Architektur verfolgt einerseits das Ziel sogenannte "`Massive View Controllers"' zu vermeiden, bei denen zu viel Logik in die Controller von MVC gesteckt wird.
Andererseits ist es ein Versuch die von Robert C. Martin vorgeschlagene Clean Architecture in iOS umzusetzen (\cite{clean-architecture}).
Der Name VIPER ist ein Backronym, das für folgende Komponenten steht (Auflistung frei aus dem Artikel von \cite{viper-objcio} übersetzt):

\begin{itemize}
	\item \textbf{View:} zeigt an, was vom Presenter mitgeteilt wird und leitet Benutzerinteraktionen an diesen weiter
	\item \textbf{Interactor:} enthält die eigentliche Businesslogik
	\item \textbf{Presenter:} beinhaltet Logik für die View, um die Daten vom Interactor aufzubereiten und reagiert auf Benutzereingaben
	\item \textbf{Entities:} sind grundlegende Datenmodelle, die primär durch den Interactor genutzt werden
	\item \textbf{Router/Wireframe:} enthält die Navigationslogik und ist Verbindungsglied zwischen einzelnen Modulen/Bildschirmen
\end{itemize}

\bild[https://www.objc.io/issues/13-architecture/viper/]{0.7}{viper-diagram-objc}{VIPER Diagramm der Komponenten}

Abbildung \ref{fig:viper-diagram-objc} zeigt den Zusammenhang der Komponenten in der Grundidee von VIPER.
Pro Bildschirm wird grundsätzlich ein Modul erstellt, das aus den VIPER Komponenten besteht.
Wireframes kennen die jeweils für sie relevanten Wireframes anderer Module, um diese aufzurufen.
Das erlaubt eine starke Trennung der verschiedenen Module und innerhalb der Module die Trennung von Verantwortlichkeiten.

Basierend darauf wurde die Grundarchitektur für das Projekt wie in Abbildung \ref{fig:project-viper-architecture} dargestellt aufgebaut.
Für den Wechsel zwischen Modulen ruft das aktive Wireframe die statische \texttt{createModule()} Methode auf dem entsprechenden Ziel-Wireframe auf.
Das aufgerufene Wireframe instanziiert alle konkreten Implementationen der Komponenten des Moduls (inklusive sich selber) und injiziert die notwendigen Abhängigkeiten.
Schlussendlich gibt sie die View als Rückgabewert.
Das aufrufende Wireframe präsentiert die erhaltene View und gibt so die Kontrolle ab.
Die \texttt{weak} Referenz vom Presenter zur View vermeidet eine zyklische Abhängigkeit, wodurch beide Objekte im Speicher bleiben würden, da der Referenzenzähler bei 1 bliebe (\cite{automatic-reference-counting}).

\bild{1}{project-viper-architecture}{Projekt Architektur nach VIPER}

% TODO: Wahl für VIPER begründen
% TODO: Verweis auf Architekturdokumentation Teil zum Stil und Sichten

\subsubsection{Bestandteile} % TODO: bessere Bezeichnung finden

\bild{0.3}{xcode-projekt-struktur}{Projektstruktur von ARMarbleRun in Xcode}

In Xcode wurden das Projekt wie in Abbildung \ref{fig:xcode-projekt-struktur} ersichtlich in verschiedenen Gruppen organisiert.
Pro Bildschirm wird grundsätzlich ein VIPER \textbf{Modul} erstellt, diese werden folgend in Abschnitt \ref{sub:umsetzung-module} im Detail erläutert.
Weitere Gruppen sind \textbf{Nodes} und \textbf{Entities}.
In ersteren sind die in den AR Views benötigten Nodes für die Kugelbahn, Elemente, Bounding Boxen und Flächen enthalten.
In letzterem sind die beiden zu persistierenden Entitäten für die Kugelbahn und die Elemente.
In \textbf{Commons} sind Klassen, Protokolle und Enumerationen enthalten, die an mehreren Orten verwendet werden.

\subsection{Module}\label{sub:umsetzung-module}

Im Folgenden werden die einzelnen Module der Demo-App einzeln beschrieben.
Die App besteht aus folgenden VIPER-Modulen:

\begin{itemize}
	\item \textbf{SelectMode:} Startbildschirm mit der Auswahl zwischen Editor und Guide
	\item \textbf{MarbleRunList:} Auflistung der gespeicherten Kugelbahnen zur Auswahl durch den Benutzer, für den Editor wird zusätzlich die Option zum Erstellen einer neuen Bahn angeboten
	\item \textbf{AREditor:} AR Bildschirm für den Editor Modus
	\item \textbf{ARGuide:} AR Bildschirm für die Bauanleitung Modus
	\item \textbf{ElementList:} Auflistung der verfügbaren Elementtypen zur Auswahl durch den Benutzer (vom Editor verwendet)
\end{itemize}

Das Zusammenspiel der Module ist in Abbildung \ref{fig:viper-modules} ersichtlich.
Sie zeigt, von wo man zu welchen Modulen gelangt.

\bild{0.6}{viper-modules}{Module der Demo-App}

\subsubsection{Select Mode}

Das \texttt{SelectMode} Modul ist der Startbildschirm der App.
Abbildung \ref{fig:classes-selectmode} zeigt das Klassendiagramm des Moduls.

\bild{1}{classes-selectmode}{Klassendiagramm des Moduls "`Select Mode"'}

Der Bildschirm besteht im Wesentlichen aus den beiden Buttons "`Editor"' und "`Guide"' für die entsprechenden Modi.
Das Drücken auf einen der Buttons wird direkt an den Presenter weitergegeben, welcher die \texttt{presentMarbleRunListView(from:with:)} Methode des Wireframes aufruft.
Der Parameter \texttt{with} ist vom Enum Typ \texttt{ARInteractionMode}, das zwei Fälle für Editor und Guide hat (mehr dazu unter \ref{sub:umsetzung-module}) und entsprechend dem gedrückten Button mitgegeben wird.

In der Navigationbar des Bildschirms befindet sich ausserdem ein Info-Button, der direkt im Storyboard mit der "`About View"' verbunden ist und diese modal präsentiert. % TODO: ref zu AboutView Abschnitt

\subsubsection{Marble Run List}

Im Modul \texttt{MarbleRunList} werden alle gespeicherten Kugelbahnen abgerufen und angezeigt.
Abbildung \ref{fig:classes-marblerunlist} zeigt das Klassendiagramm des Moduls.

\bild{1}{classes-marblerunlist}{Klassendiagramm des Moduls "`Marble Run List"'}

Für beide Modi wird nach der Moduswahl die Marble Run List angezeigt.
Das Wireframe erhält die Information zum gewählten Modus über das \texttt{ARInteractionMode} Enum.
Die View besitzt eine Collection View in der sie die Kugelbahnen anzeigt und adoptiert die entsprechenden Protokolle \texttt{UICollectionViewDataSource} und \texttt{UICollectionViewDelegate}.
Sobald sie geladen ist, informiert sie den Presenter (\texttt{viewDidLoad()}), der daraufhin vom Interactor die gespeicherten Kugelbahnen anfragt.
Der Interactor erhält von \texttt{MarbleRunDataManager} (siehe \ref{subsub:umsetzung-datamanager}) die Liste der Kugelbahnen als Array des Typs \texttt{MarbleRunEntity}, welches der Presenter dann über die Methode \texttt{reloadMarbleRunList(with:)} an die View gibt.

Falls der Benutzer den Editor Modus ausgewählt hat, setzt das Wireframe bei der Instanziierung das \texttt{canAddNew} Attribut der View auf \texttt{true}.
Dadurch wird die View an der ersten Stelle der Collection View eine Zelle zum Erstellen einer neuen Kugelbahn setzen.
Beide Zelltypen der Collection (normale Kugelbahn Zelle und Zelle zum Erstellen) sind im Storyboard mit den Bezeichnungen "`MarbleRunCell"' und "`NewRunCell"' erstellt.
Für das Hinzufügen einer neuen Bahn präsentiert die \texttt{showNewDialogue()} einen \texttt{UIAlertCollection} mit Textfeld für den Namen der Bahn.
Dieser Name wird bei einem "`OK"' des Benutzers dem Presenter mitgeteilt, der den Interactor eine neue Kugelbahn mit diesem Namen erstellen lässt und anschliessend das Wireframe auffordert das nächste Modul zu laden.

Die Interactor Methode \texttt{createNewMarbleRun()} (Code \ref{code:createnewmarblerun}) erstellt eine neue \texttt{MarbleRunEntity} (Zeile 2) und fügt ihr einen soliden Block (Element vom Typ 12) and der Nullkoordinate hinzu (Zeilen 3 und 4).
Schliesslich wird die neue Entität dem \texttt{MarbleRunDataManager} zur Persistierung übergeben.

\begin{code}{createnewmarblerun}{\texttt{createNewMarbleRun(with:)} Methode des Marble Run List Interactors}
    func createNewMarbleRun(with name: String) -> MarbleRunEntity {
        let marbleRun = MarbleRunEntity(name: name)
        let baseElement = ElementEntity(type: 12, location: Triple(0,0,0))
        marbleRun.elements.append(baseElement)
        MarbleRunDataManager.persist(marbleRun)
        return marbleRun
    )
\end{code}

In beiden Fällen gibt das Wireframe die gewählte (oder erstellte) \texttt{MarbleRunEntity} an das nächste Modul weiter.

\subsubsection{AR Guide}

\texttt{ARGuide} ist eines der beiden AR Module und damit Kernstück der App.
Es beinhaltet die interaktive Bauanleitung.
Abbildung \ref{fig:classes-arguide} zeigt das Klassendiagramm des Moduls.

\bild{1}{classes-arguide}{Klassendiagramm des Moduls "`AR Guide"'}

Das Initialisieren der \texttt{SceneView} und des AR World Tracking übernimmt der \texttt{ARViewController} (siehe \ref{subsub:umsetzung-arviewcontroller}).
Sobald der Benutzer eine erkannte Fläche gewählt hat, übernimmt die \texttt{ARGuideView} die weiteren Aktionen und teilt dem Presenter mit, dass sie bereits ist, die Kugelbahn darzustellen.
Der Presenter fragt vom Interactor die Elemente der Kugelbahn an und fügt sie der View zu.
In \texttt{InitializeMarbleRun} der View wird eine \texttt{MarbleRunNode} erstellt und der Szene hinzugefügt, Elemente werden dann als Kindknoten der Kugelbahn ergänzt.
Die Positionierung der Kugelbahn erfolgt nach dem in Prototyp \ref{subsub:prot-kugelbahn} erarbeiteten Prinzip.

Ist die Position der Kugelbahn fixiert und der Startbutton wird gedrückt, lässt der Interactor von \texttt{DepthFirstSort} die Bauanleitung erstellen (siehe \ref{subsub:umsetzung-depthfirst}).
Der Rückgabewert ist ein Array von Koordinaten, in der Reihenfolgen, in der sie abgearbeitet werden sollen.
Mittels Vor- und Zurück-Buttons kann der Benutzer in der Anleitung navigieren.
Diese Aktionen erhöhen oder verringern im Interactor den \texttt{currentPointer}, der auf die aktuelle aktive Koordinate im Array \texttt{guide} zeigt.

Die View hat keine Kenntnisse über die Anleitung, sondern erhält als Reaktion auf Benutzerinteraktionen vom Presenter mit der Methode \texttt{set(elementAt:to:)} Anweisungen, welche Elemente der Kugelbahn welchen Status erhalten sollen.
So wird beispielsweise das aktive Element der Bauanleitung mit \texttt{ElementState.highlighted} hervorgehoben.

Als einziges Modul in diesem Projekt, nutzt der AR Editor für den Interactor ein Eingabe und Ausgabe Protokoll.
Während das \texttt{ARGuideInteractorInputProtocol} wie bei anderen Modulen vom Presenter genutzt wird um den Interactor anzusprechen, ist \texttt{ARGuideInteractorOutputProtocol} für die umgekehrte Richtung zuständig.

% TODO: Menu Action?

\subsubsection{AR Editor}

Im \texttt{AREditor} Modul wird die gewählte Kugelbahn in Augmented Reality bearbeitbar gemacht.
Abbildung \ref{fig:classes-areditor} zeigt das Klassendiagramm des Moduls.

\bild{1}{classes-areditor}{Klassendiagramm des Moduls "`AR Editor"'}

\subsubsection{Element List}

Das \texttt{ElementList} Modul listet alle verfügbaren Elementtypen auf, die der aktuell bearbeiteten Kugelbahn hinzugefügt werden können.
Es ist ähnlich dem \texttt{MarbleRunList} Modul.
Abbildung \ref{fig:classes-elementlist} zeigt das Klassendiagramm des Moduls.

\bild{1}{classes-elementlist}{Klassendiagramm des Moduls "`Element List"'}


\subsection{Persistenz}

\subsubsection{Data Manager} \label{subsub:umsetzung-datamanager}

Über den \texttt{MarbleRunDataManager}.

\subsubsection{Entitäten und Nodes}

Über die beiden Entitäten \texttt{MarbleRunEntity} und \texttt{ElementEntity} und ihre Codierung/Decodierung.

Zu den SceneKit Nodes \texttt{BoundingBoxNode}, \texttt{ElementNode}, \texttt{MarbleRunNode} und \texttt{PlaneNode}. Inklusive dem \texttt{ElementProtocol}.


\subsection{Storyboard}

Abbildung \ref{fig:project-storyboard} zeigt das Storyboard des Projekts.

\bild{1}{project-storyboard}{Das Storyboard des Projekts}


\subsection{Weitere Bestandteile}

\subsubsection{Triple}

\subsubsection{Enums} \label{subsub:umsetzung-enums}

Über \texttt{ARModeState}, \texttt{ARInteractionMode}

\subsubsection{DepthFirstSort} \label{subsub:umsetzung-depthfirst}

\subsubsection{ARViewController} \label{subsub:umsetzung-arviewcontroller}

\subsubsection{Initializer}

\subsubsection{Informationsbildschirm}

Kurzer Text zur \texttt{AboutView}.
