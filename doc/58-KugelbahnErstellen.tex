\subsubsection{Schrittweiser Aufbau einer Bahn durch Benutzer}\label{subsub:prot-kugelbahneditor}
\begin{description}
	\item[Fragestellung:] Wie kann der Schrittweiser Aufbau einer Bahn durch Benutzer erfolgen?
	\item[Resultat:] Eine spezifische Klasse \texttt{TrackEditor} enthält die Logik zum augbauen einer neuen Bahn. Die Klasse nutzt die Methoden von \texttt{MarbleTrack}, um die Bahn zu manipulieren und neue Elemente der Bahn hinzuzufügen.
	\item[Versuchsaufbau:] Der Versuch baut auf dem Prototyp \ref{subsub:prot-kugelbahnaufbau} auf. Es wurde nicht nur versucht einen Editor zu erstellen sondern auch Wissensgewinne und Implementationen des vorherigen Prototyps einzubinden. Grundsätzlich können nur Elemente gebaut werden die direkt an einem bereits gebauten Element angrenzen. Zusätzlich können Elemente auf gebauten Elementen erstellt werden. 

	\textbf{Übersicht Klassen}
	Die Klasse \texttt{TrackEditor} wurde dem Demoprojekt hinzugefügt. Zusätzlich musste der \texttt{ViewController} und \texttt{MarbleTrack} angepasst werden. Die übrigen Klassen wie \texttt{BasicCube}, \texttt{Triple} und \texttt{TrackMap} sind die Klassen im Wesentlichen wie von Prototyp \ref{subsub:prot-kugelbahnaufbau}.
	\begin{itemize}
		\item \texttt{MarbleTrack}: erbt von \texttt{SCNNode}, hält Koordinaten verschiedener Kugelbahnen, kann eine solche Bahn mit \texttt{BasicCube} Elementen erstellen, verfügt über Methoden zur Positionierung und Ausrichtung der gesamten Bahn, besitzt Referenzen auf die Elemente der Bahn in einer \texttt{TrackMap}
		\item \texttt{TrackEditor}: nutzt die Klasse \texttt{MarbleTrack} um den Status von Elementen der Kugelbahn zu verändern oder Elemente der Bahn hinzuzufügen/zu entfernen. Beim hinzufügen von neuen Elementen wird berechnet welche umliegende Elemente als Mögliche Elemente anzuzeigen sind.
	\end{itemize}

	\textbf{Hinzufügen neuer Elemente}\\
	Die Klasse \texttt{TrackEditor} erstellt bei der Initialisierung das Root-Element auf der Referenz (0, 0, 0) in der aktuellen \texttt{MarbleTrack} Klasse. Nun wird folgender Algorithmus verwendet um neue Elemente hinzuzufügen:
	
	\begin{enumerate}
		\item Beim erstellen des Root-Elements werden die Umliegenden Felder geprüft und Elemente mit dem Status \texttt{planned} hinzugefügt.
		\item Mit dem Antippen von einem Element wird ein Hit-Test durchgeführt und prüft ob es sich um ein Element mit dem Status \texttt{planned} handelt. Falls dies der Fall ist wird der Status des Elements auf \texttt{build} gesetzt
		\item Anschliessend wird berechnet wo die nächsten neuen Elemente erstellt werden könnten und diese Werden über die Klasse \texttt{MarbleTrack} mit dem Status \texttt{planned} hinzugefügt.
	\end{enumerate}

	Bei der Besprechung des Prototyps wurde festgestellt, dass das Löschen von Elementen eine ähnliche Überprüfung benötigt. Grundsätzlich dürfen nur Elemente entfernt werden die keine Inseln in der Bahn verursachen bzw. es muss immer eine Verbindung zum Ursprung bestehen.

\end{description}
