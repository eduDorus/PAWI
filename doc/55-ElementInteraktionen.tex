\subsubsection{Mittels Touchgesten mit virtuellen Objekten interagieren}
\begin{description}
	\item[Fragestellung:] Wie kann mit Touch- und Swipegesten mit virtuellen Objekten interagiert werden?
	\item[Resultat:] Es konnte erfolgrich ein Prototyp entwickelt werden, bei dem es Möglich ist mittels verschiedenen Gesten mit einem virtuellen Objekt zu interagieren. Eine besondere Schwierigkeit bestand darin das Element zu kippen da je nach Kameraperspektive um die Z- oder X Achse gedreht werden muss. 
    \item[Versuchsaufbau:] Für diesen Versuch wurde ein neues XCode Projekt initialisiert. Anschliessend wurde die Logik implementiert um einen Würfel auf einer Fläche zu augmentieren. Es wurde die Funktionalität entwickelt mittels verschiedenen Gesten mit dem Würfel zu interagieren. Zu den funktionalitäten gehören:
    
    
    \begin{itemize}
        \item \texttt{UITapGestureRecognizer} mit einmaligem Tippen
        \item \texttt{UITapGestureRecognizer} mit zweimaligem Tippen
        \item \texttt{UILongPressGestureRecognizer} für langes antippen eines Objektes
        \item \texttt{UISwipeGestureRecognizer} für die Rotationen
    \end{itemize}


    \textbf{\texttt{UITapGestureRecognizer} mit einmaligem Tippen}\\
    Damit mit Touchgesten gearbeitet werden kann müssen diese der SceneView hinzugefügt werden Code \ref{code:prot-gesture-addTapGestureToSceneView}. Für einzelne Taps können wir den \texttt{UITapGestureRecognizer} initialisieren und als action parameter eine Methode angeben. Wir geben hierfür die Methode \texttt{ViewController.didTap(withGestureRecognizer:)}. Zusätzlich sollte die Anzahl benötigter Taps mit \texttt{tapGestureRecognizer.numberOfTapsRequired} auf Eins gesetzt werden, da wir auch einen DoubleTap erkennen wollen. 
    \begin{code}{prot-gesture-addTapGestureToSceneView}{Methode \texttt{addTapGestureToSceneView()} um das einmalige antippen der \texttt{SceneView} hinzuzufügen}
    func addTapGestureToSceneView() {
        let tapGestureRecognizer = UITapGestureRecognizer(target: self, action: #selector(ViewController.didTap(\_:)))
        tapGestureRecognizer.numberOfTapsRequired = 1
        sceneView.addGestureRecognizer(tapGestureRecognizer)
    }
    \end{code}

    Sollte erfolgreich ein Tap auf der SceneView registriert werden, so wird die Methode \texttt{didTap(\_ recognizer: UIGestureRecognizer)} im Code \ref{code:prot-gesture-didTap} ausgeführt. Bei dieser Methode werden die 2D Koordinaten des Taps ausgelesen und anschliesend an einem \texttt{hitTest(\_:types:)} im 3D Raum weitergegeben. Dieser HitTest prüft ob es sich um ein SceneKit Node handelt. Falls dies der Fall ist wird dieses Element selektiert und als interagierbares Objekt gesetzt. Dies geschied mir der Methode \texttt{updateSelectedObject(node: node)} aus dem Code \ref{code:prot-gesture-updateSelectedObject}. In diesem Code abschnitt werden auch zwei \texttt{SCNTransaction} ausgeführt damit der Benutzer anhand einer Visuellen animation erkennt welches das aktuell selektierte Objekt ist. Mittels \texttt{SCNTransaction} können komplexe Animationen erstellt werden und innerhalb einer Transaktion ausgeführt werden.

    \begin{code}{prot-gesture-didTap}{Methode \texttt{didTap(\_ recognizer: UIGestureRecognizer)} die beim einmaligen antippen der \texttt{SceneView} ausgeführt wird}
    @objc
    func didTap(_ recognizer: UIGestureRecognizer) {
        let tapLocation = recognizer.location(in: sceneView)
        let hitTestOptions: [SCNHitTestOption: Any] = [.boundingBoxOnly: true]
        let hitTestResults = sceneView.hitTest(tapLocation, options: hitTestOptions)

        guard let node = hitTestResults.first?.node else {return}
        
        updateSelectedObject(node: node)
    }
    \end{code}

    \begin{code}{prot-gesture-updateSelectedObject}{Methode \texttt{updateSelectedObject(node: node)} zur aktualisierung des ausgewählten Objekts}
    func updateSelectedObject(node: SCNNode) {
        // reset color of old selected object
        SCNTransaction.begin()
        SCNTransaction.animationDuration = 0.5
        selectedObject?.geometry?.firstMaterial?.diffuse.contents = UIColor(red: 182.0 / 255.0, green: 155.0 / 255.0, blue: 76.0 / 255.0, alpha: 1)
        SCNTransaction.commit()
        
        // Set new selected node
        selectedObject = node
        
        // Set selected color
        SCNTransaction.begin()
        SCNTransaction.animationDuration = 0.5
        selectedObject?.geometry?.firstMaterial?.diffuse.contents = UIColor(red: 130.0 / 255.0, green: 82.0 / 255.0, blue: 1.0 / 255.0, alpha: 1)
        SCNTransaction.commit()
    }
    \end{code}

    \textbf{\texttt{UITapGestureRecognizer} mit zweimaligem Tippen}\\

    \begin{code}{prot-gesture-addDoubleTapGestureToSceneView}{Methode \texttt{addDoubleTapGestureToSceneView()} um das zweimalige antippen der \texttt{SceneView} hinzuzufügen}
    doubleTapGestureRecognizer.numberOfTapsRequired = 2
    \end{code}

    \begin{code}{prot-gesture-didTap}{Methode \texttt{didTap(withGestureRecognizer recognizer: UIGestureRecognizer)} die beim einmaligen antippen der \texttt{SceneView} ausgeführt wird}
    @objc
    func didDoubleTap(withGestureRecognizer recognizer: UIGestureRecognizer) {
        let tapLocation = recognizer.location(in: sceneView)
        let hitTestResults = sceneView.hitTest(tapLocation)
        
        guard let node = hitTestResults.first?.node else {
            let hitTestResultWithFeaturePoints = sceneView.hitTest(tapLocation, types: .estimatedHorizontalPlane)
            if let hitTestResultWithFeaturePoints = hitTestResultWithFeaturePoints.first {
                let translation = hitTestResultWithFeaturePoints.worldTransform.translation
                addBox(x: translation.x, y: translation.y, z: translation.z)
            }
            return
        }
    }
    \end{code}

    \textbf{\texttt{UILongPressGestureRecognizer} für langes antippen eines Objektes}\\

    \begin{code}{prot-gesture-addLongTapGestureToSceneView}{Methode \texttt{addLongTapGestureToSceneView()} um das lange antippen der \texttt{SceneView} hinzuzufügen}
    func addLongTapGestureToSceneView() {
        let longTapGestureRecognizer = UILongPressGestureRecognizer(target: self, action: #selector(ViewController.didLongPress(withGestureRecognizer:)))
        sceneView.addGestureRecognizer(longTapGestureRecognizer)
    }
    \end{code}

    \begin{code}{prot-gesture-addLongTapGestureToSceneView}{Methode \texttt{addLongTapGestureToSceneView()} um das lange antippen der \texttt{SceneView} hinzuzufügen}
    @objc
    func didLongPress(withGestureRecognizer recognizer: UILongPressGestureRecognizer) {
        let longPressLocation = recognizer.location(in: sceneView)
        let hitTestResults = sceneView.hitTest(longPressLocation)
        
        guard let node = hitTestResults.first?.node else { return }
        node.removeFromParentNode()
    }
    \end{code}
    
    \textbf{\texttt{UISwipeGestureRecognizer} für die Rotationen}\\
    
    \begin{code}{prot-gesture-addSwipeGestureToSceneView}{Methode \texttt{addSwipeGestureToSceneView()} um die Wischgesten nach links, rechts, unten und oben der \texttt{SceneView} hinzuzufügen}
    func addSwipeGestureToSceneView() {
        let swipeRightGesture = UISwipeGestureRecognizer(target: self, action: #selector(didSwipe(_:)))
        swipeRightGesture.direction = .right
        
        let swipeLeftGesture = UISwipeGestureRecognizer(target: self, action: #selector(didSwipe(_:)))
        swipeLeftGesture.direction = .left
        
        let swipeUpGesture = UISwipeGestureRecognizer(target: self, action: #selector(didSwipe(_:)))
        swipeUpGesture.direction = .up
        
        let swipeDownGesture = UISwipeGestureRecognizer(target: self, action: #selector(didSwipe(_:)))
        swipeDownGesture.direction = .down
        
        sceneView.addGestureRecognizer(swipeLeftGesture)
        sceneView.addGestureRecognizer(swipeRightGesture)
        sceneView.addGestureRecognizer(swipeUpGesture)
        sceneView.addGestureRecognizer(swipeDownGesture)
    }
    \end{code}


    \textbf{Drehen von Würfeln anhand der akutellen Kameraposition}\\

    \begin{code}{prot-gesture-addSwipeGestureToSceneView}{Methode \texttt{didSwipe(\_ gesture: UISwipeGestureRecognizer)} die beim Wischgesten nach links, rechts, unten und oben der \texttt{SceneView} ausgeführt wird}
        @objc
        func didSwipe(_ gesture: UISwipeGestureRecognizer) {
            let currentAngle = sceneView.session.currentFrame?.camera.eulerAngles.y
            var action: SCNAction!
            
            if gesture.direction == .right {
                action = SCNAction.rotateBy(x: 0, y: CGFloat(Double.pi/2), z: 0, duration: 0.5)
            }
            else if gesture.direction == .left {
                action = SCNAction.rotateBy(x: 0, y: CGFloat(-(Double.pi/2)), z: 0, duration: 0.5)
            }
            else if gesture.direction == .up {
                if (currentAngle! > Float(0.785) && currentAngle! < Float(2.356)) {
                    action = SCNAction.rotateBy(x: 0, y: 0, z: CGFloat(Double.pi/2), duration: 0.5)
                }
                else if (currentAngle! > Float(2.356) || currentAngle! < Float(-2.356)) {
                    action = SCNAction.rotateBy(x: CGFloat(Double.pi/2), y: 0, z: 0, duration: 0.5)
                }
                else if (currentAngle! > Float(-2.356) && currentAngle! < Float(-0.785)) {
                    action = SCNAction.rotateBy(x: 0, y: 0, z: CGFloat(-(Double.pi/2)), duration: 0.5)
                }
                else if (currentAngle! > Float(-0.785) && currentAngle! < Float(0.785)) {
                    action = SCNAction.rotateBy(x: CGFloat(-(Double.pi/2)), y: 0, z: 0, duration: 0.5)
                }
            }
            ...
    \end{code}

    \textbf{Zusätzliches}\\

    \begin{code}{prot-gesture-addSwipeGestureToSceneView}{Methode \texttt{updateSelectedObject(node: SCNNode)} um das ausgewählte Element zu animieren}
    func updateSelectedObject(node: SCNNode) {
        // Set selected color
        SCNTransaction.begin()
        SCNTransaction.animationDuration = 0.5
        selectedObject?.geometry?.firstMaterial?.diffuse.contents = UIColor(red: 182.0 / 255.0, green: 155.0 / 255.0, blue: 76.0 / 255.0, alpha: 1)
        SCNTransaction.commit()
        
        // Set new selected node
        selectedObject = node
        
        // Set selected color
        SCNTransaction.begin()
        SCNTransaction.animationDuration = 0.5
        selectedObject?.geometry?.firstMaterial?.diffuse.contents = UIColor(red: 130.0 / 255.0, green: 82.0 / 255.0, blue: 1.0 / 255.0, alpha: 1)
        SCNTransaction.commit()
    }
    \end{code}
    

\end{description}    
