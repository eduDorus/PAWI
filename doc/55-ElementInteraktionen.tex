\subsubsection{Mittels Touchgesten mit virtuellen Objekten interagieren}\label{subsub:prot-interagieren}
\begin{description}
	\item[Fragestellung:] Wie kann mit Tap- und Swipe-Gesten mit virtuellen Objekten interagiert werden?
	\item[Resultat:] Es konnte erfolgreich ein Prototyp entwickelt werden, bei dem mittels Tap- und Swipe-Gesten mit einem virtuellen Objekt interagiert werden kann. Eine besondere Schwierigkeit bestand darin ein Element zu kippen, da je nach Kameraperspektive um die Z- oder X-Achse gedreht werden muss. 
    \item[Versuchsaufbau:] Für diesen Versuch wurde ein neues Xcode Projekt initialisiert. Anschliessend wurde die Logik implementiert, um einen Würfel auf einer Fläche zu augmentieren. Es wurde die Funktionalität entwickelt mittels verschiedenen Gesten mit dem Würfel zu interagieren. Zu den Funktionalitäten gehören:

    \begin{itemize}
        \item \texttt{UITapGestureRecognizer} mit einmaligem Tippen
        \item \texttt{UITapGestureRecognizer} mit zweimaligem Tippen
        \item \texttt{UILongPressGestureRecognizer} für langes Antippen eines Objektes
        \item \texttt{UISwipeGestureRecognizer} für die Rotationen
    \end{itemize}

    \textbf{\texttt{UITapGestureRecognizer} mit einmaligem Tippen}\\
    Um mit Gesten zu arbeiten, müssen diese der Scene View hinzugefügt werden (Code \ref{code:prot-gesture-addTapGestureToSceneView}). Für einzelne Tippgesten kann man den \texttt{UITapGestureRecognizer} initialisieren und als Action Parameter eine Methode angeben (Zeile 2). Zusätzlich sollte die Anzahl benötigter Tippe auf Eins gesetzt werden (Zeile 3), da hier kein doppeltes Antippen berücksichtigt werden soll. 

    \begin{code}{prot-gesture-addTapGestureToSceneView}{Methode \texttt{addTapGestureToSceneView()} für das einmalige Antippen}
    func addTapGestureToSceneView() {
        let tapGestureRecognizer = UITapGestureRecognizer(target: self, action: #selector(ViewController.didTap(\_:)))
        tapGestureRecognizer.numberOfTapsRequired = 1
        sceneView.addGestureRecognizer(tapGestureRecognizer)
    }
    \end{code}

    Sollte erfolgreich eine Tippgeste auf der Scene View registriert werden, so wird die Methode \texttt{didTap(\_:)} im Code \ref{code:prot-gesture-didTap} ausgeführt. Bei dieser Methode werden die 2D Koordinaten der Tippgeste ausgelesen und in einem Hit-Test verwendet. Dieser Hit-Test prüft ob es sich um eine SceneKit Node handelt. Falls dies der Fall ist, wird dieses Element selektiert und als interagierbares Objekt gesetzt. Dies geschieht mit der Methode \texttt{updateSelectedObject(node:)}. Darin werden auch zwei \texttt{SCNTransaction} ausgeführt damit der Benutzer anhand einer visuellen Animation erkennt, welches das aktuell selektierte Objekt ist. Mit \texttt{SCNTransaction} können komplexe Animationen erstellt und innerhalb einer Transaktion ausgeführt werden.

    \begin{code}{prot-gesture-didTap}{Methode \texttt{didTap(\_:)}, die beim einmaligen Antippen ausgeführt wird}
    func didTap(_ recognizer: UIGestureRecognizer) {
        let tapLocation = recognizer.location(in: sceneView)
        let hitTestOptions: [SCNHitTestOption: Any] = [.boundingBoxOnly: true]
        let hitTestResults = sceneView.hitTest(tapLocation, options: hitTestOptions)
        guard let node = hitTestResults.first?.node else {return}
        updateSelectedObject(node: node)
    }
    \end{code}

    \textbf{\texttt{UITapGestureRecognizer} mit zweimaligem Tippen}\\
    Das zweimalige Antippen der Scene View setzt einen neuen Würfel. Das Prinzip um dass zweimalige Antippen zu registrieren ist gleich wie im obigen Abschnitt, ausser dass die Anzahl der Tippesten auf zwei erhöht wird:
    \mint[style=xcode,breaklines]{swift}{doubleTapGestureRecognizer.numberOfTapsRequired = 2}

    Als auszuführende Methode wird \texttt{didDoubleTap(\_:)} gesetzt. Bei der Methode (Code \ref{code:prot-gesture-didDoubleTap}) wird wie beim einfachen Antippen geprüft, ob ein SceneKit Objekt beim Hit-Test getroffen wird. Ist dies nicht der Fall können wir den neuen Würfel platzieren. Dies wird mit der Methode \texttt{addBox(x:y:z:)} gemacht. Die Koordinaten des Hit-Test werden hierfür von einer Matrix zu x, y, z konvertiert. Dies wurde anhand einer Extension gemacht (Code \ref{code:prot-gesture-extension}).

    \begin{code}{prot-gesture-didDoubleTap}{Methode \texttt{didDoubleTap(\_:)}, die beim einmaligen Antippen ausgeführt wird}
    func didDoubleTap(_ recognizer: UIGestureRecognizer) {
        let tapLocation = recognizer.location(in: sceneView)
        let hitTestResults = sceneView.hitTest(tapLocation)
        guard let node = hitTestResults.first?.node else {
            let hitTestResultWithFeaturePoints = sceneView.hitTest(tapLocation, types: .estimatedHorizontalPlane)
            if let hitTestResultWithFeaturePoints = hitTestResultWithFeaturePoints.first {
                let translation = hitTestResultWithFeaturePoints.worldTransform.translation
                addBox(x: translation.x, y: translation.y, z: translation.z)
            }
            return
        }
    }
    \end{code}

    \begin{code}{prot-gesture-extension}{Extension einer float4x4 Matrize um die Koordinaten in x, y und z zurückzugeben}
    extension float4x4 {
        var translation: float3 {
            let translation = self.columns.3
            return float3(translation.x, translation.y, translation.z)
        }
    }
    \end{code}

    \textbf{\texttt{UILongPressGestureRecognizer} für langes Antippen eines Objektes}\\
    Die neu erstellten Würfel sollen auch wieder entfernt werden können. Dies wird als langes Antippen eines Würfels mit \texttt{UILongPressGestureRecognizer} realisiert. Dieser muss wie bei den obigen Beispielen der Scene View hinzugefügt werden. Anschliessend wird ein \texttt{UILongPressGestureRecognizer} registriert mit der Funktion \texttt{didLongPress(\_:)} (Code \ref{code:prot-gesture-didLongPress}), die bei einem langen Antippen aufgerufen wird. Hier findet wie bei den obigen Beispielen ein Hit-Test statt. Es wird geprüft ob sich an der lang angetippten Stelle ein Würfel befindet. Sollte dies der Fall sein, wird die Methode \texttt{removeFromParentNode()} aufgerufen und das Element von der Scene View entfernt (Zeile 6).

    \begin{code}{prot-gesture-didLongPress}{Methode \texttt{didLongPress()}, die bei einem langen Antippen ausgeführt wird}
    func didLongPress(\_ recognizer: UILongPressGestureRecognizer) {
        let longPressLocation = recognizer.location(in: sceneView)
        let hitTestResults = sceneView.hitTest(longPressLocation)
        guard let node = hitTestResults.first?.node else { return }
        node.removeFromParentNode()
    }
    \end{code}

    \textbf{\texttt{UISwipeGestureRecognizer} für die Rotationen}\\
    Die Rotation eines Würfels sollte mit einer Wischgeste ermöglicht werden. Die Wischgeste für links, rechts, oben und unten funktioniert mit dem \texttt{UISwipeGestureRecognizer}. Im Code \ref{code:prot-gesture-addSwipeGestureToSceneView} wird eine der Gesten initialisiert. Wichtig hierbei ist, dass jeweils die Richtung der Wischgeste mittels \texttt{swipeRightGesture.direction} gesetzt wird. Als auszuführende Aktion kann anschliessend die Methode \texttt{didSwipe(\_:)} angegeben werden.

    \begin{code}{prot-gesture-addSwipeGestureToSceneView}{Methode \texttt{addSwipeGestureToSceneView()} um die Wischgesten hinzuzufügen}
        func addSwipeGestureToSceneView() {
            let swipeRightGesture = UISwipeGestureRecognizer(target: self, action: #selector(didSwipe(_:)))
            swipeRightGesture.direction = .right
            sceneView.addGestureRecognizer(swipeRightGesture)
            // ...
        }
    \end{code}

    \textbf{Drehen von Würfeln anhand der akutellen Kameraposition}\\
    Beim Drehen der Würfel bestehen zwei Schwierigkeiten. Einerseits soll der Würfel immer relativ zur Kamera nach oben oder unten gekippt werden. Es muss also entschieden werden, ob die Rotation zur X-Achse oder Z-Achse erfolgt. Zusätzlich dreht beim Rotieren eines Würfels das relative Koordinatensystem mit. Dies bedeutet, dass nach dem Kippen eines Elements sein relatives Koordinatensystem ebenfalls anders positioniert ist und beim nächsten Kippen um eine andere Achse rotiert werden muss. 

    Das erste Problem kann gelöst werden indem anhand der aktuellen Kameraposition die Y-Achse ausgewertet wird. Je nach Winkel kann somit entschieden werden um welche Achse der Würfel sich drehen sollte (Code \ref{code:prot-gesture-didSwipe}, Zeile 5).

    Das zweite Problem kann mittels Überschreiben des relativen Würfelkoordinatensystems gelöst werden. Nach der Rotation muss das relative Koordinatensystem des Würfels zurückgesetzt werden (Zeile 13).

    \begin{code}{prot-gesture-didSwipe}{Methode \texttt{didSwipe(\_:)} die bei Wischgesten ausgeführt wird}
    func didSwipe(_ gesture: UISwipeGestureRecognizer) {
        let currentAngle = sceneView.session.currentFrame?.camera.eulerAngles.y
        var action: SCNAction!
        if gesture.direction == .up {
            if (currentAngle! > Float(0.785) && currentAngle! < Float(2.356)) {
                action = SCNAction.rotateBy(x: 0, y: 0, z: CGFloat(Double.pi/2), duration: 0.5)
            }
            // ...
        }
        // ...
        if action != nil {
            selectedObject?.runAction(action, forKey: "rotate")
            selectedObject?.orientation = SCNQuaternion(x: 0.0, y: 0.0, z: 0.0, w: 1.0)
        }
    }
    \end{code} 

\end{description}    
