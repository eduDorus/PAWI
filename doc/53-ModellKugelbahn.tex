\subsubsection{Virtuelles Modell einer Kugelbahn projizieren}\label{subsub:prot-kugelbahn}
\begin{description}
	\item[Fragestellung:] Wie kann eine gesamte Kugelbahn modelliert und als Ganzes in Augmented Reality auf eine Fläche projiziert werden?
	\item[Resultat:] Durch einen hierarchischen Aufbau von SceneKit Nodes, kann eine Kugelbahn aus mehreren einzelnen Elementen modelliert und als Ganzes modifiziert werden (bspw. verschieben, drehen). Eine Klasse, die von \texttt{SCNNode} erbt, enthält und verwaltet die Elemente als Kindknoten. Der \texttt{SCNBillboardConstraint} von SceneKit vereinfacht zudem das Orientieren von Objekten zur Kamera hin. % TODO: Verweis auf entsprechenden Prototyp (Xcode Projekt) im Repo, bzw. vollständigen Code im Anhang(?)
	\item[Versuchsaufbau:] Da zum Zeitpunkt dieses Versuchs keine vollständige 3D Modelle von echten cuboro Elemente vorhanden waren, wurde der Versuch mit einer \texttt{BasicCube} Klasse auf der Grundlage von Prototyp \ref{subsub:prot-overlay} aufgebaut. Dieser Versuch bildete auch die Basis für die hier verwendete Flächenerkennung, -auswahl und das Platzieren der Kugelbahn.

	\textbf{Tracking Statusanzeige} \\ % TODO: dieser Abschnitt war zwar Teil dieses Versuchs, gehört aber Thematisch irgendwie nicht hierher?
	Die Methode \texttt{session(\_:cameraDidChangeTrackingState:)} wird bei jeder Änderung des Tracking Status durch das Framework aufgerufen. Das erhaltene \texttt{ARCamera} Objekt hat ein Attribut \texttt{trackingState} mit den möglichen Zuständen \texttt{normal}, \texttt{notAvailable} und \texttt{limited}. Für letzteres gibt es zudem eine der folgenden Begründungen:
	\begin{itemize}
		\item \texttt{initializing}: Die Session hat noch nicht genügend Informationen für das Tracking. Diese Phase dauert auf einem iPhone 6s in der Regel ein paar Sekunden beim Start der Session.
		\item \texttt{insufficientFeatures}: Die sichbare Szene enthält zu wenige Features. Dies ist beispielweise bei einer einfarbigen Tischoberfläche ohne Musterung der Fall. Solche Tischflächen als Grundlage für eine ARKit Szene zu nutzen ist daher schwierig. Sie werden meistens nicht erkannt. Auch eine ungenügende Beleuchtung des Raumes kann Ursache für diesen Zustand sein.
		\item \texttt{excessiveMotion}: Das Gerät bewegt sich zu stark, sodass präzises Feature Tracking nicht möglich ist.
		\item \texttt{relocalizing}: Die Session wurde unterbrochen und es wird versucht den vorherigen Zustand wiederherzustellen.
	\end{itemize}
	Um den Tracking State anzuzeigen, wird ein Label mit entsprechendem Text gesetzt. Damit Labels im Interface Builder zusammen mit der Scene View genutzt werden können, müssen die Elemente gegenüber dem Standard AR Projekt von Xcode in einer gemeinsamen \texttt{UIView} als Container gesetzt werden.

	\textbf{Kugelbahn modellieren}\\
	Eine neue eigene Klasse \texttt{MarbleTrack} hält alle Informationen zu einer Kugelbahn und bietet Methoden zur Manipulation der Kugelbahn an. Die Klasse erbt von \texttt{SCNNode} und kann damit direkt als Node in einer AR Szene genutzt werden. Im Wesentlichen soll die Klasse aus ganzzahligen Koordinaten eine Kugelbahn mit \texttt{BasicCube} Nodes als Kindknoten zusammenstellen. Die Koordinaten eines Elements beschreiben seine Position in Anzahl Würfel, ausgehend von einem Basiselement in (0,0,0). Die Achsenrichtungen korrespondieren mit denen der ARKit Session.

	Ein Array von Tupeln mit den Koordinaten der drei Achsen bildet die Struktur für eine Kugelbahn. Folgender Code beschreibt zwei aufeinander stehende Elemente und ein drittes Element rechts daneben:
	\mint[style=xcode,breaklines]{swift}{let track = [(0,0,0), (0,1,0), (1,0,0)]}

	Beim Iterieren durch das Array wird für jedes Tupel ein \texttt{BasicCube} als Kindknoten hinzugefügt. Zur Positionierung werden die gegebenen Koordinaten mit der Seitenlänge des Elements multipliziert (Code \ref{code:prot-kugelbahn-addcube}, Zeile 3).

	\begin{code}{prot-kugelbahn-addcube}{Methode \texttt{addCube(x:y:z:)} um einen \texttt{BasicCube} als Kindknoten hinzuzufügen}
		func addCube(x: Int, y: Int, z: Int) -> BasicCube {
			let cube = BasicCube()
			let pos = SCNVector3(CGFloat(x) * cube.sidelength, CGFloat(y) * cube.sidelength, CGFloat(z) * cube.sidelength)
			cube.set(position: pos)
			addChildNode(cube)
			return cube
		}
	\end{code}

	\textbf{Kugelbahn projizieren}\\
	Ähnlich dem Vorgehen in Versuch \ref{subsub:prot-overlay}, wird zuerst die Flächenerkennung und -auswahl gemacht. Statt eines Positionierungswürfel wird hier nun aber eine \texttt{MarbleTrack} Node hinzugefügt. Die Verwendung von \texttt{eulerAngle} zur Rotation hatte sich nicht bewährt. Um ein Objekt stets zur Kamera hin zu orientieren, bietet SceneKit den Constraint \texttt{SCNBillboardConstraint}. Indem nur Y als frei bewegliche Achse gesetzt wird (Code \ref{code:prot-kugelbahn-billboardconstraint}, Zeile 3), bleibt die Kugelbahn in der Horizontalen, dreht sich aber zur Kamera. Die Constraints zu entfernen reicht aus, um die automatische Ausrichtung zu beenden (Zeile 9). Die Node bleibt dann in ihrer aktuellen Position stehen. Die beiden Methoden in Code \ref{code:prot-kugelbahn-billboardconstraint} werden aus dem View Controller bei \texttt{touchesBegan(\_:with:)} abwechslungsweise aufgerufen, wodurch zwischen den beiden Zuständen mittels Toucheingabe gewechselt werden kann.

	\begin{code}{prot-kugelbahn-billboardconstraint}{Verwendung von Constraints um Nodes zur Kamera auszurichten}
		func constraintToCamera() {
			let constraint = SCNBillboardConstraint()
			constraint.freeAxes = SCNBillboardAxis.Y
			constraints = [constraint]
		}
    
		func removeConstraints() {
			constraints = []
		}
	\end{code}

	% TODO: Screenshots einfügen der Drehung? Von einer platzierten Kugelbahn?

\end{description}
