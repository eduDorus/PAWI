\section{Architekturdokumentation}
\subsection{Funktionale Anforderungen}
\subsubsection{Epics}
Die App soll folgende \textbf{Epics} erfüllen:
\begin{enumerate}
	\item \textbf{Bauanleitung:} Der Benutzer kann von einer gewählten Kugelbahn eine Augmented Reality Bauanleitung erhalten, um die Bahn physisch nachbauen/aufbauen zu können.
	\item \textbf{Editor:} Der Benutzer kann in Augmented Reality eine virtuelle Kugelbahn neu erstellen oder eine bestehende verändern, damit diese als Bauanleitung zur Verfügung stehen.
\end{enumerate}

\subsubsection{User Stories}
\begin{enumerate}
	\setcounter{enumi}{-1}	% starte bei 0
	\item Allgemein:
	\begin{longtable}{l l p{12cm}}
		\hline
		\textbf{Nr.} & \textbf{Prio.} & \textbf{Beschreibung} \\
		\hline
		1 & & Allgemein \\
		\hline
		1.1 & M & Als Benutzer kann ich eine Fläche der realen Welt als Ebene für Augmented Reality auswählen, damit diese als Basis für die Anwendung verwendet wird. \\
		1.2 & M & Als Benutzer kann ich eine Kugelbahn auf der ausgewählten Ebene platzieren. \\
		1.3 & M & Als Benutzer kann ich die Kugelbahn auf einer Fläche ausrichten, damit sie wie gewünscht positioniert ist. \\
		1.4 & M & Als Benutzer will ich erkennen in welchem Modus (Bauanleitung, Editor) ich mich befinde. \\
		1.5 & S & Als Benutzer kann ich die Position und Ausrichtung einer Kugelbahn korrigieren, damit sie mit der physischen Kugelbahn übereinstimmt. \\
		1.6 & K & Als Benutzer kann ich zwischen den Modi wechseln, damit ich die aktuelle Bahn im anderen Modus direkt weiterverwenden kann. \\
		\hline
		2 & & Bauanleitung \\
		\hline
		2.1 & M & Als Benutzer kann ich eine gespeicherte Kugelbahn auswählen, um ihre Bauanleitung zu verwenden. \\
		2.2 & M & Als Benutzer sehe ich, welches cuboro Element als nächstes physisch platziert werden soll. \\
		2.3 & M & Als Benutzer kann ich in der Bauanleitung einzelne Elemente durchgehen, damit ich eine Bahn schrittweise aufbauen kann. \\
		2.4 & S & Als Benutzer kann ich in der Bauanleitung einzelne Elemente zurück gehen. \\
		2.5 & K & Als Benutzer kann ich die Bauanleitung neu starten. \\
		\hline
		3 & & Editor \\
		\hline
		3.1 & M & Als Benutzer kann ich 4 verschiedene cuboro Elemente in der App verwenden, um eine Bahn zu erstellen. \\
		3.2 & M & Als Benutzer kann ich eine neue Kugelbahn erstellen. \\
		3.3 & M & Als Benutzer kann ich eine Kugelbahn auf dem Gerät speichern, damit ich sie später wiederverwenden kann. \\
		3.4 & M & Als Benutzer kann ich eine gespeicherte Kugelbahn bearbeiten. \\
		3.5 & M & Als Benutzer sehe ich welche Positionen zur Wahl stehen, um ein cuboro Element hinzuzufügen. \\
		3.6 & M & Als Benutzer kann ich einer Kugelbahn ein neues cuboro Element hinzufügen. \\
		3.7 & M & Als Benutzer kann ich ein die Ausrichtung (Rotation) eines cuboro Elements verändern. \\
		3.8 & M & Als Benutzer kann ich ein cuboro Element von der Kugelbahn entfernen. \\
		3.9 & S & Als Benutzer werde ich darauf aufmerksam gemacht, wenn Elemente nicht aneinander passen, damit ich am Schluss eine funktionierende/zusammenhängende Kugelbahn habe. \\
		\hline
	\end{longtable}
\end{enumerate}

\subsubsection{Zielhierarchie}

\subsection{Nichtfunktionale Anforderungen}

	\begin{longtable}{l l l p{9cm}}
		\hline
		\textbf{Nr.} & \textbf{Prio.} & \textbf{Typ} & \textbf{Beschreibung} \\
		\hline
		1 & & Constraint & \\
		\hline
		1.1 & M & Technologie & Die App ist in Swift für iOS programmiert. \\
		1.2 & M & Technologie & Die App läuft auf aktuellen iPhones (ab iPhone 6s) mit iOS 11.3. \\
		1.3 & M & Technologie & Die App verwendet ARKit 1.5. \\
		1.8 & M & Geschäftlich & Die App zeigt die technologischen Möglichkeiten auf und muss keinen Business Value erzielen. \\
		\hline
		1 & & Quality & \\
		\hline
		1.4 & M & Laufzeit & Die App verwendet Farben, die auch für Personen mit Sehschwächen (z. B. Rot-Grün-Schwäche) erkenn- und unterscheidbar sind. \\ 
		1.5 & M & Laufzeit & Der Benutzer wird bei limitiertem Tracking Status über Massnahmen informiert. \\
		1.9 & M & Nicht Laufzeit & Klassen- und Variablennamen im Code sind aussagekräftig. \\
		1.10 & M & Nicht Laufzeit & Schlüsselstellen im Code sind gut kommentiert und verständlich aufgebaut. \\
		1.6 & M & Leistung & Beim Augmentieren von maximal 20 cuboro Elementen wird konstant mehr als 30 Frames pro Sekunde erreicht. \\
		1.7 & M & Verfügbarkeit, Sicherheit & Alle Daten der App werden nur lokal gespeichert (Offline). \\
		\hline
	\end{longtable}

\subsubsection{Szenarien}

\subsection{Fachdomäne (Domain Driven Design)}

\subsubsection{Fachbegriffe}
\begin{itemize}
	\item \textbf{Element:} Ein einzelner cuboro Kugelbahnwürfel mit keinem (ganzer Würfel), einem oder mehreren Wegen für eine Kugel.
	\item \textbf{Kugelbahn:} Ein zusammenhängendes Konstrukt aus einzelnen Elementen, sodass eine Kugel einen durchgehenden Weg durch die Bahn hat.
	\item \textbf{Ebene:} Eine horizontale Fläche, die von ARKit als solche erkannt wird und als Untergrund für die Kugelbahn dient.
	\item \textbf{Bauanleitung:} Eine interaktive Anleitung, die den Benutzer schrittweise (Element um Element) durch den Aufbau einer Kugelbahn führt.
	\item \textbf{Tracking:} Der Vorgang, bei dem ARKit versucht Merkmale der realen Welt anhand Kamerabilder und Gerätesensoren zu verfolgen und anhand dessen die virtuellen Objekte ausrichtet.
\end{itemize}

\subsubsection{Entitäten und ihre Value Objects}
\begin{itemize}
	\item \textbf{Kugelbahn}
	\begin{itemize}
		\item Ausrichtung
		\item Position
		% \item Elemente?
	\end{itemize}
	\item \textbf{Element}
	\begin{itemize}
		\item Ausrichtung
		\item Position
		\item Geometrie
	\end{itemize}
\end{itemize}

\subsubsection{Domain Events}

\subsection{Systemkontext}
\subsection{Systemvision und -idee}
\subsubsection{Systemvision}

Für Kugelbahnenthusiasten, cuboro Fans und solche die es noch werden wollen,
die ihre Kugelbahn mitnehmen wollen, planen wollen und nach Anleitung nachbauen wollen,
bietet unsere Kugelbahn App
mit Augmented Reality
eine komforable, moderne und effiziente Methode seine Bahn zu planen oder zu bauen.
Anders als das offizielle cuboro Webkit,
ist die Kugelbahn mit der App mobil und man kann sie virtuell in einen realen Raum stellen.

For fans of marbleruns and cuboro and everyone who wants to become one,
who want to take their marbleruns with them, plan them and rebuild them with a building guide,
our marblerun mobile app
with augmented reality
offers comfortable, modern and efficient methods to plan and build a marblerun.
Compared to the official cuboro webkit,
with the app a marblerun is mobile and a user can place them virtually in a real scene.

\subsubsection{Systemidee}
\subsection{Architekturprinzipien}
\subsection{Taktiken}
\subsection{Architekturstil}
\subsection{EAI Pattern}
\subsection{Systemsichten}
\subsection{Architekturentscheidungen}
\subsection{Architekturbewertung mit ATAM}
