\section{Architekturdokumentation}
\subsection{Funktionale Anforderungen}
\subsubsection{Epics}
Die App soll folgende \textbf{Epics} erfüllen:
\begin{enumerate}
	\item \textbf{Bauanleitung (Guide):} Der Benutzer kann von einer gewählten Kugelbahn eine Augmented Reality Bauanleitung erhalten, um die Bahn physisch nachbauen/aufbauen zu können.
	\item \textbf{Baumodus (Builder):} Der Benutzer kann in Augmented Reality eine virtuelle Kugelbahn neu erstellen oder eine bestehende verändern, damit diese als Bauanleitung zur Verfügung stehen.
\end{enumerate}

\subsubsection{User Stories}

\begin{longtable}{l l p{13cm}}
	\hline
	\textbf{Nr.} & \textbf{Prio.} & \textbf{Beschreibung} \\
	\hline
	\textbf{1} & & \textbf{Allgemein} \\
	\hline
	1.1 & M &
		\begin{tabular}[t]{@{}p{13cm}@{}}
			Als Benutzer kann ich eine Fläche der realen Welt als Ebene für Augmented Reality auswählen, damit diese als Basis für die Anwendung verwendet wird. \\
			Akzeptanzkriterien:
			\begin{itemize}
				\item Von ARKit erkannte Flächen werden als rechteckige Flächen dargestellt.
				\item Durch antippen einer Fläche wird diese ausgewählt und alle anderen Flächen ausgeblendet und gelöscht.
				\item Die vertikale Position einer neu zu setzenden Kugelbahn orientiert sich an der ausgewählten Fläche, bzw. kommt darauf zu stehen.
			\end{itemize} \vspace*{-\baselineskip}
		\end{tabular} \\
	\hline
	1.2 & M &
		\begin{tabular}[t]{@{}p{13cm}@{}}
			Als Benutzer kann ich eine Kugelbahn auf der ausgewählten Ebene platzieren. \\
			Akzeptanzkriterien:
			\begin{itemize}
				\item Wenn eine Fläche ausgewählt ist, wird darauf eine Kugelbahn angezeigt.
				\item Die Kugelbahn befindet sich stets zentral im Kamerabild auf der Fläche.
				\item Durch Benutzereingabe, kann die Bahn an der aktuellen platziert werden.
			\end{itemize} \vspace*{-\baselineskip}
		\end{tabular} \\
	\hline
	1.3 & M & 
		\begin{tabular}[t]{@{}p{13cm}@{}}
			Als Benutzer kann ich die Kugelbahn auf einer Fläche ausrichten, damit sie wie gewünscht positioniert ist. \\
			Akzeptanzkriterien:
			\begin{itemize}
				\item Die Kugelbahn ist während des Platzierens stets zur Kamera ausgerichtet, sodass eine Drehung des Geräts auch die Rotation der Kugelbahn gegenüber der realen Welt verändert.
			\end{itemize} \vspace*{-\baselineskip}
		\end{tabular} \\
	\hline
	1.4 & M & 
		\begin{tabular}[t]{@{}p{13cm}@{}}
			Als Benutzer will ich erkennen in welchem Modus (Bauanleitung, Editor) ich mich befinde. \\
			Akzeptanzkriterien:
			\begin{itemize}
				\item Auf dem AR Bildschirm wird unmissverständlich angezeigt, welcher Modus aktiv ist.
			\end{itemize} \vspace*{-\baselineskip}
		\end{tabular} \\
	\hline
	1.5 & S & 
		\begin{tabular}[t]{@{}p{13cm}@{}}
			Als Benutzer kann ich die Position und Ausrichtung einer Kugelbahn korrigieren, damit sie mit der physischen Kugelbahn übereinstimmt. \\
			Akzeptanzkriterien:
			\begin{itemize}
				\item Die Position der Kugelbahn lässt sich über Bedienelemente in Richtung aller drei Achsen verschieben.
			\end{itemize} \vspace*{-\baselineskip}
		\end{tabular} \\
	\hline
	1.6 & K & 
		\begin{tabular}[t]{@{}p{13cm}@{}}
			Als Benutzer kann ich zwischen den Modi wechseln, damit ich die aktuelle Bahn im anderen Modus direkt weiterverwenden kann. \\
			Akzeptanzkriterien:
			\begin{itemize}
				\item Es besteht die Möglichkeit den aktiven Modus zu wechseln.
				\item Beim Wechsel bleibt die aktuelle Szene (Kugelbahn und Position) bestehen, die Fläche muss nicht neu erkannt und die Kugelbahn nicht neu gesetzt werden.
			\end{itemize} \vspace*{-\baselineskip}
		\end{tabular} \\
	\hline
	\textbf{2} & & \textbf{Guide} \\
	\hline
	2.1 & M & 
		\begin{tabular}[t]{@{}p{13cm}@{}}
			Als Benutzer kann ich eine gespeicherte Kugelbahn auswählen, um ihre Bauanleitung zu verwenden. \\
			Akzeptanzkriterien:
			\begin{itemize}
				\item Alle gespeicherten Kugelbahnen werden angezeigt und lassen sich auswählen.
				\item Die Kugelbahn, die für die Bauanleitung gesetzt wird entspricht der ausgewählten, gespeicherten Kugelbahn.
			\end{itemize} \vspace*{-\baselineskip}
		\end{tabular} \\
	\hline
	2.2 & M & 
		\begin{tabular}[t]{@{}p{13cm}@{}}
			Als Benutzer sehe ich, welches cuboro Element als nächstes physisch platziert werden soll. \\
			Akzeptanzkriterien:
			\begin{itemize}
				\item Das als nächstes zu bauende Element ist deutlich von den anderen Elementen hervorgehoben.
			\end{itemize} \vspace*{-\baselineskip}
		\end{tabular} \\
	\hline
	2.3 & M & 
		\begin{tabular}[t]{@{}p{13cm}@{}}
			Als Benutzer kann ich in der Bauanleitung einzelne Schritte vorwärts gehen, damit ich eine Bahn schrittweise aufbauen kann. \\
			Akzeptanzkriterien:
			\begin{itemize}
				\item Es gibt ein Bedienelement um den nächsten Schritt der Anleitung auszulösen.
			\end{itemize} \vspace*{-\baselineskip}
		\end{tabular} \\
	\hline
	2.4 & S & 
		\begin{tabular}[t]{@{}p{13cm}@{}}
			Als Benutzer kann ich in der Bauanleitung einzelne Elemente zurück gehen. \\
			Akzeptanzkriterien:
			\begin{itemize}
				\item Es gibt ein Bedienelement um zum vorhergehenden Schritt der Anleitung zurück zu gehen.
			\end{itemize} \vspace*{-\baselineskip}
		\end{tabular} \\
	\hline
	2.5 & K & 
		\begin{tabular}[t]{@{}p{13cm}@{}}
			Als Benutzer kann ich die Bauanleitung neu starten. \\
			Akzeptanzkriterien:
			\begin{itemize}
				\item Die Anleitung lässt sich direkt vom aktuellen Stand zurück auf den ersten Schritt zurücksetzen.
				\item Beim zurücksetzen bleibt die Position und Ausrichtung der Kugelbahn bestehen, es ist kein neu setzen notwendig.
			\end{itemize} \vspace*{-\baselineskip}
		\end{tabular} \\
	\hline
	\textbf{3} & & \textbf{Builder} \\
	\hline
	3.1 & M & 
		\begin{tabular}[t]{@{}p{13cm}@{}}
			Als Benutzer kann ich mindestens 4 verschiedene cuboro Elemente in der App verwenden, um eine Bahn zu erstellen. \\
			Akzeptanzkriterien:
			\begin{itemize}
				\item Bei der Auswahl des Elements stehen mindestens 4 unterschiedliche Typen zur Verfügung.
			\end{itemize} \vspace*{-\baselineskip}
		\end{tabular} \\
	\hline
	3.2 & M & 
		\begin{tabular}[t]{@{}p{13cm}@{}}
			Als Benutzer kann ich eine neue Kugelbahn erstellen und benennen. \\
			Akzeptanzkriterien:
			\begin{itemize}
				\item Bei der Wahl der Kugelbahn gibt es eine Option, eine neue Bahn zu erstellen.
				\item Beim Erstellen einer neuen Kugelbahn erhält der Benutzer die Möglichkeit einen Namen einzugeben.
				\item Die neu erstellte Kugelbahn erscheint in der Liste der Kugelbahnen mit ihrem Namen.
			\end{itemize} \vspace*{-\baselineskip}
		\end{tabular} \\
	\hline
	3.3 & M & 
		\begin{tabular}[t]{@{}p{13cm}@{}}
			Als Benutzer kann ich eine Kugelbahn auf dem Gerät speichern, damit ich sie später wiederverwenden kann. \\
			Akzeptanzkriterien:
			\begin{itemize}
				\item Die aktuelle Bahn kann durch den Benutzer gespeichert werden.
				\item Beim nächsten Abruf der Bahn erscheint diese im Zustand, in dem sie gespeichert wurde.
			\end{itemize} \vspace*{-\baselineskip}
		\end{tabular} \\
	\hline
	3.4 & M & 
		\begin{tabular}[t]{@{}p{13cm}@{}}
			Als Benutzer kann ich eine gespeicherte Kugelbahn bearbeiten. \\
			Akzeptanzkriterien:
			\begin{itemize}
				\item Der Benutzer kann eine gespeicherte Bahn auswählen.
				\item An der gewählten Bahn können Änderungen vorgenommen werden (Hinzufügen, Verändern oder Entfernen von Elementen).
			\end{itemize} \vspace*{-\baselineskip}
		\end{tabular} \\
	\hline
	3.5 & M & 
		\begin{tabular}[t]{@{}p{13cm}@{}}
			Als Benutzer sehe ich welche Positionen zur Wahl stehen, um ein Element hinzuzufügen. \\
			Akzeptanzkriterien:
			\begin{itemize}
				\item Valide Positionen für neue Elemente sind nach der Wahl des zu hinzufügenden Elements dargestellt.
				\item Valide Positionen sind nur solche, die physikalisch möglich sind (keine schwebende Elemente) und direkt an ein bestehendes Element anschliessen.
			\end{itemize} \vspace*{-\baselineskip}
		\end{tabular} \\
	\hline
	3.6 & M & 
		\begin{tabular}[t]{@{}p{13cm}@{}}
			Als Benutzer kann ich einer Kugelbahn ein neues Element an einer validen Position hinzufügen. \\
			Akzeptanzkriterien:
			\begin{itemize}
				\item Eine zur Wahl stehende Position kann ausgewählt werden, um an dieser Stelle das Element zu setzen.
				\item Das Platzieren abseits hervorgehobener Positionen ist nicht möglich.
			\end{itemize} \vspace*{-\baselineskip}
		\end{tabular} \\
	\hline
	3.7 & M & 
		\begin{tabular}[t]{@{}p{13cm}@{}}
			Als Benutzer kann ich ein die Ausrichtung (Rotation) eines Elements verändern. \\
			Akzeptanzkriterien:
			\begin{itemize}
				\item Ein Element kann ausgewählt werden und wird hervorgehoben.
				\item Durch Streichgesten kann das selektierte Element entlang einer der drei Achsen in Richtung der Geste um 90° gedreht werden.
			\end{itemize} \vspace*{-\baselineskip}
		\end{tabular} \\
	\hline
	3.8 & M & 
		\begin{tabular}[t]{@{}p{13cm}@{}}
			Als Benutzer kann ich ein Element von der Kugelbahn entfernen. \\
			Akzeptanzkriterien:
			\begin{itemize}
				\item Ein einzelnes Element kann von der Kugelbahn entfernt werden, solange die Kugelbahn zusammenhängend bleibt und physikalisch möglich ist (keine schwebende Elemente).
			\end{itemize} \vspace*{-\baselineskip}
		\end{tabular} \\
	\hline
	3.9 & S & 
		\begin{tabular}[t]{@{}p{13cm}@{}}
			Als Benutzer werde ich darauf aufmerksam gemacht, wenn Elemente nicht aneinander passen, damit ich am Schluss eine funktionierende/zusammenhängende Kugelbahn habe. \\
			Akzeptanzkriterien:
			\begin{itemize}
				\item Die App überprüft nach jeder Änderung der Kugelbahn, ob Rillen und Löcher der Elemente zusammenpassen und es einen durchgängigen Weg für die Kugel gibt.
				\item Elemente, die nicht zusammenpassen werden hervorgehoben oder markiert.
				\item Unpassende Elemente müssen nicht für das Funktionieren der App korrigiert werden, da es eine bewusste Entscheidung des Benutzers sein kann.
			\end{itemize} \vspace*{-\baselineskip}
		\end{tabular} \\
	\hline
\end{longtable}

\subsubsection{Zielhierarchie}
Abbildung \ref{fig:zielhierarchie-cockburn} zeigt einen Überblick über die Zielhierarchie der Anforderungen, zugeordnet nach den drei Farbstufen nach Cockburn.
\bild{0.9}{zielhierarchie-cockburn}{Zielhierarchie nach Cockburn}

\subsection{Nichtfunktionale Anforderungen}

\begin{longtable}{l l l p{10cm}}
	\hline
	\textbf{Nr.} & \textbf{Prio.} & \textbf{Typ} & \textbf{Beschreibung} \\
	\hline
	 & & Constraints & \\
	\hline
	01 & M & Technologie & Die App ist in Swift für iOS programmiert. \\
	02 & M & Technologie & Die App läuft auf aktuellen iPhones (ab iPhone 6s) mit iOS 11.3. \\
	03 & M & Technologie & Die App verwendet ARKit 1.5. \\
	04 & M & Geschäftlich & Die App zeigt die technologischen Möglichkeiten auf und muss keinen Business Value erzielen. \\
	\hline
	 & & Qualitäten & \\
	\hline
	05 & M & Laufzeit & Die App verwendet Farben, die auch für Personen mit Sehschwächen (z. B. Rot-Grün-Schwäche) erkenn- und unterscheidbar sind. \\ 
	06 & M & Laufzeit & Der Benutzer wird bei limitiertem Tracking Status über Massnahmen informiert. \\
	07 & M & Laufzeit & Beim Augmentieren von maximal 20 cuboro Elementen wird konstant mehr als 30 Frames pro Sekunde erreicht. \\
	08 & M & Laufzeit & Alle Daten der App werden nur lokal gespeichert (Offline). \\
	09 & M & Compilier-Zeit & Klassen- und Variablennamen im Code sind aussagekräftig. \\
	10 & M & Compilier-Zeit & Schlüsselstellen im Code sind gut kommentiert und verständlich aufgebaut. \\
	\hline
\end{longtable}

\subsubsection{Szenarien}

Die Tabelle \ref{tab:szenario-nfa-06} zeigt ein Szenario der Benutzbarkeit anhand der nichtfunktionalen Anforderung 06.
\begin{table}
	\begin{tabular}{r p{12cm}}
		\hline
		Teil des Szenarios & Wert \\
		\hline
		Quelle des Auslösers & System (ARKit) \\
		Auslöser/Ereignis & Die Kamera ändert ihren Tracking Status \\
		Umgebung & ARMarbleRun App \\
		Systembestandteil & Gesamtsystem \\
		Antwort/Reaktion & Tracking Status wird dem Benutzer angezeigt und eine mögliche Massnahme zur Behebung des Problems vorgeschlagen \\
		Antwortmetrik & Der Benutzer weiss, warum das Tracking nicht wie gewünscht funktioniert und kann darauf reagieren \\
		\hline
	\end{tabular}
	\caption{Szenario für die nichtfunktionale Anforderung 06}
	\label{tab:szenario-nfa-06}
\end{table}

Tabelle \ref{tab:szenario-nfa-07} zeigt das Szenario für die nichtfunktionale Anforderung 07.

\begin{table}
	\begin{tabular}{r p{12cm}}
		\hline
		Teil des Szenarios & Wert \\
		\hline
		Quelle des Auslösers & Benutzer \\
		Auslöser/Ereignis & Der Benutzer setzt das 20. Element \\
		Umgebung & ARMarbleRun App \\
		Systembestandteil & Gesamtsystem \\
		Antwort/Reaktion & Die Applikation wird optimiert, damit 20 Elemente zu keinen Einbussen unter 30 Bilder pro Sekunde führen. \\
		Antwortmetrik & Die App läuft mit mindestens 30 Bilder pro Sekunde, wenn 20 Elemente einer Kugelbahn angezeigt werden. \\
		\hline
	\end{tabular}
	\caption{Szenario für die nichtfunktionale Anforderung 07}
	\label{tab:szenario-nfa-07}
\end{table}

\subsection{Fachdomäne (Domain Driven Design)}

\subsubsection{Fachbegriffe}

\begin{itemize}
	\item \textbf{Element:} Ein einzelner cuboro Kugelbahnwürfel mit keinem (ganzer Würfel), einem oder mehreren Wegen für eine Kugel.
	\item \textbf{Kugelbahn:} Ein zusammenhängendes Konstrukt aus einzelnen Elementen, sodass eine Kugel einen durchgehenden Weg durch die Bahn hat.
	\item \textbf{Ebene:} Eine horizontale Fläche, die von ARKit als solche erkannt wird und als Untergrund für die Kugelbahn dient.
	\item \textbf{Bauanleitung:} Eine interaktive Anleitung, die den Benutzer schrittweise (Element um Element) durch den Aufbau einer Kugelbahn führt.
	\item \textbf{Tracking:} Der Vorgang, bei dem ARKit versucht Merkmale der realen Welt anhand Kamerabilder und Gerätesensoren zu verfolgen und anhand dessen die virtuellen Objekte ausrichtet.
\end{itemize}

\subsubsection{Entitäten und ihre Value Objects}

\begin{itemize}
	\item \textbf{Kugelbahn}
	\begin{itemize}
		\item Name
		\item Ausrichtungswinkel
		\item Positionskoordinaten
	\end{itemize}
	\item \textbf{Element}
	\begin{itemize}
		\item ID
		\item Ausrichtungswinkel
		\item Positionskoordinaten
		\item Geometrie
	\end{itemize}
\end{itemize}

\subsubsection{Domain Events}

\subsection{Systemkontext}
\subsection{Systemvision und -idee}
\subsubsection{Systemvision}

Für Kugelbahnenthusiasten, cuboro Fans und solche die es noch werden wollen,
die ihre Kugelbahn mitnehmen wollen, planen wollen und nach Anleitung nachbauen wollen,
bietet unsere Kugelbahn App
mit Augmented Reality
eine komforable, moderne und effiziente Methode seine Bahn zu planen oder zu bauen.
Anders als das offizielle cuboro Webkit,
ist die Kugelbahn mit der App mobil und man kann sie virtuell in einen realen Raum stellen.

For fans of marbleruns and cuboro and everyone who wants to become one,
who want to take their marbleruns with them, plan them and rebuild them with a building guide,
our marblerun mobile app
with augmented reality
offers comfortable, modern and efficient methods to plan and build a marblerun.
Compared to the official cuboro webkit,
with the app a marblerun is mobile and a user can place them virtually in a real scene.

\subsubsection{Systemidee}
\subsection{Architekturprinzipien}
\subsection{Taktiken}
\subsection{Architekturstil}
\subsection{EAI Pattern}
\subsection{Systemsichten}
\subsection{Architekturentscheidungen}
\subsection{Architekturbewertung mit ATAM}
