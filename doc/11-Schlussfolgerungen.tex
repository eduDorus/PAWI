\section{Schlussfolgerungen}

\subsection{Lessons Learned}
Mit der erstellung der VIPER Architektur wurde kostbare Zeit aufgewendet. Diese Architektur ist für grössere Projekte durchaus geeignet jedoch konnten wir beim erstellen des Prototyps auf eine interative Architektur schliessen um mehr Funktionalität zu entwickeln. Zusätzlich wird von Apple das MVC Prinzip bevorzugt, wobei die meisten Beispiele diese Architektur besitzen.

Das Framework ARKit zeigt bereits was die Zukunft bringen könnte, braucht jedoch noch einen gewissen Umfang an Funktionalität.
% TODO: erweitern. Hab da grad en bissl schreib blockade

\subsubsection{Persönliches Fazit von Dorus Janssens}
Das Industrieprojekt war für mich ein sehr spannender Einblick in die iOS Welt. Es war spannend die Swift Programmiersprache und Xcode als Entwicklungsumgebung kennenzulernen. Die einarbeitung hat aber auch viel Zeit und nerven gebraucht. 

\subsubsection{Persönliches Fazit von Lucas Schnüriger}


\subsection{Ausblick}

Anbei werden mögliche Weiterführungen dieser Arbeit beschrieben.

\subsubsection{Objekterkennung}
Im Kapitel \ref{subsub:prot-boundingbox} wird ein Konzept vorgestellt, welches die Erkennung von einem Element innerhalb einer Bounding Box beinhaltet. Diese automatische Erkennenung könnte den Ablauf beim Aufbau der Kugelbahn verbessern. Um diese Funktionalität umzusetzen muss recherchiert werden, wie solche Hit-Tests durchgeführt werden können. Zusätzlich muss die Genauigkeit der Hit-Tests evaluiert werden damit das Element in der Bounding Box richtig erkannt wird.

\subsubsection{Kugelsimulation}
Im Kapitel \ref{sub:scene-kit} wird erläutert, dass das SceneKit eine eingebaute physik Engien besitzt. Diese könnte als weiterführende Arbeit an der Demo-App implementiert werden. Dazu müssen die Beschaffenheiten und Geometrie der Elemente hinterlegt werden. Anschliessend kann die Funktionalität entwickelt werden eine virtuelle Kugel auf die Bahn zu platzieren. Mit der Interaktion einer virtuellen Kugel kann die Bahn die vollständige Funktionalität der physischen Kugelbahn abbilden.

\subsubsection{ARKit 2.0 und iOS 12}
Kurz vor Ende dieser Arbeit wurde die Entwicklerkonferenz WWDC2018 abgehalten. An dieser Konferenz war Augmented Reality ein ausgeprägtes Thema. Mit dem neuen iOS 12 Release wurde ebenfalls die neue ARKit Version 2.0 vorgestellt. ARKit 2.0 bringt eine eingebaute Funktionalität um dreidimensionale Objekte zu erkennen, bis zu vier Personen simultan mit der gleichen augmentierten Welt zu interagieren und eine verbesserte Leistung bei der Erkennung von horizontalen sowie vertikalen Flächen. Zusätzlich wurde ein neues Dateiformat vorgestellt, dass das Einfügen, Bearbeiten und Austauschen von dreidimensionalen Objekten vereinfacht. 

Mit der neuen ARKit Version 2.0 könnte das setzen der virtuellen Bahn mit dem Erkennen eines physischen Elements ersetzt werden. Dies würde zwar bedeuten, dass es notwendig ist eine Bahn zu besitzen jedoch kann der Grundstein oder die Bahn physikalisch ausgerichtet werden und denn Prozess für den Endanwender interaktiver gestalten.
